\begin{disclaimer}
	Large parts of this chapter are based on \ccite{Stoll:2021ori,Koenigstein:2021llr}. The individual sections include more detailed disclaimers. The involved collaborators are J. Stoll, N. Zorbach, A. Koenigstein, and S. Rechenberger as well as L. Pannullo and M. Winstel.
	
	\Ccite{Stoll:2021ori} contains results from the Master's theses~\cite{Stoll:2021thesis,Zorbach:2021thesis} of J. Stoll and N. Zorbach and is also discussed in A. Koenigstein's dissertation~\cite{Koenigstein:2022phd}. J. Stoll, N. Zorbach, A. Koenigstein, and me contributed in equal shares to this work. S. Rechenberger was involved in early stages of this project before leaving academia and contributed to the final draft.
	
	 Preliminary analytical and numerical computations for \ccite{Koenigstein:2021llr} were performed by S. Rechenberger (see, \eg{}, \ccite{Rechenberger:2018talk}) before he left academia but prior to \ccite{Koenigstein:2021llr} they were not published. The authors of \ccite{Koenigstein:2021llr} completed the computations and wrote \ccite{Koenigstein:2021llr} as a ``pedagogical and detailed supplement, and completion of existing literature''. All authors contributed in equal shares to this work.
	
	The following introduction of this chapter has been compiled from the introductions and abstracts of \ccite{Stoll:2021ori,Koenigstein:2021llr}.
\end{disclaimer}
After our lengthy discussion of theories in zero space-time dimensions the time has finally come to discuss ``real'' \qft{}, \ie{}, theories which have a notion of space-time and actually deal with fluctuating quantum fields rather than scalars and Grassmann numbers. We have chosen to study the \twoDimensional{} \acrrepeat{gn} model~\cite{Gross:1974jv} for its relative simplicity, at least when compared with \qcd{} or even the \qmm{}.

In theoretical \hep{} the \gnm{} has been studied extensively as a toy model or testing ground to study effects like asymptotic freedom, (chiral) symmetry breaking and restoration, and inhomogeneous phases.
In this chapter we will study the \gnm{} and its variants with the \frg{} in \lpa{} within our developed \cfd{} framework for \frg{} flow equations. 
We study the model at finite $N$, exploring the role of bosonic fluctuations, and at infinite $N$, where bosonic fluctuations are completely suppressed.\bigskip

In \cref{sec:gny} we introduce the \gnm{} and the \gnym{} as a partially bosonized variant, which we will study with the \frg{}.
In \cref{sec:gny_frg} we adapt the \cfd{} methods, which we developed in \cref{chap:zeroONSU2}, to the \lpa{} flow equation of the \gnym{}.

To set up our \frg{} study of the \gnym{} at finite $N$ in \cref{sec:gnyFiniteN}, we first study the model in the infinite-$N$/mean-field limit.
We use the \frg{} in \lpa{} to recover the well known results for the homogeneous phase diagram at infinite $N$ in \cref{sec:gnInfHomo}.
Renormalization and asymptotic freedom are discussed as important aspects for the subsequent analysis at finite $N$.

At infinite $N$ the \gnm{} is renowned for its robust inhomogeneous phase, \cf{} \cref{subsec:inhomoLiterature} and \cref{fig:GNthisPD}.
We will use the established explicit, literature results of Michael Thies \etal{} to study inhomogeneous phases indirectly in \cref{sec:gnInfInhomo} using a stability analysis of the homogeneous phase.
Using the literature results as reference we assess the stability analysis and the related \acrrepeat{ggl} analysis both qualitatively and quantitatively.

We finish the discussions in this chapter with our \frg{} study of the \gnym{} at finite $N$ in \cref{sec:gnyFiniteN}.
We employ our \cfd{} numerics and perspective to discuss the role of fermionic and bosonic fluctuations in \frg{} flows at zero and non-zero temperature and chemical potential.
With this explicit \lpa{} study we address the long standing open research question: whether there is symmetry breaking at finite $N$ in the \gnm{} or not.
There are a lot of literature results on this topic including, heuristic arguments, large-$N$ studies, and even lattice results but the situation remained unclear.
The main result of our explicit computations is that bosonic thermal fluctuations vaporize the chiral condensate at any finite $N$: we do not find symmetry breaking at non-zero temperature and arbitrary chemical potential.

In \cref{sec:gnConclusion} we summarize our key research results of our studies with the \gnyBm{}.
Extending the present discussion beyond the \lpa{} is identified as the logical and necessary next step.

This chapter has two primary digital auxiliary files~\cite{Steil:2023GNnotebook,Steil:2023GNcpp}.
The \WAM{} notebook~\cite{Steil:2023GNnotebook} includes some of the symbolic and (semi-)analytic computations of this chapter, while the \Cpp{} code~\cite{Steil:2023GNcpp} was used for numerical computations in the infinite-$N$ limit.
