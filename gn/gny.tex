\section{The Gross-Neveu(-Yukawa) model}\label{sec:gny}
\begin{disclaimer}
	This section follows the discussion presented in Secs. I and II of \ccite{Stoll:2021ori}.
\end{disclaimer}
Since its original publication in 1974, the \acrrepeat{gn} model~\cite{Gross:1974jv} \dash{} a relativistic \qft{} of $N$ massless Dirac fermions that are self-interacting via the scalar channel of the four-Fermi interaction \dash{} was subject of intensive research \wrt{} various aspects of strongly interacting systems.
In $1 + 1$ space-time dimensions it was studied as a (perturbatively) renormalizable prototype model for asymptotic freedom~\cite{Gross:1974jv,Wetzel:1984nw,Rosenstein:1990nm,Gracey:1990sx,Gracey:1990wi,Gracey:1991vy,Luperini:1991sv,ZinnJustin:1991yn,Peskin:1995ev,ZinnJustin:2002ru,Quinto:2021lqn}, while for $2 + 1$ space-time dimensions it served as a toy model for asymptotically safe \qfts{}~\cite{Braun:2010tt} and is renormalizable with non-perurbative methods~\cite{Rosenstein:1988pt,Rosenstein:1988dj,Rosa:2000ju,Hofling:2002hj}.
In $3 + 1$ dimensions it mimics the dynamics of the important scalar channel of the \njlm{}~\cite{Nambu:1961fr,Nambu:1961tp} and is non-renormalizable, but can be seen as being part of a low-energy effective model of \qcd{}, \cf{} \cref{subsec:chiralLEFT}.
However, for the rest of this chapter, we exclusively focus on its formulation in $1 + 1$ space-time dimensions.

We will use the model as a testing ground to study the stability and \ggla{} for inhomogeneous phases at large $N$ and to apply our \frg{} \cfd{} methods, developed so far in zero dimensions only, to a real \qft{}.
As such we consider the \gnm{} as a toy model in the context of theoretical \hep{} but it has a multitude of applications beyond such a purely theoretical scope. 
We will briefly comment on some applications of the \gnm{} in the next paragraph but as mentioned in \customref{paragraph:introPhysics}{A disclaimer about physics} in \cref{chap:introduction}, we will not attempt to comment on the possible implications of our results for physical systems in this context. 
While certainly interesting and relevant, such a discussion is beyond the scope of the current work, since we are ultimately interested in studies in $3+1$ dimensions, \cf{} \cref{chap:QMM}, and we consider the results of this chapter as a stepping stone towards applications in $3+1$ dimensions.

\paragraph{Applications of the GN model for physical systems}\phantomsection\label{paragraph:gnApplications}\mbox{}\\%
The \gn{} has various connections to non-relativistic models in solid-state physics, see, \eg{}, Refs.~\cite{Schnetz:2004vr,Fradkin:2013}.
\gn{}-type four-fermi models can arise in the continuum limit (describing large distance physics) of one-dimensional solid-state systems.
Examples of such one-dimensional systems are quantum antiferromagnets (described by {spin\nobreakdash-$\tfrac{1}{2}$} Heisenberg models~\cite{Heisenberg:1928mqa}), interacting electrons in a one-dimensional conductor (described by a Tomonaga–Luttinger liquid~\cite{Tomonaga:1950,Luttinger:1963zz}), and polyacetylene polymer chains~\cite{Chodos:1993mf,Thies:2006ti} \dash{} ${(\mathrm{C}\mathrm{H})_x}$ \dash{} (described in the limit $x\rightarrow\infty$ with the Su-Schrieffer-Heeger model~\cite{Su:1979ua} or the subsequent Takayama-Lin-Liu-Maki model~\cite{Takayama:1980zz}).
Recently, a one-dimensional probabilistic cellular automaton, where classical bits can be interpreted as Ising spins, was shown to be equivalent to a relativistic fermionic quantum field theory~\cite{Wetterich:2021exk}.
A central concept behind the emergence of fermionic \qfts{} in two dimensions from spin-systems in one dimension is the mapping of spin operators onto fermionic creation and annihilation operators by means of the Jordan–Wigner transformation~\cite{Jordan:1928wi}, for a pedagogic discussion see \ccite{Fradkin:2013}.
\gn{}-type models also arise naturally in the continuum limit of two-dimensional spin-systems like the Ashkin-Teller model~\cite{Ashkin:1943zza}, which is related to the well-known Potts model~\cite{Potts:1951rk} and as such used to study various phenomena of solid-state physics, see, \eg{}, \ccite{Wu:1982} for a general overview. In the following, we will list some explicit connections between the \gnm{} and models used in solid-state physics.

The lattice field theoretical formulation of the \gn{} model in the limit $N\rightarrow\infty$ is equivalent to an Ising model~\cite{Affleck:1981bn}.

At finite $N$ the \gnm{} can be considered as the continuum limit of the $N$-color ($N$ Ising spin) Ashkin-Teller model~\cite{Ashkin:1943zza,Fradkin:1984,Shankar:1985zc}, which describes $N$ coupled Ising spins on a two-dimensional lattice~\cite{Shankar:1985zc}.

For $N=1$ the \gnm{} is equivalent to the Thirring model~\cite{Thirring:1958in,Witten:1978qu} due to Fierz identities, \cf{} \ccite{Peskin:1995ev,ZinnJustin:2002ru,Fradkin:2013}.
In the continuum limit the one-dimensional spin-$\tfrac{1}{2}$ Heisenberg model is equivalent to a $N = 1$ \gn{} (Thirring) model~\cite{Fradkin:2013}.
The Thirring model also arises in the infinite volume limit of the Luttinger model~\cite{Luttinger:1963zz} with strictly local interactions~\cite{Heidenreich:1980}, see \ccite{Fradkin:2013} for further details.
The massive Thirring model is equivalent to the sine-Gordon model~\cite{Coleman:1974bu,Delepine:1997bz}, which in turn is (among its other application in mathematical physics) the continuum-limit of the Frenkel–Kontorova model~\cite{Frenkel:431595}.
The latter is a simple model of a harmonic chain in a periodic potential known from solid-sate physics~\cite{Kivshar:2000}.
The equivalence of the Thirring model and sine-Gordon model is based on an Abelian bosonization transformation, see \ccite{Fradkin:2013,Delepine:1997bz} and references therein, which connects equivalent bosonic and fermionic two-dimensional quantum field theories.

Apart from research efforts in high-energy and solid-state physics the \gn{} model is also of interest in the context of holographic methods~\cite{Maldacena:1997re,Witten:1998qj} especially in the study of the AdS/CFT correspondence involving higher-spin fields, see \ccite{Giombi:2016ejx} for a recent review.

\paragraph{Phenomenology of the Gross-Neveu model}\phantomsection\label{paragraph:GN_pheno}\mbox{}\\%
A peculiar feature of the massless \gn{} model is, that at leading order of an $\tfrac{1}{N}$-expansion, thus in the infinite-$N$ limit (sometimes also referred to as 't~Hooft limit), the \gnm{} dynamically develops a mass gap for the fermions, which is associated with an anti-fermion-fermion condensate $\langle \Fpsib \, \Fpsi \rangle \neq 0$.
In turn, this results in the breakdown of the discrete chiral symmetry of the initial microscopic \uv{} theory.
The formation of a mass gap is a purely non-perturbative effect, see, \eg{}, the \ccite{ZinnJustin:1991yn,ZinnJustin:2002ru} for details, and is a prime example for \textit{dimensional transmutation} \dash{} the emergence of a dimensionful scale in a theory which has only dimensionless couplings in its \uv{} classical action, see, \eg{}, \ccite{Kleinert:2016,Peskin:1995ev,ZinnJustin:2002ru}.
Hence, by summing up loop-contributions of all orders in the four-Fermi coupling in a $\tfrac{1}{N}$-expansion the discrete chiral symmetry spontaneously breaks down and is absent in the macroscopic theory in the \ir{}.
In the partially bosonized version, this amounts to integrating out the fermion-loop contribution to the bosonic effective potential, which develops a non-trivial minimum in the \ir{} \dash{} the condensate~\cite{ZinnJustin:1991yn,ZinnJustin:2002ru,Rosenstein:1990nm,Luperini:1991sv,Quinto:2021lqn}.

Shortly after D.~J.~Gross and A.~Neveu had published their results~\cite{Gross:1974jv} on \csb{}, the question arose, to what extent condensate formation is stable against thermal effects due to non-zero temperature $T$ or an increase in density, induced by a non-zero quark chemical potential $\mu$.
Within the infinite-$N$ limit and allowing only for spatially homogeneous condensates, the answer to these questions was quickly settled and is remarkable~\cite{Dolan:1973qd,Harrington:1974te,Harrington:1974tf,Jacobs:1974ys,Dashen:1974xz,Dashen:1975xh,Wolff:1985av,Treml:1989,Pausch:1991ee,Karbstein:2006er}: The homogeneous phase diagram of the \gn{} model consists of a region, where the discrete chiral symmetry is broken and a region of vanishing chiral condensate, \cf{}\ \cref{fig:GNlargeN_PD}.
The phase-transition line between these regions splits up into a second-order phase transition (starting at $\mu = 0$ and some critical temperature $T_\mathrm{C} \neq 0$ and ending in a \cp{} with $\mu_\mathrm{CP} \neq 0$ and $T_\mathrm{CP} \neq 0$) and a first-order phase transition (starting at the critical point and ending on the $T = 0$ axis and some non-zero chemical potential $\mu_1$).
The entire phase diagram solely depends on a single dimensionful parameter, which is related to a renormalization condition, and can be chosen freely.
All other dimensionful quantities are fixed multiples of this parameter and choosing a different renormalization condition (fixing some other parameter) corresponds to simple rescalings of all dimensionful quantities, but does not change their ratios, the phenomenology and the phase diagram.\\

Notwithstanding these early successes, the discussion on the physics of the \gn{} model did not stop.
One of the assumptions, which has led to the above results, is the assumption of a spatially homogeneous condensation of the fermions.
Relaxing this assumption, but retaining the $N \rightarrow \infty$ limit, it was shown in \ccite{Thies:2003kk,Schnetz:2004vr,deForcrand:2006zz} that there are regions in the $\mu$-$T$-plane, where a spatially inhomogeneous but static condensate $\langle \Fpsib \, \Fpsi \rangle ( x ) \neq \mathrm{const}.$\ is energetically favored over homogeneous condensation including a vanishing of the condensate, \cf{} \cref{subsec:inhomoLiterature} and \cref{fig:GNthisPD}.
We will discuss those inhomogeneous condensates in more detail in \cref{sec:gnInfHomo}.

\subsection{The models in vacuum and at non-zero \texorpdfstring{$T$}{T} and \texorpdfstring{$\mu$}{mu}}\label{subsec:gnyTmu}
\begin{disclaimer}
	This subsection is compiled from Secs.~II.A and B and App.~B of \nbccite{Stoll:2021ori} and Sec.~II.A of \nbccite{Koenigstein:2021llr}.
\end{disclaimer}
In this subsection, we introduce the \gn{} model, its bosonized counterpart and the \gny{} model in Euclidean space-time.
We comment on its symmetries and its in-medium generalization for non-zero chemical potentials $\mu$ and temperatures $T$.

\paragraph{In vacuum}\phantomsection\label{paragraph:gnyVac}\mbox{}\\%
The \gnm{} in one spatial and one temporal dimension in Euclidean space-time is defined by its classical action, \cf{}\ \ccite{Gross:1974jv},
\begin{align}
	S_\mathrm{GN} [ \Fpsi,\Fpsib ] = \int \dif^{\,2}\! x \, \big( \Fpsib \, \gamma^\nu\partial_\nu \, \Fpsi - \tfrac{g^2}{2 N} \, ( \Fpsib \, \Fpsi )^2 \big) \, ,	\label{eq:gn-model}
\end{align}
where $\Fpsi$ is a $N$-component object in flavor space ($f = 1, 2, \dots, N$ and $N > 1$)\footnote{We explicitly exclude $N = 1$, where the \gn{} model is identical to the Thirring model~\cite{Thirring:1958in,Witten:1978qu}, which has a vanishing perturbative one-loop beta function and different phenomenology than the \gn{} model at $N>1$~\cite{Peskin:1995ev,ZinnJustin:2002ru}. Some details and References concerning the Thirring model can be found in the \customref{paragraph:gnApplications}{previous paragraph}.} and a two-component spinor in Dirac space.
Our conventions for spinors in \dtwo{} can be found in \cref{app:clifford2}.
The action \eqref{eq:gn-model} involves a kinetic term and a four-fermion-interaction term with the dimensionless positive coupling constant $g\in \Reals{}^+$.
	
Apart from the usual Euclidean space-time symmetries (translations and a rotation), the action \eqref{eq:gn-model} is invariant under transformations of the symmetry group ${G = U(1) \times SU(N) \times \mathds{Z}_2}$. The group acts on the fermion fields as follows 
\begin{subequations}\label{eq:gn-group-action}
\begin{align}
	G \times \Complex{}^{2 N} \to \, & \Complex{}^{2 N} \, ,\\[.1em]
	( ( \alpha, \theta, n ), \Fpsi ) \mapsto \, & \alpha \, ( \theta \, \gamma_+ + n \, \theta \, \gamma_- ) \, \Fpsi \, ,
\end{align}
\end{subequations}
with $n \in \{-1, +1\}$, left- and right-handed chiral projection operators $\gamma_{\mp}$ from \cref{eq:gammaChProjection}, and $\alpha \in U ( 1 )$ and $\theta \in SU ( N )$. 
The full group $G$ is defined as the direct product of the groups $U(1)$, $SU(N)$, and \ZII{}.
Hereby, the $SU(N)$ symmetry is usually called flavor or color symmetry and causes, according to Noether's theorem, the conservation of a vector current.
The \ZII{} symmetry is called discrete chiral symmetry.
The $U(1)$ symmetry is called phase symmetry and leads to a conserved Noether-charge density $\Fpsib \, \gamma^2 \, \Fpsi/N$, which is usually called baryon number density, which is tuned by the chemical potential $\mu$ for the fermions~\cite{Fitzner:2010nv,Dunne:2011wu,Thies:2017fkr,Lenz:2020cuv}, see also \cref{app:grandCanonicalPartitionFunction}.
It is also worth mentioning, that the $SU(N)$ symmetry group of the \gnm{} in $1 + 1$ space-time dimensions is an $O(N)$ symmetry, which is why it is sometimes also denoted as an $O(N)$-symmetric model~\cite{Jacobs:1974ys,ZinnJustin:2002ru}. Furthermore the \gn{} model in two dimensions has an additional hidden $O(2N)$ symmetry between Majorana components of the fermion fields~\cite{Dashen:1975xh} which prevents the appearance of different four-fermion interaction channels during renormalization~\cite{Luperini:1991sv}.\bigskip

The partition function of the \gn{} model is equivalent to the partition function of a \bgn{}, which can be derived by means of a \acrrepeat{hs} transformation~\cite{Stratonovich:1957,Hubbard:1959ub}.
One introduces a Gaussian integral over a bosonic auxiliary field $\FSff{\xi}$ using
\begin{align}
	1 = \mathcal{N} \int \mathcal{D}[ \FSff{\xi} \vts] \, \eu^{- \int \dif^{\,2}\! x \, \frac{\FSff{\xi}^2}{2g^2}} \, ,\label{eq:gnHStrafo}
\end{align}
with a normalization factor $\mathcal{N}$.
Combining this with the purely fermionic grand-canonical partition function, based on the action \eqref{eq:gn-model}, \cf{} \cref{app:grandCanonicalPartitionFunction}, we find
\begin{align}
	\mathcal{Z} \propto \, &  \int \mathcal{D}[ \FSff{\xi},\Fpsi,\Fpsib\vts]  \, \eu^{- \int \dif^{\, 2}\! x \, [ \Fpsib \, \gamma^\nu\partial_\nu \, \Fpsi - \frac{g^2}{2 N} \, ( \Fpsib \, \Fpsi )^2 + \frac{\FSff{\xi}^2}{2g^2} ]} \, .
\end{align}
Next, we shift the bosonic field integration variable
\begin{align}
	\FSff{\xi} = h \, \phi + \tfrac{g^2}{\sqrt{N}} \, \Fpsib \, \Fpsi \, ,
\end{align}
where we introduced the Yukawa coupling constant $h$ in order to have bosonic fields $\phi$\footnote{Following the convention used throughout \cref{chap:zeroONSU2} we use $\phi$ to denote a fluctuating field (not $\FSff{\phi}$) since we use $\varphi$ for the expectation value $\langle \phi\rangle=\varphi$.} with zero energy dimension, which is natural in two dimensions.
The real scalar field $\phi$ usually called ``auxiliary'' or ``constraint'' field~\cite{Harrington:1974tf,Jacobs:1974ys,Luperini:1991sv,Braun:2011pp}.
Using
\begin{align}
	\tfrac{1}{2 g^2} \, \FSff{\xi}^2 = \tfrac{h^2}{2 g^2} \phi^2 + \tfrac{h}{\sqrt{N}} \, \Fpsib \, \phi \, \Fpsi + \tfrac{g^2}{2 N} \, ( \Fpsib \, \Fpsi )^2 \, ,
\end{align}
we can completely eliminate the four-Fermi interaction term in favor of a Yukawa interaction term,
\begin{align}
	\mathcal{Z} \propto \, &  \int \mathcal{D}[\phi, \Fpsi,\Fpsib\vts] \, \eu^{- \int \dif^{\, 2}\! x \, [ \Fpsib \, ( \gamma^\nu\partial_\nu + \frac{h}{\sqrt{N}} \phi ) \, \Fpsi + \frac{h^2}{2 g^2} \phi^2 ]} \, .\label{eq:Zbgn}
\end{align}
This complete bosonization of the four-fermi interaction channel is reminiscent to the construction discussed in \cref{paragraph:qcdDynHad} in the context of dynamical hadronization in composite \frg{} flows of \qcd{}.
	
Using \cref{eq:Zbgn}, we define the action of the \bgn{} as
\begin{align}
	S_\mathrm{bGN} [\phi,\Fpsi,\Fpsib] = \int \dif^{\,2}\! x \, \big[ \Fpsib \, \big( \gamma^\nu\partial_\nu + \tfrac{h}{\sqrt{N}} \, \phi \big) \, \Fpsi + \tfrac{h^2}{2 g^2} \, \phi^2 \big] \, ,	\label{eq:bgn-model}
\end{align}
which is equivalent to the action \cref{eq:gn-model} of the original \gnm{}.
Equivalent in this context means, that both theories share the same correlation functions, see the discussion at the beginning of \cref{paragraph:GNbGNGNY} or the textbooks~\cite{ZinnJustin:2002ru,Peskin:1995ev} for additional details. 
	
In the bosonized version \eqref{eq:bgn-model}, the four-Fermi interaction is replaced by a Yukawa interaction term with coupling constant $h$ as well as a quadratic (mass) term $h^2/g^2$ for the auxiliary field $\phi$.
If we postulate
\begin{subequations}\label{eq:bGNchiralSym}
\begin{align}
	G \times \Reals{} \to \, & \Reals{} \, ,\\
	( ( \alpha, \theta, n ), \phi ) \mapsto \, & n \phi \, ,	
\end{align}
\end{subequations}
then the action \eqref{eq:bgn-model} is invariant under the same symmetry group $G$ as the original action \eqref{eq:gn-model}.
Within this work, we are especially interested in the discrete chiral symmetry transformation, which we understand as the group element $(\alpha, \theta, n) = (1, \Id_N, -1) \in G$, \ie{},
	\begin{align}
		\Fpsi \mapsto \Fpsi^\prime = \gamma_\mathrm{ch} \, \Fpsi \, ,	\qquad \Fpsib \mapsto \bar{\psi}^\prime = - \Fpsib \, \gamma_\mathrm{ch}\, ,\qquad	\phi \mapsto \phi^\prime = - \phi \, .\label{eq:discrete-chiral-transformation}
	\end{align}
It is this symmetry which prevents the \gnm{} from perturbatively generating a mass gap, see, \eg{}, \ccite{Gross:1974jv,ZinnJustin:2002ru}.

Correlation functions of the \gn{} and \bgn{} are linked through Ward-Takahashi identities.
Most notably among them: the expectation value of the scalar field can be related to the fermionic expectation value $\langle \bar{\psi} \, \psi \rangle$,
	\begin{align}
		\langle \phi \rangle \equiv \varphi = -\frac{g^2}{h N} \langle \Fpsib \, \Fpsi \rangle \, ,	\label{eq:sigm_eqofmotion} 
	\end{align}
see, \eg{}, \ccite{Pannullo:2019} for a derivation of this well known identity.
For this expectation value the discrete symmetry transformation \eqref{eq:discrete-chiral-transformation} is realized as follows,
	\begin{equation}
		\langle \Fpsib \, \Fpsi \rangle \mapsto - \langle \Fpsib \, \Fpsi \rangle \quad \Leftrightarrow \quad \varphi \mapsto -\varphi \, .
	\end{equation}
Since the expectation value of $\langle \phi \rangle \equiv \varphi$ is directly proportional to this condensate, a non-vanishing $\varphi$ implies a spontaneous breaking of the discrete chiral symmetry.\bigskip

By including an additional kinetic term for the bosonic field we obtain the \acrrepeat{gny} model~\cite{ZinnJustin:1991yn,ZinnJustin:2002ru,Rosa:2000ju}
\begin{align}
	S [ \phi ,\Fpsi,\Fpsib] = \, & \int \dif^{\,2}\! x \, \big[ \Fpsib \, \big( \gamma^\nu\partial_\nu + \tfrac{h}{\sqrt{N}} \, \phi \big) \, \Fpsi - \phi \, ( \Box \phi ) + \tfrac{h^2}{2 g^2} \, \phi^2 \big] \, ,		\label{eq:gny-model}
\end{align}
which we will use for practical computations in the \frg{} framework. We elaborate on the specific model choice and differences between the (b)\gn{} and \gnym{} in \cref{paragraph:GNbGNGNY}.

\paragraph{In medium}\phantomsection\label{paragraph:gnyTmu}\mbox{}\\%
In this work, we are mainly interested in the in-medium properties of the \gnym{}.
To work at non-zero baryon density, we fix the net baryon number density by introducing a quark chemical potential $\mu$ in the usual manner, \cf{} \cref{app:grandCanonicalPartitionFunction}, \viz{}
by subtracting
\begin{align}
	\mu N \equiv \mu \int\dif x \int_{0}^{\beta} \dif \tau \, \, \Fpsib \, \gamma^2 \, \Fpsi
\end{align}
from the classical \uv{} action, which yields the grand canonical partition function
\begin{align}
	\mathcal{Z} \propto \int \mathcal{D}[\phi,\Fpsi,\Fpsib\vts] \, \eu^{- S [\phi,\Fpsi,\Fpsib] + \mu N } \, .	\label{eq:GNpartition_function}
\end{align}

Furthermore, we introduce non-zero temperature via a compactification of the Euclidean time-direction in the usual manner, \cf{} \cref{app:grandCanonicalPartitionFunction,app:matsubaraSums} and specifically in this context App.~C of \nbccite{Stoll:2021ori}.

The aforementioned steps lead us to the in-medium or thermal \gnym{},
\begin{align}
	S[\phi, \Fpsi,\Fpsib] = \, & \int\dif x \int_{0}^{\beta} \dif \tau \, \big[ \Fpsib \, \big( \gamma^\nu\partial_\nu - \mu \, \gamma^2 + \tfrac{h}{\sqrt{N}} \, \phi \big) \, \Fpsi - \phi \, ( \Box \phi ) + \tfrac{h^2}{2 g^2} \, \phi^2 \big] \, .\label{eq:thermal_gny_action}
\end{align}