\clearpage
\section{Conclusion and outlook}\label{sec:gnConclusion}
\begin{disclaimer}
	This section has been compiled from Secs.~VII and IV of \nbccite{Stoll:2021ori,Koenigstein:2021llr} respectively.
\end{disclaimer}
In the introduction of this chapter, in \cref{subsec:phenomenology}, and with our literature review in \cref{subsec:z2breaking} \dash{} regarding the fate of \ZII{} symmetry breaking at finite $N$, we provided a broad phenomenological overview of the \gnm{}. Nevertheless, we have considered the \gnm{} within this chapter just as our next testing ground in \dtwo{} after leaving the roam of \dzero{}.
Within this chapter we have studied the \gnm{} at infinite-$N$ and the \gnym{} \dash{} as a closely related bosonized variant of the \gnm{} \dash{} at finite $N$.
Identifying the \gnym{} as a suitable candidate for an \lpa{} \frg{} study of the partially bosonized \gn{} model we considered its formulation in our \cfd{} frame work for the \frg{}

\paragraph{The LPA flow equation of the GNYM as a highly non-linear diffusion-source/sink equation}\phantomsection\label{paragraph:gnConSetup}\mbox{}\\%
Setting up the framework for numerical computation with the \lpa{} flow equation of the \gnym{}, included as a first step the reformulation of the flow equation in conservative form.
Using the previously established methodology we formulate the \lpa{} flow equation as a non-linear diffusion equation with a source/sink term.
The diffusive contributions can be clearly attributed to bosonic quantum fluctuations, of a massive radial mode, which we studied at length already in \dzero{}.
While we already encountered source/sink-like contributions from Grassmann numbers, the manifestation of the source/sink term in the \gnym{} as a real \qft{} with a chemical potential is novel to our discussion.
We have dedicated \cref{paragraph:chemical_potential_shock_wave} to a discussion of its dynamic: depending on \rgscale{}, location in field space, chemical potential, and temperature  it can either act like a source or a sink. Fermionic fluctuations can therefore both act towards a breaking and a restoration of \ZII{} symmetry.

\paragraph{The homogeneous phase at infinite $N$}\phantomsection\label{paragraph:gnConHomo}\mbox{}\\%
For infinite $N$ the bosonic fluctuations are completely suppressed within the \gnym{}, which is then equivalent to the \gnm{}.
Within the \frg{} framework in the limit $N \rightarrow \infty$ we recover well-known infinite-$N$ (mean-field) results for the homogeneous phase diagram of the \gnm{}. 
The discrete chiral \ZII{} symmetry is spontaneously broken at small chemical potentials and temperatures and gets restored at high temperatures and chemical potentials across a second-order (first-order) phase transition for high temperatures (low temperatures). The restoration of \ZII{} symmetry in the infinite-$N$ limit is purely driven by fermionic thermal and density fluctuations.

\paragraph{The inhomogeneous phases at infinite $N$}\phantomsection\label{paragraph:gnConInhomo}\mbox{}\\%
Before turning our attention to the \frg{} studies at finite $N$. We make use of the well established explicit results for the inhomogeneous phase in the \gnm{} to benchmark the stability analysis as one of the most popular indirect detection methods for inhomogeneous condensation.

It was shown in \cref{subsubsec:phase_diagram_stability_analysis} that the stability analysis is able to accurately predict the second-order phase transition between the \gls{ip} and \gls{sp} as the amplitude of the inhomogeneous condensate $\smin$ at this phase boundary becomes infinitesimal and its functional form is described by a harmonic wave.
Matching the initial expectation, the stability analysis fails to correctly detect the phase boundary between the \gls{hbp} and \gls{ip}, because large perturbations would be required.
Such perturbations can not be captured within the present expansion scheme.
Therefore, the region of the \gls{ip}, where $\sminhom (\mu, T)\neq 0$, is completely undetected by the stability analysis. 

Moreover, we compared the wave vector that minimizes the bosonic two-point function $\qmin $ with the dominating wave vector of the inhomogeneous condensate $\qs$ in \cref{subsubsec:wavevector}.
Inside the \gls{ip} close to the phase boundary between the \gls{sp} and \gls{ip} these two quantities agree very well.
Further away from this phase boundary, the amplitude of the inhomogeneous condensate is large, thus violating an assumption of the stability analysis.
This is reflected in a small but finite tension of $\qmin$ and $\qs$.

Also, the bosonic wave-function renormalization $Z$ was investigated.
The existence of a region where the wave-function renormalization is negative and the homogeneous minimum is stable under inhomogeneous perturbations, \ie{}, $Z<0$ and $\forall q, \ \gtwovar{\sminhom (\mu, T)}{\mu}{T}{q} >0$, explicitly shows that a negative $Z$ is only a necessary condition for an inhomogeneous phase.

In summary, these findings show that the stability analysis can indeed be an appropriate tool in the search for second-order phase boundaries of inhomogeneous phases.
By using the \twoDimensional{} \gls{gn} model as a test ground the shortcomings of this methods were quantified and it was demonstrated that it can also give a reasonable estimate of quantities within the inhomogeneous phase like the dominating wave vector of the condensate.\bigskip

In \cref{subsec:GNggl} we briefly discuss the related \ggla{} and identify it as a very simple jet potent tool in the vicinity of \glspl{lp}.

\paragraph{The Gross-Neveu-Yukawa model at finite \texorpdfstring{$N$}{N}}\phantomsection\label{paragraph:gnConFinN}\mbox{}\\%
Using the \frg{} framework and the established finite volume methods for it, we have performed computations for the \gnym{} at finite and infinite number of flavor $N$ at finite temperature $T\geq0$ and quark chemical potential $\mu\geq0$.\bigskip

At finite $N$ the \gny{} and \gnm{} are not equivalent in \lpa{} truncation. We argue however that, due to our specific choice for the classical action of the \gnym{}, the phenomenology of both at least on \lpa{} level should be similar. Direct computations with the \gn{} model in \lpa{} truncation within the \frg{} framework are not feasible. 
A proper resolution of the differences between the two models at finite $N$ would be possible by improving the truncation scheme. Especially the addition of a scale-dependent wave-function renormalization for the scalar $\sigma$-channel is a natural next step in this direction. First results in this direction are discussed in \ccite{Koenigstein:2023wso} and indicate that due to the intricate dynamics on the level of the two-point function, discussed here in the context of the stability analysis, higher orders in the derivative expansion might indeed be crucial to properly resolve the underlying \frg{} flow.

Numerical results for various finite $N$ and especially $N=2$ at non-zero temperature and/or chemical potential were presented and discussed in \cref{subsec:gnyFiniteNresults} and have revealed that there is no spontaneous \ZII{} symmetry breaking at non-zero temperature for finite $N$. This binary result is in agreement with heuristic arguments of L.~D.~Landau \etal{}~\cite{Landau:1980mil} and in the context of the \gn{} model of R.~F.~Dashen \etal{}~\cite{Dashen:1974xz}, which were summarized and discussed in \cref{subsec:z2breaking}. The situation at vanishing temperature is not completely settled yet. 

Direct computations in vacuum are numerically challenging but possible and suggest spontaneous \ZII{} symmetry breaking even for finite $N$ at $T=\mu=0$.
Computations at zero and very low temperatures and non-zero chemical potentials within the used \lpa{} flow equation are also challenging and arguably, see \cref{paragraph:chemical_potential_shock_wave}, impossible at zero temperature. This novel aspect of \lpa{} flow equations arising in the \cfd{} framework certainly warrants further research and development.
Direct computations beyond very low temperatures are however possible without conceptual or numerical challenges even at non-zero chemical potentials.
Considering the vacuum results and an extrapolation from results at low temperatures we have strong reasons to believe that a quantum phase transition between a phase of broken \ZII{} symmetry at low chemical potentials and a restored phase at higher chemical potentials is a highly likely scenario at $T = 0$ and finite $N$. 

\paragraph{Outlook}\phantomsection\label{paragraph:gnConOut}\mbox{}\\%
Beyond the aforementioned further research directions in regards to truncation and a more refined understanding of the involved sources, studies in related models in \dimPlus{1}{1} dimensions might be very interesting. Another extremely interesting research direction is the study of the \gnyBm{} in finite volumes.
 
Further remarks regarding the application of the more and more refined \cfd{} framework for \frg{} flow equations will be made in \cref{chap:conclusion}.
