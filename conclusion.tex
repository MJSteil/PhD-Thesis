We have already summarized the findings of our studies in \dzero{} and \dtwo{} in \cref{sec:0dconclusion} and \cref{sec:gnConclusion} respectively. 
At this point, we want comment on overarching concepts and key findings with a special focus on applications in \dfour{}.\bigskip

Our work and research has been focused on technical developments within the framework of the \frg{}.
With our research in zero and two dimensions we have developed a firm understanding of \frg{} flow equations, particularly of the \lpa{} flow equation, as conservation laws.
Within the framework of (numerical) fluid dynamics we have established an understanding of pionic modes and fluctuations as non-linear advective contributions.
While we identified radial/massive modes and fluctuations with non-linear diffusive contributions.
Fermionic quantum and thermodynamic fluctuations are understood as sources and sinks on the level of the \lpa{} flow.
We have observed highly non-linear dynamics during our studies of \frg{} flows both driven by internal and external factors.

The field of \cfd{} did not only provide the powerful and robust \fv{} method of our choosing, namely the \kt{} scheme, but it also provided a potent language to concisely describe the dynamics and interplay of different fluctuations on the level of the flow equation.
Especially the notion of (numerical) entropy and characteristic curves can provide quite unique insights.
We established a connection between numerical entropy and the inherent irreversible nature of the \frg{}, which manifests itself clearly on the level of the flow equation.

We have further developed and studied tools for  both the direct and indirect detection of inhomogeneous phases. We find ourselves in a very promising position. With a firm understanding of the stability analysis as an indirect method and a \lpa{} flow equation for an explicit inhomogeneous condensate as a direct method, we hope to use them in conjunction to gain novel insights into inhomogeneous phases beyond mean-field.
The adaptation of the developed \cfd{} approach for the \lpa{} flow equation to the \qmm{} seems to be the logical next step.
First one would establish contact with recent results, \eg{}, \ccite{Ihssen:2023xlp}, using the \fv{} method as an alternative to the employed discontinuous Galerkin methods~\cite{Grossi:2019urj,Grossi:2021ksl,Ihssen:2023xlp}. 
This would provide a valuable cross-check of both approaches. 
Hopefully gaining control over the homogeneous computations to a sufficient level one could turn towards inhomogeneous phases  using both the homogeneous stability analysis and the explicitly inhomogeneous \cdw{} condensate.
The latter, in particular, will still involve more research and development especially when it comes to the involved momentum integrals.\bigskip

Our developments with \rgct{} \mf{} studies, parameter fitting, and \Poincare{} invariance deserve more attention and space than we gave them in this work.
They are a rather unique combination and culmination of different results from the field of mean-field studies in the context of the \frg{}.
We plan to finish \ccite{Steil:2024RGMF} in the near future before any further new developments as it has been put of for too long already.
Even for this thesis the plan was a more detailed discussion of the involved concepts but again the preceding discussions in zero and two dimensions took the space.\bigskip

Talking about space it seems only appropriate to come back to zero-dimensional theories at this point.
I want to again stress the wealth of knowledge hidden in these simple theories.
Our work with Grassmann numbers in zero-dimensions is still in its infancy. 
I personally have high hopes when it comes to the study of Grassmann numbers with the \frg{}. I think there is a lot of potential for developments for fermion-boson systems using just Grassmann numbers and scalars in zero dimensions.