% Individual commands to generate standalone abstract
\renewcommand{\twoDimensional}{\texorpdfstring{${(1 \mkern-3mu + \mkern-3mu 1)}$-dimensionalen}{(1+1)-dimensionalen}}
\renewcommand{\fourDimensional}{\texorpdfstring{${(3 \mkern-3mu + \mkern-3mu 1)}$-dimensionalen}{(3+1)-dimensionalen}}
\renewcommand{\csb}{\texorpdfstring{$\chi$SB}{χSB}}
%
In dieser Arbeit untersuchen wir stark wechselwirkende Quantenfeldtheorien unter Verwendung der funktionalen Renormierungsgruppe (FRG) als unsere primäre Rechenmethode.
Das Ziel ist es, FRG-Rechnungen im Kontext der Quantenchromodynamik (QCD) zu ermöglichen, um die Phasenstruktur von dichter, stark wechselwirkender Materie zu studieren.
Der Hauptteil dieser Arbeit ist in drei Kapitel unterteilt, die sich in der Raumzeitdimension der betrachteten Theorien unterscheiden.

Wir beginnen mit dem Studium von null-dimensionalen Theorien, was letztendlich bedeutet, gewöhnliche Integrale mit komplizierten FRG-Flussgleichungen zu lösen.
Dies mag zunächst wie eine unnötig umständliche Methode erscheinen, um ein einfaches Problem zu lösen. 
Es ist jedoch genau diese Tatsache \dash{} die Anwendung der FRG auf solch einfache Theorien \dash{} die es uns auf rigorose Weise ermöglicht, enorme Einblicke in die FRG zu gewinnen.
Die wohl relevanteste Entwicklung ist das neuartige Verständnis der FRG-Flussgleichungen im Kontext der Fluiddynamik.
Dies ermöglicht die Anwendung von Methoden und Konzepten aus dem hochentwickelten Bereich der numerischen Strömungsmechanik (CFD) auf die FRG.
Zwei Schlüsselerkenntnisse sind die Identifizierung von bosonischen (fermionischen) Fluktuationen als konvektive (Quellen- oder Senken-artige) Beiträge zum FRG-Fluss und die daraus resultierende Verknüpfung des CFD-Konzepts der numerischen Entropie und der Irreversibilität von nicht-perturbativen Renormierungsgruppen (RG) Flüssen.
Diese Entwicklungen stellen einen entscheidenden Schritt zur Ermöglichung der folgenden Anwendungen dar.

Wir fahren fort mit Berechnungen im \twoDimensional{} Gross-Neveu (GN) Modell.
Wir verwenden es, um spontane chirale Symmetriebrechung (\csb{}) zu untersuchen \dash{} ein Phänomen, das für das Verständnis von QCD von entscheidender Bedeutung ist.
Mit den zuvor etablierten CFD-Methoden für die FRG untersuchen wir die Auswirkungen von fermionischen und insbesondere bosonischen Quanten- und thermodynamischen Fluktuationen auf spontane \csb{}. 
Das Hauptergebnis dieses Teils unserer Forschung ist, dass thermische bosonische Fluktuationen \csb{} im \twoDimensional{} GN Modell verhindern.
Wir untersuchen des Weiteren inhomogene \csb{} indirekt mittels einer Stabilitätsanalyse in \textit{Mean-Field} (MF) Näherung, d.h. nur unter Berücksichtigung fermionischer Fluktuationen.
Mit unserer Forschung tragen wir dazu bei, diese Methode als robustes Werkzeug für sowohl qualitative als auch quantitative Aussagen über inhomogene \csb{} zu etablieren.

Wir schließen den Hauptteil dieser Arbeit mit unseren Studien zum \fourDimensional{} Quark-Meson (QM) Modell ab, welches wir hauptsächlich als eine effektive Niederenergietheorie von QCD betrachten.
Wir konzentrieren uns auf inhomogene chirale Kondensate, indem wir das QM Modell im Rahmen der FRG untersuchen und dabei einen positionsabhängigen Ansatz für das chirale Kondensat verwenden, namentlich die chirale Dichtewelle (CDW), für die wir explizite FRG-Flussgleichungen ableiten konnten.
Wir untersuchen erneut die Auswirkungen von Fluktuationen auf spontane \csb{}, indem wir diese Flussgleichungen im Rahmen von RG-konsistenten MF Rechnungen lösen.
Dadurch stellen wir eine Verbindung zu bestehenden Literaturergebnissen für das QM Modell mit CDW Kondensaten her.
Diese Berechnungen \dash{} die derzeit nur fermionische Fluktuationen berücksichtigen \dash{} sind ein erster Schritt hin zu einer vollständigen Lösung der abgeleiteten Flussgleichungen unter Verwendung unserer etablierten CFD-Methoden.