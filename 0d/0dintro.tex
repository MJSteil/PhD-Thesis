\begin{disclaimer}
	Large parts of this chapter are based on \ccite{Koenigstein:2021syz,Koenigstein:2021rxj,Steil:2021cbu,Steil:partIV,Koenigstein:fixedPoint}.
	The individual sections include more detailed disclaimers.
	The involved collaborators are A. Koenigstein, N. Wink, E. Grossi, J. Braun, M. Buballa, and D. H. Rischke. 

	\Ccite{Koenigstein:2021syz,Koenigstein:2021rxj,Steil:2021cbu} are published manuscripts and parts of a series discussing zero-dimensional \ON{} models in the context of the \frg{}. The discussed research is also part of the dissertation~\cite{Koenigstein:2022phd} of A. Koenigstein. Most symbolic calculations, all numerical computations, and the majority of these manuscripts have been prepared by A. Koenigstein and me in equal shares. The other co-authors have been primarily involved in the conceptualization, discussion, and the finishing of the manuscripts.
		
	\Ccite{Steil:partIV} is an unpublished manuscript draft, which has been planed as a continuation and supplement to the series on \frg{} in zero dimensions.
	\Ccite{Koenigstein:fixedPoint} are unpublished notes on fixed-point solutions in the context of \ccite{Koenigstein:2021rxj}.
	
	The following introduction of this chapter has been compiled from the introductions and abstracts of \ccite{Koenigstein:2021syz,Koenigstein:2021rxj,Steil:2021cbu,Steil:partIV}.
\end{disclaimer}

We begin the main part of this thesis with the detailed discussion of our research regarding theories in zero space-time dimensions.
As demonstrated beautifully by Jan Keitel and Lorenz Bartosch with their work~\cite{Keitel:2011pn} and by numerous other authors, see, \eg{}, \ccite{\zerodRefs}, zero-dimensional theories are by no means trivial.
While it is true that they are easily solvable by just computing simple, ordinary integrals, their treatment within perturbative and non-perturbative methods employed in \qft{} is non-trivial.
Applying the \frg{}, \dses{}, nPI techniques, $\tfrac{1}{N}$-expansion schemes, and/or perturbative methods to zero-dimensional theories can be very illuminating.
Not only academic and didactic insights can be gained, but also deep conceptual and methodological developments are indeed possible with such studies.\bigskip 

As outlined in the introduction~\ref{paragraph:introZeroD}, the main purpose of the following discussions of this chapter is twofold.
\begin{enumerate}
	\item Using zero-dimensional theories, we want to gain academic, didactic, and conceptual insight into the \frg{} framework.
	\item Making use of the unique properties of \qfts{} in zero dimensions, we want to adapt and benchmark numerically stable methods for the solution of \frg{} flow equations.
\end{enumerate}
To achieve these goals we will apply and adapt methods from the field of \cfd{}, which we introduced in \cref{sec:conservationLaws}, to the \frg{}, which we introduced in \cref{sec:FRG}.

Zero-dimensional \qfts{} are uniquely well suited for such discussions, since the functional flow equations encountered in the \frg{} manifest directly as \pdes{} in zero dimensions.
It is possible to study the \frgEq{} and the related flow \cref{eq:dWkdtflow,eq:dZkdtflow} for $W_k[\FSff{J}]$ and $Z_k[\FSff{J}]$ directly, \ie{}, without the need of any truncations or approximations.
The flow equations can be readily expressed as conservation laws, allowing for a direct adaptation of concepts and methods from the field of \cfd{} to \frg{}.
Applications in zero dimensions make the somewhat vague notion of functional flow equations explicit as flow equations for functions. 
This supports and extends on the findings~\cite{Grossi:2021ksl} of Eduardo Grossi and Nicolas Wink and firmly establishes the very explicit notion of \frg{} flow equations as flow equations in a fluid-dynamical sense.
The exact flow equations encountered in zero dimensions share many crucial qualitative and, to an extent, even quantitative features with flow equations encountered in non-zero dimensions.
As such, our research and development in zero dimensions is far more than just an academic exercise, since it is very relevant and to a significant extent directly applicable in non-zero dimensions.\bigskip

We begin this chapter with a detailed discussion of zero-dimensional \qft{} using the $O(1)$ model of a single scalar as an instructive example in \cref{sec:0dQFT}.
We discuss general peculiarities of such \qfts{} in a single space-time point, but then shift focus especially on the manifestation of \frg{} in zero dimensions.
The \frg{} can be understood very intuitively  as an integral deformation \dash{} making the underlying idea of Wilson's \rg{} approach and the governing equations very tangible.
We retrace the general derivation of the \frgEq{} of \cref{sec:FRG} in zero-dimensions commenting in detail on the involved subtleties.
The fact that this discussion is based on functions and integrals instead of functionals and functional integrals allows for a simpler, but at the same time in many regards more concise discussion.\bigskip

In \cref{sec:0dON} we extend our discussion to zero-dimensional $O(N)$ models, where we allow for condensation in one radial direction leading to one massive \sigmaMode{} and $(N-1)$ \pionModes{}.
We use this model to study $O(N)$ symmetry restoration, by studying the \frg{} flow for various initial conditions/actions.
Basic symmetries of the underlying integrals prevent symmetry breaking in the \ir{} in zero dimensions.
This can be seen as an extreme limiting case of the \cmwhTheoremWithRefs{}, \cf{} \MWApp{}.

We will discuss the \frg{} flow evolution equation for the zero-dimensional $O(N)$ model at length.
Starting by casting it in conservative form, we continue to discuss initial conditions, boundary conditions, \rgcy{}, and irreversibility at length.
The conservative formulation allows for an identification of the \sigmaMode{} as diffusive contribution and of the \pionModes{} as advective contributions.

In \cref{subsec:0dONresults} we construct a series of test cases (initial conditions/actions) in the zero-dimensional $O(N)$ model at finite $N$.
For our explicit numerical computations  with the \frg{} flow equation, we adopt the \kt{} scheme of \cref{subsec:hydroKT} as our finite volume method of choice.
We discuss the advective and diffusive nature of \frg{} flows in detail using explicit numerical computations.
The impact of initial scales in the context of \rgcy{}, boundary conditions, and discretization in field space is discussed at length.
We also discus the, as it turns out rather limited, applicability of the \frg{} Taylor expansion as a possible truncation scheme in zero dimensions.

After our discussions of various finite $N$ results, we focus on the limiting cases $N=1$ and $N\rightarrow\infty$ in \cref{subsec:0dO1Entropy} and \cref{subsec:0dLargeN} respectively.
The focus of our discussion of the purely diffusive system at $N=1$ is the irreversibility of \grg{} flows from a \cfd{} perspective and the associated concept of (numerical) entropy.
A connection between the later and the concept of $\mathcal{C}$-/$\mathcal{A}$-functions is also discussed.
At large $N$ and ultimately in the limit $N\rightarrow\infty$ we study advection dominated and in the limit ultimately purely advective systems. 
The focus here are limitations of the large-$N$ saddle-point approximation and related \frg{}/\cfd{} concepts.
For our discussion we construct yet another test case, which we study using both numeric and analytic methods.
Shocks and rarefaction waves and their implications for the large-$N$ limit are discussed, including again comments on the irreversibility of \grg{} flows in such scenarios.\bigskip

In the penultimate \cref{sec:0dSU2} of this chapter we discuss Grassmann numbers as zero-dimensional analogs to fermions.
We present a $SU(2)$ model including two pairs of associated Grassmann numbers and three scalars, which we constructed as a zero-dimensional analog to the \qmm{}.
We discuss the model and the involved flow equations and comment at length on our plans for further research with such theories.
Compared to our, for the most parts, complete discussion of scalars in zero-dimensions, this work on Grassmann numbers is in a very early state.\bigskip

In \cref{sec:0dconclusion} we summarize our key research results of our extensive studies in zero dimensions and give an outlook for even further research with zero-dimensional \qfts{}.
Especially zero-dimensional models involving Grassmann numbers are identified as a very interesting and relevant area for further research.