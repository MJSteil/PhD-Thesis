\section{Quantum \texorpdfstring{\xcancel{field}}{(field)} theory in zero dimensions}\label{sec:0dQFT}
\begin{disclaimer}
	This section follows the discussion presented in Sec. II of \ccite{Koenigstein:2021syz}.
\end{disclaimer}
This section provides an introduction to zero-dimensional \qft{} using the theory of a single scalar as an instructive example.
We will focus on details and structure of the flow equations and the technical subtleties in their solution in zero dimensions thus without any direct reference to regularization and renormalization.
In addition we use this introduction to establish some notation and special features of zero-dimensional field theory.\\

As already mentioned in the introduction to this chapter, the efficient and sufficiently precise calculation of correlation functions is key to understanding the properties of a particular model or theory.
Usually this is done by introducing a partition function or functional integral that provides a probability distribution for the microstates of the model and serves as a generating functional for the $n$-point (correlation) functions~\cite{Weinberg:1996kr,Peskin:1995ev,ZinnJustin:2002ru,Kleinert:2004ev}.
The partition function is based on an energy function that can be a discrete or continuous Hamilton function or an action, which determines the microscopic properties of the model, \cf{} \cref{eq:ZkDef}.
The \frg{} provides an alternative to a direct computation using the functional integral.
For generic \qfts{} we introduced this approach in \cref{sec:FRG} but in this section we want to focus on the special case of a zero-dimensional \qft{} of a single scalar.
We will discuss the direct computation of observables using the generating functional and the alternative approach using the \frg{} flow equation.

\subsection{The partition function}\label{subsec:partition_function}
Consider a zero-dimensional \qft{} with a single real bosonic scalar field $\phi$\footnote{
Throughout this chapter we will use $\phi$ denote a fluctuating scalar (not $\FSff{\phi}$) since we will use $\varphi$ for the expectation value $\langle \phi\rangle=\varphi$.
Later in \cref{sec:0dSU2}, we will also introduce $\Ftheta$ and $\Fthetab$ as fluctuating Grassmann numbers with the corresponding expectation values $\langle\Ftheta\rangle=\MFtheta$ and $\langle\Fthetab\rangle=\MFthetab$.
Sadly variant versions are not available for all Greek characters, which led us to introduce the tilde-overline in our general discussion of \cref{sec:FRG}.
}.
In zero dimensions the ``field'' $\phi$ is due to the complete absence of a notion of space-time in zero dimensions mathematically not a field but just a single scalar degree of freedom, \ie{}, a plain real number \dash{} hence the typographic pun with striking out the word \textit{field} in the title of this section.
In our following discussion we will however maintain the term field even though mathematically we are just discussing numbers.
Due to the absence of space-time in zero dimensions derivatives and space-time integrals simply do not exist.
This implies that the action $\mathcal{S}[\phi]$ of the model is identical to the Lagrangian $\mathcal{L}[\phi]$.
The action, the Lagrangian, and also the Hamiltonian $\mathcal{H}[\phi]$ are simply functions of $\phi$ instead of functionals%
\footnote{%
	Nevertheless, we will stick to the notation of functionals using square brackets, to maintain a degree of consistency with the corresponding expressions in non-zero space-time dimensions, as long as we do not focus on particular zero-dimensional examples.
}.
The complete absence of space-time derivatives/integrals, fields, and functionals makes the following discussion mathematically rather simple in stark contrast to the situation encountered by \qft{} practitioners in $d>0$: where dealing with potentially divergent space-time integrals and complicated functional integrals is the norm.
This simplicity is the beauty of zero-dimensional \qfts{} which are by no means trivial: a lot can be learned from their study.
Because of the absence of a space-time derivative and thus of kinetic terms, $\mathcal{S}[\phi] = \mathcal{L}[\phi] = \mathcal{H}[\phi] = U(\phi)$, where $U(\phi)$ is the (effective) potential.
Therefore, the only requirement for these functions is that they must be bounded from below, in order to exclude ``negative-energy states''%
\footnote{
	We put ``negative-energy states'' in quotation marks, because all quantities in zero-dimensional field theory are dimensionless, \viz{} bare numbers without physical dimensions.
	For convenience, we will still use the well-established notions from higher-dimensional \qfts{} in our discussion.
}
and to obtain positive normalizable probability distributions.
Apart from this requirement, for the moment we do not demand any additional properties, like symmetries (\eg{}, \ZII{}, $\phi \rightarrow - \phi$) or analyticity.

If we choose a specific model with action $\mathcal{S}[\phi]$ all expectation values of arbitrary functions $f(\phi)$ that do not grow exponentially in $\phi$ are defined and can be calculated via the following expression
\begin{align}
	\langle f(\phi) \rangle \equiv \frac{\int_{-\infty}^{+\infty} \dif \phi \, f(\phi) \, \eu^{ - \mathcal{S}[\phi] }}{\int_{-\infty}^{+\infty} \dif \phi \, \eu^{ - \mathcal{S}[\phi] }} \, ,	\label{eq:expectation_value_1}
\end{align}
where $\eu^{ - \mathcal{S}[\phi] }$ provides the partition of probabilities among the microstates.
Note that due to the zero-dimensional nature all expectation values for such a model reduce to indefinite one-dimensional integrals over $\phi$.
Such integrals can be computed to extremely high precision using standard techniques of numerical integration~\cite{Press:1992zz,PresTeukVettFlan92}.
It is worth emphasizing that the current discussion holds also for non-analytic $\mathcal{S}[\phi]$ and/or $f(\phi)$.
Some specific choices of $\mathcal{S}[\phi]$ and $f(\phi)$ even allow for an analytic evaluation of \cref{eq:expectation_value_1}, see, \eg{}, \ccite{Keitel:2011pn}.
The possibility to compute expectation values to high precision makes zero-dimensional field theory of great interest as a testing ground for approximations and/or numerical methods.

Some explicit examples of zero-dimensional field theories used as a testing ground for methods in statistical mechanics and \qft{} can be found in \ccite{Bessis:1980ss,Zinn-Justin:1998hwu,DiVecchia:1990ce,Hikami:1978ya,Nishigaki:1990sk,Schelstraete:1994sc,Keitel:2011pn,Pawlowski:talk,Moroz:2011thesis,Fl_rchinger_2010,SkinnerScript,Strocchi:2013awa,Kemler:2013yka,Rentrop_2015,Rosa:2016czs,Liang:2017whg,Millington:2019nkw,Alexander:2019cgw,Catalano:2019,Millington:2020Talk,Millington:2021ftp,Kades:2021hir,Fraboulet:2021amf}.
In \ccite{Strocchi:2013awa}, for example, the asymptotic convergence and the vanishing convergence radius of perturbation theory of $\phi^4$-theory is discussed.
Approximation schemes such as the large-$N$, the \frg{} vertex expansion, or the \frg{} Taylor expansion were analyzed in \ccite{Keitel:2011pn}.
Zero-dimensional field theory was also used to study density-functional theory in \ccite{Kemler:2013yka,Rentrop_2015,Liang:2017whg} and applied to fermionic fields in \ccite{SkinnerScript}.
Recently, it was used to study and visualize 2PI effective actions~\cite{Millington:2019nkw} \dash{} also in the \frg{} framework~\cite{Alexander:2019cgw,Millington:2020Talk,Millington:2021ftp}.

The calculation of expectation values is facilitated by a suitably defined generating functional
\begin{align}
	\mathcal{Z} [J] \equiv \mathcal{N} \int_{-\infty}^{+\infty} \dif \phi \, \eu^{- \mathcal{S}[\phi] + J \, \phi} \, , \label{eq:partition_function}
\end{align}
from which one can derive all correlation functions by taking the corresponding number of derivatives \wrt{} the external source $J$,
\begin{align}
	\langle f(\phi) \rangle = \frac{f(\tfrac{\delta}{\delta J}) \, \mathcal{Z}[J]}{\mathcal{Z}[J]} \bigg|_{J = 0} \, .	\label{eq:expectation_value_2}
\end{align}
One should note that if $f(\phi)$ is non-analytic, then \cref{eq:expectation_value_2} is to be understood symbolically.
Otherwise, it is defined through a Taylor series in $\frac{\delta}{\delta J}$.
Irrespective of that, \cref{eq:expectation_value_1} and \eqref{eq:partition_function} are always well defined and \cref{eq:partition_function} can be always calculated for arbitrary $J$.
One can even show in zero dimensions that $\mathcal{Z}[J] \in C^\infty$, hence, $\mathcal{Z}[J]$ is a smooth function, see \ccite{Moroz:2011thesis} and \MWApp{}.

The normalization $\mathcal{N}$ is not an observable quantity and for our discussion here, it is convenient to choose
\begin{align}
		\mathcal{Z} [ 0 ] \overset{!}{=} 1 \qquad \longleftrightarrow \qquad \mathcal{N}^{-1} = \int_{-\infty}^{+\infty} \dif \phi \, \eu^{ - \mathcal{S} [\phi] } \, .	\label{eq:normalization}
\end{align}
As already mentioned above, calculating expectation values in a zero-dimensional \qft{} via \cref{eq:expectation_value_1} is (at the very least numerically) rather straightforward.
In contrast, for higher-dimensional models or theories with non-trivial field content, calculating functional integrals similar to \cref{eq:expectation_value_1}, \cf{} \cref{eq:expectationOJ}, with sufficient precision is usually extremely challenging or might even be impossible with limited computational resources.
Therefore, alternative methods like the \frg{} or approximation schemes apart from ``direct numerical integration'', like in lattice simulations, are of great interest.

In the following, we will again focus on the \frg{} as a specific method for calculating \nptFunctions{} in zero-dimensional \qfts{}.
In contrast to the usual motivation of the \frg{}, \cf{} the introduction of \cref{sec:FRG}, we will use a different but as it turns out closely related approach to motivate and ultimately arrive at the \frg{} flow equation \eqref{eq:WetterichEq0d} in zero-dimensions.
To this end, we will follow and extend the discussion in \ccite{Keitel:2011pn,Pawlowski:talk,Moroz:2011thesis,Fl_rchinger_2010,SkinnerScript} and discuss its technical properties as an alternative way of solving the integrals in \cref{eq:expectation_value_1} and \eqref{eq:partition_function}.

\subsection{Solving integrals with flow equations}\label{subsec:0dintegrals}
The starting point of our following discussion, is the observation that there is one well-known non-trivial class of actions $\mathcal{S}[\phi]$ for which the calculation of integrals like \cref{eq:expectation_value_1} is straightforward, even in higher dimensions and even for more complicated field content.
These actions are \qfts{} for ``(massive) free particles'' and correspond to Gaussian-type integrals.
In the present case the Gaussian-type action takes the following simple form,
\begin{align}
	\mathcal{S} [ \phi ] = \tfrac{m^2}{2} \, \phi^2 \, .	\label{eq:mass_term}
\end{align}
where $m$ is called a ``mass'' for convenience, although it is actually a dimensionless quantity in zero space-time dimensions.

For non-trivial actions $\mathcal{S}[\phi]$, \cref{eq:expectation_value_1} can still be approximated by a Gaussian integral, as long as $\mathcal{S}[\phi]$ contains a mass term \eqref{eq:mass_term} with a coefficient $m^2$ that is much larger than all other scales contained in $\mathcal{S}[\phi]$.
If this is the case, the Gaussian part of the integrand $\eu^{-\mathcal{S}[\phi]}$ completely dominates the integrals in \cref{eq:expectation_value_1} and \eqref{eq:partition_function}.
The reason is that the mass term $\sim\phi^2$ is dominant for small and moderate $\phi$, and most of the area under the curve $\eu^{-\mathcal{S}[\phi]}$ lies in the region of small $\phi$, similar to a pure Gaussian integral.
For very large values of $\phi$ other terms in the action $\mathcal{S}[\phi]$ may become more important.
Nevertheless, if $m^2$ is large enough, the corresponding area under the curve $\eu^{-\mathcal{S}[\phi]}$ is completely negligible in regions where $\phi$ is large, because $\mathcal{S}[\phi]$ is bounded from below such that $\eu^{-\mathcal{S}[\phi]}$ tends to zero exponentially fast for $\phi\to\infty$.
In summary, the Gaussian part with the huge mass term dominates the integral and even integrals involving non-trivial actions can be approximated.
This is illustrated in \cref{fig:rg_flow_integrand}.

\subsubsection{The scale-dependent partition function}
Based on the above observation, let us now (re)introduce the following quantity:
\begin{align}
	\mathcal{Z}_t [ J ] \equiv \mathcal{N} \, \int_{-\infty}^{+\infty} \dif \phi \, \eu^{- \mathcal{S} [ \phi ] - \Delta \mathcal{S}_t [ \phi ] + J \, \phi } \, ,	\label{eq:scale_dependent_z}
\end{align}
which is called the \textit{scale-dependent generating functional} or \textit{scale-dependent partition function}.
It differs from the usual partition function \eqref{eq:partition_function} only by a scale-dependent mass term
\begin{align}
	\Delta \mathcal{S}_t [ \phi ] \equiv \tfrac{1}{2} \, r ( t ) \, \phi^2 \, .	\label{eq:regulator_insertion}
\end{align}
We directly adopt the common notation from the \frg{} community and call $r(t)$ the \textit{regulator (shape) function}, which depends on the \textit{\rgscale{}} (``\textit{time}'') $t \in [ \, 0 , \infty )$, see, \eg{}, \ccite{Litim:2000ci,Pawlowski:2015mlf}.
We have now constructed the zero-dimensional analog to the \rgscaledependent{} generating functional \eqref{eq:ZkDef} purely based on the notion of an integral deformation.
For now, we only demand that the function $r(t)$ has such properties that $\mathcal{Z}_t [J]$ interpolates between an almost Gaussian-type partition function%
\footnote{%
	This is also why the \uv{} fixed point of \grg{} flows is denoted as the \textit{trivial} or \textit{Gaussian fixed point}.
}
with extremely massive free fields at $t = 0$ and the actual partition function $\mathcal{Z}[J]$ of interest at $t \rightarrow \infty$.
In order to achieve this behavior, $r ( t )$ has to have the following properties:
	\begin{enumerate}
		\item	In the limit of $t \rightarrow 0$, $r(t)$ ($\Delta \mathcal{S}_t [ \phi ] $) should behave like a mass (term), similar to what we discussed at the beginning of this section, and be much larger than all other scales in $\mathcal{S}[\phi]$. We will discuss this further in \cref{subsec:exact_rg_equation}.
		
		\item	For $t \rightarrow \infty$, $r(t)$ is supposed to vanish, such that $\lim_{t \rightarrow \infty} \mathcal{Z}_t [J]=\mathcal{Z}[J]$.
		The same applies to expectation values calculated from $\mathcal{Z}_t[J]$, which become expectation values of $\mathcal{Z}[J]$.
		For practical calculations it is sufficient to assume that, for $t \rightarrow \infty$, $r(t)$ becomes much smaller than all scales in $\mathcal{S}[\phi]$, because then the contribution $\Delta \mathcal{S}_t [\phi]$ to the whole integrand $\eu^{- \mathcal{S}[\phi] - \Delta \mathcal{S}_t [\phi]}$ is negligible and the integrand is almost identical to $\eu^{- \mathcal{S}[\phi]}$.
		The value $\lim_{t \rightarrow \infty} r ( t ) = r_\mathrm{IR} \smallergtrsim 0$ is usually referred to as \textit{(numerical) \ir{} cutoff}.
		
		\item	The interpretation of $r(t)$ ($\Delta \mathcal{S}_t [ \phi ]$) as a mass (term) is guaranteed by further demanding monotonicity, ${\forall t,\ \partial_t r(t) \leq 0}$.
		We will provide additional arguments for monotonicity in \cref{subsec:flow_equation_of_the_partition_function}.
				
		\item	In order to be able to smoothly deform the integral in \cref{eq:partition_function} and for the following derivation of evolution equations, we further require $r ( t ) \in C^1$.
	\end{enumerate}
Apart from these four properties there are no further requirements on $r(t)$ in zero dimensions and ultimately the regulator choice in zero-dimensions manifests as simple reparametrizations in the flow time $t$.
The first (\uv{}) property is the zero-dimensional realization of general regulator property \customref{paragraph:regulatorUV}{3. \textit{Diverging in the Ultraviolet}} of \cref{subsubsec:regulator}.
The second (\ir{}) property is the zero-dimensional realization of general regulator property \customref{paragraph:regulatorHighP}{2. \textit{Vanishing for high momentum modes}} of \cref{subsubsec:regulator}.
Note that for higher-dimensional field theories the fourth requirement turns into $\Delta \mathcal{S}_t [ \phi ] \in C^1$.
A specific choice which is used in large parts of our work is the zero-dimensional exponential regulator (shape) function
	\begin{align}
		r ( t ) = \Lambda \, \eu^{- t} \, ,	\label{eq:exponential_regulator}
	\end{align}
with an \uv{} cutoff/scale $\Lambda$, which must be chosen much larger than all scales in $\mathcal{S} [ \phi ]$.

In order to get a better intuition of the effect of $r(t)$ on the integral \eqref{eq:scale_dependent_z}, we show the integrand at $J=0$, $\eu^{- \mathcal{S}[\phi] - \Delta \mathcal{S}_t [\phi]}$, and the respective exponent for different values of $t$ for the analytic action
\begin{align}
	\mathcal{S} ( \phi ) = - \tfrac{1}{2} \, \phi^2 + \tfrac{1}{4!} \, \phi^4 \, ,\label{eq:example_analytic_action}
\end{align}
in \cref{fig:rg_flow_integrand_z_1} and in \cref{fig:rg_flow_integrand_z_2} the same quantities for the non-analytic action
	\begin{align}
		\mathcal{S} ( \phi ) =
		\begin{cases}
			- \phi^2 \, ,								&	\text{if} \quad | \phi | \leq \tfrac{5}{4} \, ,	
			\\[.1em]
			- \big( \tfrac{5}{4} \big)^2 \, ,			&	\text{if} \quad \tfrac{5}{4} < | \phi | \leq 2 \, ,
			\\[.1em]
			\tfrac{1}{48} \,, \big( \phi^4 - 91 \big)	&	\text{if} \quad | \phi |>2 \, .
			\\
		\end{cases}	\label{eq:example_non-analytic_action}
	\end{align}
The figures show how the integrands are deformed from Gaussian-shaped integrands to the integrands $\eu^{- \mathcal{S}[\phi]}$.
One observes that, as long as $r(t)$ is much larger than all other parameters in $\mathcal{S}[\phi]$, the Gaussian-like mass term dominates, while for increasing $t$ the regulator $r(t)$ becomes negligible.
The most interesting part, where the integrands change their shapes significantly, is where $r(t)$ is of the same order as the scales in $\mathcal{S} ( \phi )$.
\subcaptionFigure%
	[t]% Placement
	{0d/figures/rg_flow_integrand_z_smooth.pdf}% Figure (a)
	[\captionsetup{justification=centering}\caption{Plots for the action \eqref{eq:example_analytic_action}}]% Caption (a)
	{fig:rg_flow_integrand_z_1}% label (a)
	{0d/figures/rg_flow_integrand_z_kink.pdf}% Figure (b)
	[\captionsetup{justification=centering}\caption{Plots for the action  \eqref{eq:example_non-analytic_action}}]% Caption (b)
	{fig:rg_flow_integrand_z_2}% label (b)
	{%
	The integrand (upper panel) and exponent (lower panel) from \cref{eq:scale_dependent_z} (at $J = 0$) as a function of the field variable $\phi$ for various \rgtimes{} $t = 0, 1, 2, \ldots, 15$ and for different actions on the left \subref{fig:rg_flow_integrand_z_1} and on the right \subref{fig:rg_flow_integrand_z_2}.
	We choose the exponential regulator~\eqref{eq:exponential_regulator} with \uv{} scale $\Lambda = 10^3$.
	The \ir{} cutoff scale is chosen at  $r_\mathrm{IR} \simeq 3.06 \cdot 10^{-4}$ which corresponds to $t = 15$.
	The numerical value of $r_\mathrm{IR}$ is significantly smaller than all scales in $\mathcal{S}[\phi]$.
	\fromFigs{1 and 2}{zerod1}%
	}% Caption
	{fig:rg_flow_integrand}% Label

\FloatBarrier
\subsubsection{A flow equation for the scale-dependent partition function}
\label{subsec:flow_equation_of_the_partition_function}

The change of the integrals with $t$ between the two limiting cases at $t=0$ and $t\rightarrow\infty$ is the in \cref{sec:FRG} established \textit{\frg{} flow} from the \uv{} to the \ir{}.
By knowing/mathematically prescribing the integral deformation, we can obtain the function ${\mathcal{Z}(J) \equiv \lim_{t \rightarrow \infty} \mathcal{Z}_t [J] = \mathcal{Z}[J]}$ right from the Gaussian-like partition function $\mathcal{Z}_{t = 0}[J]$ without the need to calculate the $\phi$-integral in the partition function~\eqref{eq:partition_function} directly.
In zero dimensions this might arguably be one of the most complicated ways to compute the partition function~\eqref{eq:partition_function}, because the integrals in field space are (at least numerically) simple to compute.
For higher dimensions, however, circumventing the corresponding challenging functional integration is a tremendous benefit.\bigskip

The \frg{} flow of $\mathcal{Z}_t [J]$, \ie{}, its deformation, is prescribed by a differential equation which can be obtained by taking the derivative \wrt{} the \rgtime{} $t$ of \cref{eq:scale_dependent_z}:
\begin{subequations}\label{eq:pde_zF}
\begin{align}
		\partial_t \mathcal{Z}_t[J] =	\, & - \big[ \tfrac{1}{2} \, \partial_t r ( t ) \big] \, \mathcal{N} \int_{-\infty}^{+\infty} \dif \phi \, \phi^2 \, \eu^{ - \mathcal{S}[\phi] - \Delta \mathcal{S}[\phi] + J \, \phi}\, =\\
		=\,& - \big[ \tfrac{1}{2} \, \partial_t r ( t ) \big] \, \frac{\delta^2 \mathcal{Z}_t [J]}{\delta J \, \delta J} \equiv \,  - \big[ \tfrac{1}{2} \, \partial_t r ( t ) \big] \, \mathcal{Z}^{(2)}_{t, J J} [J] \, ,	
	\end{align}
\end{subequations}
which directly manifests as a simple \pde{} for the function $\mathcal{Z} ( t, J )$ in the $t$-$J$-plane,
\begin{align}
	\partial_t \mathcal{Z}(t,J) = - \big[ \tfrac{1}{2} \, \partial_t r ( t ) \big] \, \partial_J^2 \mathcal{Z}(t,J) \, .	\label{eq:pde_z}
\end{align}
Solving this equation with appropriate initial and boundary conditions results in a function $\mathcal{Z} ( J )$ from which one can calculate expectation values by taking ordinary (numerical) derivatives \wrt{} $J$ at $J = 0$, \cf{}\ \cref{eq:expectation_value_2}.

The higher-dimensional analog to \cref{eq:pde_z} is \cref{eq:dZkdtflow} and thus we again recognize \cref{eq:pde_z} as a linear one-dimensional diffusion equation (\textit{heat equation})~\cite{Rosten:2010vm,SkinnerScript,Salmhofer:2020Talk,Cannon:1984}, where $t$ corresponds to the temporal direction, while $J$ corresponds to the spatial direction.
We discussed this type of conservation equation in detail in \cref{subsec:hydroDiffusion}.
The term $- \tfrac{1}{2} \, \partial_t r ( t )$ corresponds to a time-dependent (positive definite) diffusion coefficient%
\footnote{%
	Note that in zero dimensions one can get rid of $\partial_t r ( t )$ by an appropriate reparametrization of the time coordinate $t$, which nevertheless keeps the structure of the equation unchanged.
	In higher dimensions this elimination of $r ( t )$ is in general not possible.
	The positivity of the diffusion coefficient is directly related to the stability of solutions of the heat equation~\cite{LeVeque:1992,LeVeque:2002} and positivity \dash{} here guaranteed by the regulator properties \dash{} is necessary for a stable solution~\cite{Rosten:2010vm,Osborn:2011kw}.
}.
This further supports the notion of \rg{} ``time'' for the parameter $t$.
We will come back to the concept of \rg{} ``time'' in the true sense of the word and the diffusive, irreversible character of \frg{} flows in \cref{subsubsec:conservative_form}.\bigskip

For the remainder of this subsection we will discuss properties and practical issues considering the exact \pde{} \eqref{eq:pde_z}.
We will neither discuss any kind of expansions in $J$ nor its application in higher dimensions here.
However, some of the issues and questions raised in the following are also relevant for higher-dimensional theories.

Finding the correct initial and boundary conditions for numerical solutions of \cref{eq:pde_z} as an exact \pde{} is challenging.
By construction $\mathcal{Z}_{t = 0} [ J ]$ approaches a Gaussian integral,
\begin{align}
	\stepcounter{equation}\newSubEqBlock
	\mathcal{Z}_{t = 0} [ J ] = \, & \mathcal{N} \, \int_{-\infty}^{+\infty} \dif \phi \, \eu^{- \frac{1}{2} r ( 0 ) \, \phi^2 + J \, \phi} \, \eu^{- \mathcal{S} ( \phi )}\,=\subEqTag\\* % no page break
	= \, & \mathcal{N} \, \int_{-\infty}^{+\infty} \frac{\dif\tilde{\phi}}{\sqrt{r(0)}} \, \eu^{- \frac{1}{2} \, \tilde{\phi}^2 + J \, \frac{\tilde{\phi}}{\sqrt{r(0)}}} \, \Big[ 1- \order \big( \mathcal{S} \big( r ( 0 )^{-\frac{1}{2}} \big) \big) \Big]\,= \subEqTag\\* % no page break
	= \, & \mathcal{N} \, \sqrt{\tfrac{2\piu}{r(0)}} \, \eu^{\frac{J^2}{2 r(0)}} \, \Big[ 1- \order \big( \mathcal{S} \big( r(0)^{-\frac{1}{2}} \big) \big) \Big] \, ,	\label{eq:initial_condition_z}
\end{align}
with $\tilde{\phi}\equiv \sqrt{r(0)}\,\phi$ and independent of the explicit shape of $\mathcal{S}[\phi]$.
Considering different actions $\mathcal{S}[\phi]$ with couplings of the same order of magnitude we can choose the same regulator with an $r(0)$ larger than all internal scales involved in the different actions.
The \ic{} $\mathcal{Z}(0,J)$ is then independent of the explicit action under consideration.

According to the integral formulation~\eqref{eq:scale_dependent_z}, $\mathcal{Z}(t,J)$ changes for different actions when $t > 0$.
In the differential formulation of the \cref{eq:pde_z} those changes are generated by the diffusion term on the \rhs{}
However, we argued that it is permissible to use identical \ics{} $\mathcal{Z} ( 0, J)$ for different actions involving similar scales (as long as these are much smaller than $r ( 0 )$).
This then results in an identical diffusion on the \rhs{} of \cref{eq:pde_z} when the latter is computed by means of a second derivative of $\mathcal{Z}(0,J)$.
If one uses identical large-$J$ \bcs{} for the solution of the \pde{} \eqref{eq:pde_z} for different actions, this would imply that, despite different $\mathcal{S}[\phi]$, the \rgtime{} evolution leads to identical $\mathcal{Z}(J)$ for $t\rightarrow\infty$, which in general cannot be correct.

In order to resolve this problem, particular action-dependent spatial \bcs{} seem to be necessary for a direct numerical solution starting at $t = 0$ with a Gaussian for $\mathcal{Z} ( 0, J )$.
It is not obvious how to derive or formulate such \bcs{} from the asymptotics of \cref{eq:pde_z} alone.
In light of this, a numerical solution of \cref{eq:pde_z} in the $t$-$J$-plane by means of a spatial discretization in $J$-direction and an integration in $t$-direction in the spirit of \cref{subsec:hydroDiffusion} appears to be conceptually questionable.

However, this invalidates by no means the flow equation for $\mathcal{Z} ( t, J )$ in general.
Augmenting it (at $t = 0$) with information from the integral formulation~\eqref{eq:scale_dependent_z} or, equivalently, other additional information, could enable practical computations using the \pde{} \eqref{eq:pde_z}.
But it is at this point (at least to us) not obvious how one would implement a numerical solution strategy for the \pde{} \eqref{eq:pde_z} avoiding integrals of the action.\\

There is another well-known drawback in using the partition function $\mathcal{Z} [ J ]$ for calculating \nptFunctions{} (or expectation values) $\langle \phi^n \rangle$.
The latter are rather inefficient in storing information, because they contain redundant information in the form of disconnected and reducible terms, see \ccite{Iliopoulos:1974ur,ZinnJustin:2002ru,Weinberg:1996kr,Peskin:1995ev} or the mathematical theory of moment- and cumulant-generating functionals in statistics for details~\cite{McCullagh:2009}.
This is also discussed in Sec.~II.E of \nbccite{zerod1}.
However, the redundant information in $\langle \phi^n \rangle$ is not necessarily a strong argument against the use of the flow equation \eqref{eq:pde_z} in practical computations, since the irreducible information can be extracted from the correlation functions $\langle \phi^n \rangle$.\bigskip

\subsection{The functional renormalization group equation}\label{subsec:exact_rg_equation}
To resolve both the problem of initial and \bcs{} for $\mathcal{Z}(t,J)$ as well as the issue of redundant information in $\langle \phi^n \rangle$, we now consider two different generating functionals the \textit{scale-dependent Schwinger functional} in \cref{subsubsec:WtJd0} and the \textit{scale-dependent effective action} in \cref{subsubsec:scale_dependent_effective_action}. 
Especially the latter will turn out to be much better suited for practical calculations of correlation functions/expectation values.
In this subsection we will derive and discuss the \frg{} flow \cref{eq:WetterichEq0d} for our zero-dimensional toy model \qft{} by subsequently (re)discovering the zero-dimensional analogs to the general expressions of \cref{sec:FRG}.
	
\subsubsection{The scale-dependent Schwinger functional}
\label{subsubsec:WtJd0}
We begin our journey towards the \frg{} flow \cref{eq:WetterichEq0d} by introducing the \textit{scale-dependent Schwinger functional} starting from definition~\eqref{eq:scale_dependent_z},
\begin{align}
	\mathcal{W}_t [J] \equiv \ln \mathcal{Z}_t [J] \, ,	\label{eq:scale_dependent_w}
\end{align}
in direct analogy to \cref{eq:WkDef}.
It follows from our previous discussion that for $t \rightarrow \infty$ the Schwinger functional $\mathcal{W}[J] \equiv \ln \mathcal{Z}[J]$ with	$\mathcal{W} [ 0 ] = 0$ is recovered,
\begin{align}
	\lim\limits_{t \rightarrow \infty} \mathcal{W}_t [J] = \mathcal{W} [J] \, ,
\end{align}
while $\mathcal{W}_{t = 0} [J]$ is given by the logarithm of \cref{eq:initial_condition_z}.

In general $\mathcal{W}[J]$ is convex with a positive definite Hessian $\mathcal{W}^{(2)}_{J J} [ J ]$.
In the present case the convexity of $\mathcal{W} [ J ] = \mathcal{W} ( J )$ becomes apparent considering its second derivative,
\begin{align}
	\partial_J^2 \mathcal{W}(J) = \langle \phi^2 \rangle_J -\langle \phi \rangle_J \langle \phi \rangle_J= \langle (\phi-\langle \phi \rangle_J)^2\rangle_J \, ,	\label{eq:jacobian_w}
\end{align}
which, as the expectation value of a positive quantity, is always positive.
Note that also $\mathcal{Z}[J]$ is convex, which can be seen by investigating its second derivative.
In zero dimensions, also smoothness, \ie{}, $\mathcal{Z} [ J ] \in C^\infty$, directly translates to $\mathcal{W} [ J ] \in C^\infty$, because all moments of $\mathcal{W} [ J ]$ can be entirely expressed in terms of derivatives of $\mathcal{Z} [J]$, see Sec.~II.E of \nbccite{zerod1} for explicit expressions.
The insertion of the regulator \eqref{eq:regulator_insertion} into $\mathcal{Z}_t [J]$ does not spoil the convexity and smoothness (in zero dimensions) of the Schwinger functional: $\mathcal{W}_t [J]$ and $\mathcal{Z}_t [J]$ are convex and smooth for all $t$.

Completely analogous to \cref{eq:pde_zF} one can derive a \pde{} for $\mathcal{W}_t[J] = \mathcal{W}(t,J)$ in the $t$-$J$-plane,
\begin{align}
	\partial_t \mathcal{W}(t,J) =	\, - \big[ \tfrac{1}{2} \, \partial_t r ( t ) \big] \, \Big( \partial_J^2 \mathcal{W}(t,J) + \big[ \partial_J \mathcal{W}(t,J) \big]^2 \Big) \, .\label{eq:pde_w}
\end{align}
which describes the flow of $\mathcal{W}(t,J)$ from $t = 0$ to $t \rightarrow \infty$.
We have recovered the zero-dimensional analog to \cref{eq:dWkdtflow}.

We could now repeat the discussion about the issues of initial and boundary conditions for the solution of this \pde{}.
However, the problems are almost identical to those of \cref{eq:pde_z}, because on the level of the \pde{}, we only substituted the function $\mathcal{Z} ( t, J )$ by $\mathcal{W} ( t, J )$ via the logarithm, which does not change the structure of the problem fundamentally.
Formulating appropriate initial and boundary conditions in the spatial $J$-direction therefore remains as complicated as before.
Note that the \pde{} \eqref{eq:pde_w} is even more complicated when compared to \cref{eq:pde_z} due to the non-linear term on the right-hand side.
In summary, the scale-dependent Schwinger functional is, from a practical point of view, as badly suited as $\mathcal{Z}(t,J)$ to perform the (numeric) calculation of the functional integral via a flow equation starting from a Gaussian-type integral.

\subsubsection{The scale-dependent effective action}\label{subsubsec:scale_dependent_effective_action}
In the following we will focus on the scale-dependent effective (average) action and its respective flow equation.
We define the \textit{scale-dependent effective action} $\Gamma_t[\varphi]$ via the Legendre transform of \cref{eq:scale_dependent_w} \wrt{} the sources $J$ at a \rgtime{} $t$,
\begin{subequations}
\begin{align}
	\Gamma_t [\varphi] \equiv \, & \mathop{\mathrm{sup}}_J \big( J \, \varphi - \mathcal{W}_t[J] \big)\, ,\label{eq:scale_dependent_effective_action}\\
	\equiv  \, & J_t(\varphi) \, \varphi - \mathcal{W}_t[J_t(\varphi)] \, ,	\label{eq:scale_dependent_effective_action_Jt}
\end{align}
\end{subequations}
where we introduced the source $J_t(\varphi)$ which realizes the supremum.
Note that, analogous to $\mathcal{Z}_t [J]$ and $\mathcal{W}_t [J]$, the convexity and smoothness (in zero dimensions) of $\Gamma_t [\varphi]$ is not spoiled by the $t$-dependence, because the properties of the Legendre transformation still ensure both, since the Legendre transform of the convex function $\mathcal{W}_t[J]$ is convex by definition, see, \eg{}, \ccite{Fujimoto:1982tc,Wipf:2013vp} for details.
	
To obtain an explicit relation for the scale-dependent source $J_t ( \varphi )$, which realizes the supremum in \cref{eq:scale_dependent_effective_action}, we consider the functional derivative of \cref{eq:scale_dependent_effective_action} at the supremum to find the important relation
\begin{align}
	\mathcal{W}_{t,J}^{(1)}[J_t(\varphi)] \equiv \frac{\delta \mathcal{W}_t [J]}{\delta J}\bigg|_{J=J_t(\varphi)} = \varphi \, ,	\label{eq:definition_phi_t}
\end{align}
which will be used frequently in the following.
Taking the functional derivative of \cref{eq:scale_dependent_effective_action_Jt} \wrt{} $\varphi$ and using \cref{eq:definition_phi_t} we ultimately rediscover the \textit{quantum equation of motion} in zero-dimensions:
\begin{align}
	\Gamma^{(1)}_{t, \varphi}[\varphi] \equiv \frac{\delta \Gamma_t [\varphi]}{\delta \varphi} = J_t(\varphi) \, ,	\label{eq:definition_j_t}
\end{align}
as a special simplification of the general analog \eqref{eq:QEOM}.
Due to the strict convexity of $\Gamma_t [ \varphi ]$ the function $J_t ( \varphi )$ is bijective and as such can be inverted, which can be achieved by considering \cref{eq:definition_phi_t} at fixed value $J$ for $J_t$:
\begin{align}
	\varphi_t ( J ) \equiv \frac{\delta \mathcal{W}_t [ J ]}{\delta J} \, ,	\label{eq:definition_phi_t_explicit}
\end{align}
where $\varphi_t ( J )$ is the so-called \textit{scale-dependent classical field} (sometimes also referred to as \textit{scale-dependent mean-field}), \cf{} \cref{eq:dWdFSmfJ}.
	
The subtle relations between, and scale-dependencies of, $\varphi_t ( J )$ and $J_t ( \varphi )$ on this formal level are rarely discussed in literature and usually suppressed in the notation.
The relation between $\varphi_t$ and $J_t$ is of particular importance for discussion and relations between of $n$-point correlation functions, see, \eg{}, Sec.~II.E of \nbccite{zerod1}.
The scale-dependence of $\varphi_t(J)$ from \cref{eq:definition_phi_t_explicit} is not related to a rescaling (\rg{} transformation) using, \eg{}, a wave-function renormalization for $\varphi$, \cf{} \cref{eq:dFSmfdtLinear} and the corresponding discussion.

\paragraph{\ic{} and the \uv{} limit $t\rightarrow 0$}\mbox{}\label{subsubsec:zerodICS} \\
Before we reintroduce the Wetterich equation, which is the flow equation for $\Gamma_t [ \varphi ]$ and a \pde{} for the function $\Gamma ( t, \varphi )$ in the $t$-$\varphi$-plane, we check whether we will run into the same issues (related to initial and boundary conditions) as before.
Hence, first of all, we must derive the \ic{} for the \pde{} for $\Gamma ( t, \varphi )$.
To this end, we study the limit $t \rightarrow 0$ of $\Gamma_t [ \varphi ]$.
We use the definitions~\eqref{eq:scale_dependent_z}, \eqref{eq:scale_dependent_w}, \eqref{eq:scale_dependent_effective_action}, and \eqref{eq:scale_dependent_effective_action_Jt} to obtain
\begin{subequations}\label{eq:derivation_uv_condition}
\begin{align}
	\eu^{ - \Gamma_t [ \varphi ] } = \, & \eu^{ - \mathop{\mathrm{sup}}_J ( J \, \varphi - \mathcal{W}_t [ J ] ) } =\, \eu^{ \ln \mathcal{Z}_t [ J_t ( \varphi )] - J_t ( \varphi ) \, \varphi }\, =
	\\* % no page break here
	= \, & \mathcal{N} \int_{-\infty}^{+\infty} \dif \phi \, \eu^{ - \mathcal{S} [ \phi ] - \Delta \mathcal{S}_t [ \phi ] + J_t ( \varphi ) \, ( \phi - \varphi ) } \, .
\end{align}
\end{subequations}
We now shift the integration variable%
\footnote{%
	It is the same shift that is used in the background field formalism~\cite{DeWitt:1965jb,Abbott:1981ke}, where the full fluctuating quantum field $\phi$ is split into a background field configuration $\varphi$ and additional fluctuations $\phi^\prime$ about the background field.
	This is why $\varphi$ in this cortex is referred to as the \textit{classical} or \textit{mean} field.
}
$\phi \mapsto \phi^\prime = \phi - \varphi$. Using \cref{eq:regulator_insertion}, we find
\begin{align}
	\eu^{- \Gamma_t [ \varphi ] + \Delta \mathcal{S}_t [ \varphi ]} =	 \, & \mathcal{N} \int_{-\infty}^{+\infty} \dif\phi^\prime \, \eu^{ - \mathcal{S} [ \phi^\prime + \varphi ] - \Delta \mathcal{S}_t [ \phi^\prime ] - r ( t ) \, \phi^\prime \, \varphi + \Gamma^{(1)}_{t, \varphi} [ \varphi ] \, \phi^\prime } \, .	\label{eq:derivation_uv_condition_3}
\end{align}
In the next step, we reintroduce the \textit{scale-dependent effective average action},
\begin{align}
	\bar{\Gamma}_t [ \varphi ] \equiv \Gamma_t [ \varphi ] - \Delta \mathcal{S}_t [ \varphi ] \, ,	\label{eq:scale_dependent_effective_average_action}
\end{align}
which also tends to the effective action $\Gamma [ \varphi ]$ for $t \rightarrow \infty$, because the second term vanishes in this limit, \cf{}~\cref{eq:exponential_regulator}.

At any finite value of $t$ (including $t=0$), $\bar{\Gamma}_t [ \varphi ]$ differs from $\Gamma_t [ \varphi ]$ and is no longer guaranteed to be convex, which can be seen directly from the second term in \cref{eq:scale_dependent_effective_average_action}.
Convexity is only recovered for $t \rightarrow \infty$.
However, the second term in \cref{eq:scale_dependent_effective_average_action} does not violate the smoothness of $\bar{\Gamma}_t[\varphi]$ in zero dimensions for all $t$, because $\Delta \mathcal{S}_t [\varphi] \equiv \mathcal{S}_t (\varphi) \in C^\infty$ in $\varphi$.

We express \cref{eq:derivation_uv_condition_3} in terms of the scale-dependent effective average action \eqref{eq:scale_dependent_effective_average_action} and, for the sake of convenience, revert the notation $\phi^\prime \rightarrow \phi$,
\begin{align}
	& \eu^{- \bar{\Gamma}_t [ \varphi ]} =	\mathcal{N} \, \int_{-\infty}^{+\infty} \dif \phi \, \eu^{ - \mathcal{S} [ \phi + \varphi ] - \Delta \mathcal{S}_t [ \phi ] + \bar{\Gamma}^{(1)}_{t, \varphi} [ \varphi ] \, \phi } \, .	\label{eq:exp_gamma_bar_int}
\end{align}
We have arrived at the zero-dimensional analog of \cref{eq:EEAint} which has already facilitated our discussion in \cref{subsubsec:regulator} of \ics{} for general \frg{} flow equations.
In the next step one formally introduces the normalization of a Gaussian integral with mass $r ( t )$ and takes the logarithm, which results in
\begin{align}
	\bar{\Gamma}_t [ \varphi ] = \, & - \ln \int_{-\infty}^{+\infty} \dif \phi \, \sqrt{\tfrac{r ( t )}{2 \piu}} \, \eu^{ - \mathcal{S} [ \phi + \varphi ] - \frac{1}{2} r ( t ) \, \phi^2 + \bar{\Gamma}^{(1)}_{t, \varphi} [ \varphi ] \, \phi} - \ln \Big[ \mathcal{N} \sqrt{\tfrac{2\piu}{r ( t )}} \Big] \, . \label{eq:derivation_uv_condition_2}
\end{align}
We are now ready to study the limit $t \rightarrow 0$, which corresponds to the \ic{} for a possible flow equation for $\Gamma_t [ \varphi ]$ or $\bar{\Gamma}_t [ \varphi ]$, respectively.
Focusing on the $\phi$ integral in the first term on the \rhs{} of \cref{eq:derivation_uv_condition_2}, we leverage the fact that the regulator terms act like a Gaussian representation of the Dirac delta distribution,
\begin{align}
	\lim\limits_{t \rightarrow 0} \sqrt{\tfrac{r ( t )}{2\piu}} \, \eu^{- \frac{1}{2} r ( t ) \, \phi^2} \approx \delta (\phi) \, ,	\label{eq:delta_distribution}
\end{align}
as long as $r ( t )$ is much larger than all scales in $\mathcal{S} [ \phi ]$.
Thus, denoting
\begin{align}
	c ( t ) \equiv - \ln \Big[ \mathcal{N} \, \sqrt{\tfrac{2\piu}{r ( t )}} \, \Big] \, ,
\end{align}
we find as 
\begin{align}
	\lim_{t\rightarrow 0}\bar{\Gamma}_t [\varphi] = \, & - \ln \int_{-\infty}^{+\infty} \dif \phi \, \delta(\phi) \, \eu^{ - \mathcal{S}[\phi + \varphi] + \bar{\Gamma}^{(1)}_{t,\varphi}[\varphi] \, \phi } + c ( t ) = \, \mathcal{S}[\varphi] + c ( t ).	\label{eq:initial_condition_gamma}
\end{align}
This means that the \ic{} for a flow of $\bar{\Gamma}_t [ \varphi ]$ is given by the classical action $\mathcal{S}$ evaluated for the classical field $\varphi$ and some additional $t$-dependent, but $\varphi$-independent term $c ( t )$.
This choice for an \ic{} of a \pde{} for $\bar{\Gamma}_t [ \varphi ]$ has subtle consequences:

Although $c ( t )$ does not depend on $\varphi$, it is large, $c ( t ) \sim  \frac{1}{2} \ln r ( t )$.
Consequently, as far as the \ic{} for the \pde{} for $\Gamma_t[\varphi]$ or $\bar{\Gamma}_t[\varphi]$ is concerned, it seems as if we run into the same problem as before:
The \ic{} is dominated by the artificial mass of the regulator $r ( t )$, independent of the specific action $\mathcal{S} [ \phi ]$, and differences in the specific choice for $\mathcal{S} [ \phi ]$ enter the \ic{} only as small deviations from the large term $c ( t )$.
Furthermore, $c ( t )$ contains the normalization constant $\mathcal{N}$, which was fixed according to \cref{eq:normalization}.

However, precisely because $c ( t )$ appears like the normalization $\mathcal{N}$, it should be irrelevant for all physical observables.
Indeed this is the case, because all $\varphi$-independent terms in $\Gamma_t [ \varphi ]$ do not enter the \nptFunctions{}, since the latter are calculated as derivatives of $\Gamma [ \varphi ]$ \wrt{} $\varphi$ at $t \rightarrow \infty$, see, \eg{}, Sec. II E of \ccite{zerod1} for explicit expressions.
This implies that an additive, $\varphi$-independent term in the three effective actions $\Gamma [ \varphi ]$, $\Gamma_t [ \varphi ]$, and $\bar{\Gamma}_t [ \varphi ]$ is irrelevant and only relative differences in the effective actions are observable.
Therefore, we can simply omit $c ( t )$ and take $\mathcal{S} [ \varphi ]$ as \ic{} for the \pde{} for $\bar{\Gamma}_t [ \varphi ]$.
This is however only valid if the \pde{} for $\bar{\Gamma}_t [ \varphi ]$ is independent of its zeroth moment \dash{} \viz{} only derivatives $\delta_\varphi^n\bar{\Gamma}_t [ \varphi ]$ contribute to the flow equation \dash{} otherwise $c ( t )$ would influence the flow in a time-dependent manner. 
Fortunately, we know from our general discussion in \cref{subsec:wetterich} that this is the case for the \frgEq{} and by extension its zero-dimensional analog \eqref{eq:WetterichEq0d}.\bigskip

We want to conclude this discussion regarding the \ic{} $\bar{\Gamma}_t [ \varphi ]=\mathcal{S} [ \varphi ]$ with a discussion of convexity, smoothness and \rgcy{}.

After \cref{eq:scale_dependent_effective_average_action} we argued that $\bar{\Gamma}_t [ \varphi ]$ does not need to be convex, but must still be smooth for all $t$.
Let us for example consider the non-analytic action \eqref{eq:example_non-analytic_action} as an \ic{}, $\bar{\Gamma}_{t = 0} [ \varphi ] = \mathcal{S} [ \varphi ]$, which does not cause any problems for the convexity and the smoothness of $\mathcal{Z}_t [ J ]$ and $\mathcal{W}_t [ J ]$ at arbitrary $t$, as discussed in App.~B and especially with Fig.~36 of \nbccite{zerod1}.
The non-convexity of $\mathcal{S} [ \varphi ]$ is also not a problem for $\bar{\Gamma}_t [ \varphi ]$, which does not necessarily need to be convex at finite $t$.
Nevertheless, the smoothness of $\bar{\Gamma}_t [ \varphi ]$ is violated by this choice of $\mathcal{S} [ \varphi ]$ at $t = 0$.
This issue originates from relation \eqref{eq:delta_distribution}, which is exactly fulfilled only in the limit $\Lambda \rightarrow \infty$ for the \uv{} scale.
This, however, leads to a trivial theory of infinitely massive particles at $t = 0$, \cf{}\ \cref{eq:scale_dependent_z}.

If one chooses a reasonably large but finite $\Lambda$ and does not use \cref{eq:delta_distribution}, one would ensure that $\bar{\Gamma}_t [ \varphi ]$ is also smooth at $t = 0$.
However, then the \ic{} is not exactly $\mathcal{S} [ \varphi ]$.
In consequence, if we use the approximation \eqref{eq:delta_distribution} even for finite $\Lambda$, one has to pay the price of introducing errors into the \ic{} as well as violating the smoothness of $\bar{\Gamma}_t [ \varphi ]$ at $t = 0$.
But in return one has a well-defined \ic{} $\mathcal{S} [ \varphi ]$ for the \pde{} for $\bar{\Gamma}_t [ \varphi ]$.
However, if $\Lambda$ is chosen to be much larger than all scales in $\mathcal{S} [ \phi ]$, the errors from the \ic{} are minor and expected to be of magnitude
\begin{align}
	\mathrm{error} \approx \frac{\text{largest scale in\,\,} \mathcal{S}}{\Lambda} \, ,	\label{eq:error_scaling_uv_cutoff}
\end{align}
We will come back to this issue in \cref{paragraph:ONRGconsistency} in the context of \rgcy{}, see also \cref{subsec:RGconsistency}.

Additionally, we will find that also the smoothness of $\bar{\Gamma}_t [ \varphi ]$ is recovered automatically for all $t > 0$ by the structure of the PDE for $\bar{\Gamma}_t [ \varphi ]$, because it always contains diffusive contributions which immediately smear out kinks in the \ic{} right in the first time step.
We will also come back to this issue later on, after we have derived the flow equation \eqref{eq:WetterichEq0d} in zero dimensions and discussed its diffusive, irreversible character.

\subsubsection{The exact renormalization group equation in zero dimensions}\label{subsubsec:exact_rg_equation}
In analogy to the previous flow equations in zero dimensions, the flow equation for $\bar{\Gamma}_t [ \varphi ]$ is obtained by taking the derivative of $\bar{\Gamma}_t [ \varphi ]$ \wrt{} $t$ and using the definitions~\eqref{eq:scale_dependent_effective_action} and \eqref{eq:scale_dependent_effective_average_action} to express the derivative of $\bar{\Gamma}_t [ \varphi ]$ by the scale-dependent Schwinger functional,
\begin{align}
	\stepcounter{equation}\newSubEqBlock
	\partial_t \bar{\Gamma}_t [\varphi] = \, & \partial_t \, \big( \Gamma_t [\varphi] - \Delta \mathcal{S}_t [\varphi] \big) 
	= \, \partial_t \, \big( J_t(\varphi) \, \varphi - \mathcal{W}_t[J_t(\varphi)] - \Delta \mathcal{S}_t [\varphi] \big)\, =\subEqTag
	\\*[.2em] % no page break here
	= \, & [ \partial_t J_t (\varphi) ] \, \varphi - \partial_t \mathcal{W}_t [J_t (\varphi)] - [ \partial_t J_t (\varphi) ] \, \mathcal{W}^{(1)}_{t,J_t} [J_t] - \big[\tfrac{1}{2} \, \partial_t r ( t ) \big]\, \varphi^2\, = \subEqTag
	\\*[.2em] % no page break here
	= \, & - \partial_t \mathcal{W}_t [J_t (\varphi)] - \big[\tfrac{1}{2} \, \partial_t r ( t ) \big] \, \varphi^2 \, ,\label{eq:wetterich_mix_w_gamma}
\end{align}
where we used the chain rule and \cref{eq:definition_phi_t}.
Using the flow equation for the Schwinger functional \eqref{eq:pde_w} to substitute the first term on the \rhs{} of \cref{eq:wetterich_mix_w_gamma} and again employing the identity \eqref{eq:definition_phi_t}, we arrive at
\begin{align}
	\partial_t \bar{\Gamma}_t [ \varphi ] = \, & \big[\tfrac{1}{2} \, \partial_t r ( t ) \big] \, \mathcal{W}_{t,J J}^{(2)} [ J_t ( \varphi ) ] \, .	\label{eq:wetterich_mix_w_gamma_2}
\end{align}
It remains to replace the second derivative of the scale-dependent Schwinger functional by a corresponding derivative of $\bar{\Gamma}_t [\varphi]$.
This is done via the identity
\begin{align}
	1 = \frac{\delta J_t(\varphi)}{\delta \varphi} \, \frac{\delta \varphi}{\delta J_t ( \varphi )} = \Gamma^{(2)}_{t ,\varphi \varphi} [ \varphi ] \, \mathcal{W}^{(2)}_{t , J J} [ J_t ( \varphi ) ] \, ,	\label{eq:g2w2_identity}
\end{align}
which follows from \cref{eq:definition_phi_t} and \eqref{eq:definition_j_t}.
Plugging this into \cref{eq:wetterich_mix_w_gamma_2} and using \cref{eq:scale_dependent_effective_average_action} with \cref{eq:regulator_insertion} we rediscover the \textit{Exact Renormalization Group equation} or \textit{Wetterich equation}~\cite{Wetterich:1992yh,Ellwanger:1993mw,Morris:1993qb}
\begin{align}
	\partial_t \bar{\Gamma}_t [\varphi] = \big[\tfrac{1}{2} \, \partial_t r ( t ) \big] \, \big[ \bar{\Gamma}^{(2)}_{t , \varphi \varphi} [\varphi] + r ( t ) \big]^{-1} \, ,	\label{eq:WetterichEq0d}
\end{align}
as the application of the general \frgEq{} to the zero-dimensional theory of a single scalar $\phi$ with expectation value $\varphi$.
	
\Cref{eq:WetterichEq0d} manifests as a \pde{} for the scale-dependent \eaa{} $\bar{\Gamma} ( t, \varphi )$ in the $t$-$\varphi$-plane,
\begin{align}
	\partial_t \bar{\Gamma}(t,\varphi) = \frac{1}{2} \,\frac{1}{\partial_\varphi^2 \bar{\Gamma}(t,\varphi) + r ( t )}\,\partial_t r ( t ) \, ,	\label{eq:pde_gamma}
\end{align}
with the \ic{} $\bar{\Gamma} ( t = 0, \varphi ) = \mathcal{S} [ \varphi ]$.
Some remarks are in order:
\begin{enumerate}
	\item	In contrast to the \pdes{} for $\mathcal{Z} ( t, J )$ and $\mathcal{W} ( t, J )$ the Wetterich equation can be initialized with a suitable \ic{} at $t = 0$ that produces distinct flows for different actions $\mathcal{S}[\phi]$, as was discussed in the previous \cref{subsubsec:scale_dependent_effective_action}.
	
	\item	The spatial \bcs{}, \ie{}, for $\varphi \rightarrow \pm \infty$ are provided by the asymptotics of the Wetterich equation \eqref{eq:pde_gamma} itself and by the requirement that $\mathcal{S} [ \varphi ]$ must be bounded from below: The action $\mathcal{S} [ \varphi ]$ of an (interacting) field theory must at least grow like $\varphi^2$ for large $| \varphi |$ and the dominant contribution for large $| \varphi |$ must be even in $\varphi$.
	For actions $\mathcal{S} [ \varphi ]$ that grow asymptotically faster than $\varphi^2$ the denominator on the \rhs{} of the \pde{} \eqref{eq:pde_gamma} already diverges at $t \approx 0$, such that
	\begin{align}
		\lim\limits_{| \varphi | \rightarrow \infty} \partial_t \bar{\Gamma} ( t, \varphi ) \approx 0 \, .
	\end{align}
	It follows that for $| \varphi | \rightarrow \infty$ the function $\bar{\Gamma} ( t, \varphi )$ does not change at all, but keeps its initial value $\mathcal{S} [ \varphi ]$.
	These are perfectly valid \bcs{} for a \pde.
	The scenario for \ics{} with $\lim_{| \varphi | \rightarrow \infty} \mathcal{S} [ \varphi ] \sim \varphi^2$ is more delicate.
	We will return to this issue and a detailed discussion of \bcs{}, when we discuss the numerical implementation and solution of \cref{eq:pde_gamma} in \cref{subsec:boundary_conditions_finite_volume} in the context of numerical fluid dynamics.
	
	\item	The \pde{} \eqref{eq:pde_gamma} can be recast into a non-linear diffusion equation in the spirit of \cref{subsec:RGflow}, \viz{} \cref{eq:dWetterichEqFlow} by considering the flow equation for $\partial_t (\partial_\varphi\bar{\Gamma}(t,\varphi))$:
	\begin{align}
		\partial_t \del{\partial_\varphi\bar{\Gamma}(t,\varphi)}\, =\, \partial_\varphi\frac{\frac{1}{2} \, \partial_t r ( t )}{\partial_\varphi(\partial_\varphi\bar{\Gamma}(t,\varphi)) + r ( t )} \,=\,
		-\frac{\partial_t r ( t )}{\del{\partial_\varphi(\partial_\varphi\bar{\Gamma}(t,\varphi)) + r ( t )}^2} \partial_\varphi^2\del{\partial_\varphi\bar{\Gamma}(t,\varphi)}
		\, .	\label{eq:pde_dgamma}
	\end{align}
	In contrast to the \pdes{} \eqref{eq:pde_z} and \eqref{eq:pde_w} it is non-linear in the second-order spatial derivatives of $\Gamma ( t, \varphi )$.
	Using the flow equation for $\bar{\Gamma} ( t, \varphi )$ and its \ic{}, a worthwhile subject of future work could be a study of the resulting flows for $\mathcal{Z} ( t, J )$ and $\mathcal{W} ( t, J )$.
	By applying the same formalism to models with different field content, the \frg{} flow equations can also acquire convective/advective terms and source terms, see \cref{subsubsec:conservative_form,subsec:0dSU2flowEqs} for a detailed discussion.
			
	\item	In zero dimensions, similar to the flow equations for $\mathcal{Z} ( t, J )$ and $\mathcal{W} ( t, J )$, one can reparameterize the flow time $t$ in terms of $r$ in \cref{eq:pde_gamma} and get rid of the prefactor $\partial_t r ( t )$.
	Additionally, one could eliminate $r ( t )$ in the denominator in \cref{eq:pde_gamma} by shifting $\bar{\Gamma} ( t, \varphi ) \rightarrow \bar{\Gamma} ( r, \varphi ) - \tfrac{1}{2} \, r \, \varphi^2$ and switching from $t$ to $r$ as flow parameter, which corresponds to the zero-dimensional analogue of the rescaled  ``dimensionless'' flow equation in fixed-point form, but is not suited for most practical calculations in this work. The exception is our discussion in \cref{subsubsec:c-theorem_irreversibility_entropy}.
	
	This reparameterization effectively corresponds to different choices of regulator (shape) functions in zero dimensions.
	However, for higher-dimensional problems, different choices of regulators do not need to be related to each other via simple reparametrization of the \rgtime{}.
	In any case, the effective dynamics in the \pde{} during the \frg{} flow strongly depends on the parametrization of the \rgscale{} as well as the explicit choice of regulator, which has two direct consequences:
	First, although the dynamics and $t$-evolution of observables (the \nptFunctions{}) during the \frg{} flow might be highly interesting and must also be studied to ensure that the \uv{} and \ir{} cutoff scales are chosen appropriately, one must clearly state that only the \ir{} value of $\Gamma [ \varphi ]$ is physically meaningful.
	This is demonstrated and discussed again in the context of numerical precision tests of the $O(N)$ model in \cref{subsec:0dONresults}.
	Second, from a numerical point of view, some parametrizations or choices of regulators might be more challenging for the numerical integrators than others and must be adopted to the specific problems at hand.
	On the level of the PDE this corresponds to the time-dependent strength of the diffusion, \cf{} \cref{paragraph:conservative_form_diffusion}.
\end{enumerate}
Using a zero-dimensional field theory with one degree of freedom, we have therefore demonstrated that it is possible to transform the problem of solving functional integrals like \cref{eq:expectation_value_1} and \eqref{eq:partition_function} for a model with action $\mathcal{S}[\phi]$ into solving the \pde{} \eqref{eq:pde_gamma} in $t$ and $\varphi$ with the \ic{} $\mathcal{S} [ \varphi ]$.
The Wetterich \cref{eq:WetterichEq0d} thus directly implements the idea of transforming Gaussian-type functional integrals into arbitrary functional integrals, but on the level of the effective action $\Gamma[\varphi]$ rather than the partition function $\mathcal{Z}[J]$.
Both formulations of the problem of calculating $n$-point correlation functions \dash{} the functional-integral formulation and the \frg{} formulation \dash{} are mathematically equivalent.
This, however, is, as we have seen, a highly non-trivial statement and demands for numerical precision tests, which are part of this work.

For our following discussions in \cref{sec:0dON} we want to extent the current scope by including additional scalars in our theory, respecting \ON{} symmetry.
On a conceptual level this is a straight forward extension and its is of course possible to derive the generalization, \cf{} \eqref{eq:flow_equation_effective_potential}, of the flow equation \eqref{eq:pde_gamma} to $N$ scalars under \ON{} symmetry by repeating the derivation of this section.
After having derived the Wetterich equation twice already, first resulting the general \cref{eq:WetterichEq} and then in this subsubsection resulting in \cref{eq:pde_dgamma}, we will refrain from a third derivation and make use the general expression \eqref{eq:WetterichEq} in \cref{subsubsec:exact_flow_equation_potential} instead.