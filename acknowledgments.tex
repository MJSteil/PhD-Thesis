To start off the rather long list of people, collaborations, and institutions I want to acknowledge, I want to thank my supervisor, Michael Buballa, for his years of continuous support dating back to the days of my Bachelor’s thesis.
I am very grateful to him for his immense knowledge, excellent physical intuition, seemingly infinite patience, and his friendly and appreciative demeanor. % (especially regarding effective models, inhomogeneous phases, and the mean-field approximation)
I also want to thank him for giving me the opportunity to conduct my research very freely and interest-driven, resulting in this thesis and the accompanying publications.

I want to thank Jens Braun for his support throughout my research.
Especially his expertise regarding the \grg{}, continued interest, support, and encouragement have been invaluable.
As a member of my PhD committee, collaborator, and second referee of this thesis he made immense contributions to my research.

Furthermore, I want to thank Tetyana Galatyuk and Michael Vogel for joining my examination committee as the third and fourth examiner, respectively.

Dirk Rischke in his role as external supervisor in my PhD committee and principal investigator in the A03 project of the CRC-TR 211 has supervised my research from the start of my PhD project.
%His expertise in (quantum) field theory and hydrodynamics has been an immense help.
Especially his input regarding numerical schemes for conservation laws and \cfd{} has led to successful research efforts adapting the \kt{}/\muscl{} scheme to \frg{} flow equations.

I want to thank Guy D. Moore for his support and various interesting conversations ranging from physics over particularities of the German and English languages to interesting trivia.\bigskip

I acknowledge support by the \textit{Deutsche Forschungsgemeinschaft} (DFG, German Research Foundation) through the CRC-TR 211 ``Strong-interaction matter under extreme conditions'' \dash{} project number 315477589 \dash{} TRR 211.
I am grateful to the \textit{Helmholtz Graduate School for Hadron and Ion Research} for their interesting and stimulating scientific program, the excellent soft skill courses, and the travel funds I was able to enjoy due to my membership in the graduate school.
Moreover, I acknowledge the support of \textit{Stiftung Giersch} through the Giersch Excellence Grant and I want to thank Michael Buballa for the respective recommendation.\bigskip

I want to explicitly acknowledge various collaborators and colleagues.
First among them is Adrian Koenigstein, to whom I am immensely grateful.
Our close and very productive collaboration over the years is among the aspects of my PhD studies I cherish most.
His immense knowledge and his patience to share and explain at length improved my understanding of theoretical physics tremendously.
%The work on various joint publication and projects has been very fulfilling and educational.
His keen perception and attention to detail makes learning and researching together very rewarding.
Our joint talks and contributions at various seminars, colloquia, and conferences have been a joy to prepare and present.
The trips together to CRC events and conferences, and our brief research stay in Poland are among the most memorable moments of my time as a PhD student.

I want to thank Bernd-Jochen Schaefer and Michael Buballa for their work, input, and patience regarding our collaboration dealing with inhomogeneous chiral condensates in the \qmm{} with the \frg{} framework.

I am very thankful to Eduardo Grossi and Nicolas Wink for various discussions regarding the \grg{}, numerics, and hydrodynamics and their contributions to our joint project studying the \frg{} and numerics for it with zero-dimensional models.

The discussions and joint work on the \frg{} treatment of the \gnm{} with Niklas Zorbach, Jonas Stoll, Stefan Rechenberger, and Adrian Koenigstein have been very fulfilling.
Especially the critical and at times persistent questions and discussions coming up in this collaboration have pushed my own understanding of the \grg{} and the large-$N$/\mf{} approximation.

I want to thank Adrian Koenigstein, Laurin Pannullo, Stefan Rechenberger, and Marc Winstel for their work and fruitful discussions regarding our collaboration and publication dealing with the stability analysis in the \gnm{} on \mf{} level.

I am very thankful to Lennart Kurth for various discussions and critical questions regarding the \grg{}, the \mf{} approximation, and (thermal) \qft{} in general.
Our discussions and computations dealing with the \qmm{} and the \frg{} have been extremely educational and productive.

I want to thank Deniz Nitt for his expertise in C/\Cpp{}, his input at various group meetings, our unforgettable week in Schleiden, and his active participation in our ``Codenames+'' online activities during the COVID-19 pandemic.\bigskip

The years at the Theoriezentrum as a member of the NHQ group have been the most enjoyable out of my years at the TU Darmstadt.
I want to thank all my colleges at the Theoriezentrum for creating a great working and research environment and for all the great memories of lunch together, evening activities, Christmas parties, and joint trips.
%I am very grateful to the present and former members the NHQ group.%, especially to Professor Jochen Wambach, Michael Buballa, Stefano Carignano, Marco Schram, Philipp Isserstedt, Tobias Schulz, Dominic Kraatz, Steven Vereeken, Matthias Seubert, Daniel Nevermann, Carsten Wandschura, Deniz Nitt, Lennart Kurth, Christian Niehof, Dominik Erb, Patrick Falk, Marco Hofmann, and Hosein Gholami.
I am especially grateful to my past office mates of Room 306 \dash{} particularly Marc Barroso, Max Eller, Thomas Jahn, Parikshit Junnarkar, Vincent Klaer, Daniel Robaina, and Niels Schlusser \dash{} for countless hours of not only work and physics but also fun.\bigskip

%During my PhD studies I have been fortunate enough to be part of various research groups and projects.
%Discussions, presentations, and meetings (even during the challenging COVID-19 pandemic) have been enlightening, stimulating, and an integral part of my studies.

As member of the A03 project of the CRC-TR 211, I want to thank my colleagues in this project and the participants of our regular seminar meetings.
Our discussions about inhomogeneous phases, low-dimensional and low-energy-effective models, the \frg{}, and physics as well as numerics in general have been very enlightening and productive.
%I am especially grateful to Michael Buballa, Florian Divotgey, Jürgen Eser, Lutz Kiefer, Adrian Koenigstein, Lennart Kurth, Phillip Lakaschus, Laurin Pannullo, Stefan Rechenberger, Dirk Rischke, Bernd-Jochen Schäfer, Alexander Stegemann, Marc Wagner, and Marc Winstel.

I want to thank all members of the CRC-TR 211 for providing me with a stimulating research environment.
Our colloquia and retreats and especially the possibility to present my research at those events has been immensely educational and helpful.
I am especially grateful to Alessandro Sciarra for his immense engagement in the actives of the CRC-TR 211 and his expertise regarding good practices in coding, scientific data management, and \LaTeX.\bigskip

During my PhD studies, I was able to visit the universities in Bielefeld, Frankfurt, Gießen, and Heidelberg on numerous occasions.
I want to thank the colleagues from the respective institutes for their hospitality and interesting discussions.
In this context, I want to explicitly thank Bernd-Jochen Schaefer, Lorenz von Smekal, Konstantin Otto, Christopher Busch, Jan M. Pawlowski, Eduardo Grossi, Nicolas Wink, and Friederike J. Ihssen explicitly.\bigskip

I want to thank the multitude of people who have proofread parts of this manuscript for their comments and keen eyes concerning spelling, grammar, and commas.
I am very grateful to all the participants of my ``PhD Thesis Spellcheckers'' group chat.
Especially Michael Buballa, Jens Braun, Hanna Kniss, Sebastian Reinig, and Rebekka Allahyar Parsa deserve special recognition and thanks in this context.\bigskip

To conclude, I want to express my heartfelt gratitude to all my friends and family.
I thank Oliver Reinig, Simon Amann, and Sebastian Reinig for over twenty years of close friendship.
They, together with my friends Sebastian Arncken and Kalina Nikolaeva, have been and still are an invaluable support and a source of joy and happiness. 

Last but most certainly not least, I would like to thank my family.
Without their ongoing support and encouragement this work would have been impossible.
Their unconditional love \dash{} in some painful cases transcending death \dash{} has been and is a constant source of strength and safety throughout my life.
I am especially grateful to all the (extended) members of the Kniss family \dash{} Herta \& Erhard, Annina \& Torsten, Hanna \& Luca, Teresa \& Ole, and Hannes.
I am also profoundly thankful to my sister Rebekka, brother-in-law Ehsan, stepmother Gerda, father Winfried, and my beloved late mother Renate.