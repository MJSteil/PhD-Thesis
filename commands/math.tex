% ****** Mathematical commands ****** %
% *** Miscellaneous commands ***
\newcommand{\ttfrac}[2]{\raisebox{0.78pt}{\scalebox{0.8}{$\genfrac{}{}{}{3}{#1}{#2}$}}} % Small fraction for use in inline text/subscripts

% Spaces
\newcommand{\verthinmuskip}{1.0mu}
\newcommand{\vts}{\mskip\verthinmuskip}
% cf. [https://www.overleaf.com/learn/latex/Spacing_in_math_mode]
%\quad 	space equal to the current font size (= 18 mu)
%\, 	3/18 of \quad (= 3 mu) % thin space
%\: 	4/18 of \quad (= 4 mu) % medium space
%\; 	5/18 of \quad (= 5 mu) % thick space
%\! 	-3/18 of \quad (= -3 mu) % negative thin space
%\ (space after backslash!) 	equivalent of space in normal text
%\qquad 	twice of \quad (= 36 mu) 

% Symbols
\newcommand{\skeleton}[1]{%
	\mathchoice%
		{\displaystyle\ll\mkern-5mu#1\mkern-5mu\gg}%
		{\ll\mkern-5mu#1\mkern-5mu\gg}%
		{\scriptstyle\ll#1\gg}%
		{\scriptscriptstyle\ll#1\gg}%
}

% Raise \chi slightly
\let\oldchi\chi
\renewcommand{\chi}{%
\mathchoice%
	{\raisebox{0.75pt}{$\displaystyle\oldchi$}}
	{\raisebox{0.75pt}{$\oldchi$}}
	{\raisebox{0.50pt}{$\scriptstyle\oldchi$}}
	{\raisebox{0.25pt}{$\scriptscriptstyle\oldchi$}}
}

% uwave
\NewDocumentCommand{\Uwave}{mO{-1.2}O{0.8}O{-2}}{%
	\ooalign{\raisebox{#2\height}{\scalebox{#3}{$\mkern#4mu\sim$}}\cr$#1$\cr}
}
\newcommand{\uwave}[1]{\Uwave{#1}[-1.2][0.8][-2]}

% Equal with text
\newcommand\hastobe{\stackrel{!}{=}}
\newcommand\stackequals[1]{\stackrel{\normalfont\footnotesize\mbox{#1}}{=}}
\newcommand\eqrefequals[1]{\stackrel{\normalfont\footnotesize\mbox{\eqref{#1}}}{=}}

% \thickzero since \mathds{0} does not exist
\newcommand{\thickzero}{
	\mathchoice%
		{\mkern5mu\ooalign{\raisebox{.394\height}{\scalebox{.59}{$\mkern-6.3mu\textstyle|\mkern-4.5mu|\mkern-4.5mu|$}}\cr\hidewidth$\displaystyle 0$\hidewidth\cr}\mkern5mu}
		{\mkern5mu\ooalign{\raisebox{.394\height}{\scalebox{.59}{$\mkern-6.3mu\textstyle|\mkern-4.5mu|\mkern-4.5mu|$}}\cr\hidewidth$\textstyle 0$\hidewidth\cr}\mkern5mu}
		{\mkern5mu\ooalign{\raisebox{.394\height}{\scalebox{.59}{$\mkern-6.3mu\scriptstyle|\mkern-4.5mu|\mkern-4.5mu|$}}\cr\hidewidth$\scriptstyle 0$\hidewidth\cr}\mkern5mu}
		{\mkern5mu\ooalign{\raisebox{.394\height}{\scalebox{.59}{$\mkern-6.3mu\scriptstyle|\mkern-4.5mu|\mkern-4.5mu|$}}\cr\hidewidth$\scriptstyle 0$\hidewidth\cr}\mkern5mu}
}


% *** Matrices  ***
\newcommand{\Id}{\mathds{1}} % Identity matrix
\newcommand{\Zero}{\thickzero} % Zero matrix
\newcommand{\tr}{\top} % transpose sign (alternative \intercal)

% *** Sets of numbers in blackboard bold  ***
\newcommand{\Reals}{\mathds{R}}
\newcommand{\Integers}{\mathds{N}}
\newcommand{\Complex}{\mathds{C}}

% *** Mathematical constants ***
\newcommand{\eu}{\mathrm{e}} % Euler's number
\newcommand{\iu}{\mathrm{i}\mkern1mu} % Imaginary unit
\newcommand{\piu}{\mkern1.0mu\uppi\mkern1.0mu} % pi 
\newcommand{\gammau}{\upgamma} % upright gamma
\newcommand{\emc}{\upgamma} % Euler–Mascheroni constant
\newcommand{\emcApprox}{\upgamma\simeq0.577216} % Euler–Mascheroni constant \simeq 0.577215
\newcommand{\const}{\mathrm{const.}} % constant addition
\newcommand{\lcs}{\varepsilon} % Levi-Civita Symbol
\newcommand{\grad}{\nabla} % gradient
\newcommand{\order}{\mathcal{O}} % order symbol
\newcommand{\ordern}[1]{\mathcal{O}(#1)} % order

% *** Operators ***
\newcommand{\difx}{\dif_{\mkern4mu x}} % total derivative w.r.t x
\newcommand{\dift}{\dif_{\mkern4mu t}} % total derivative w.r.t t

\DeclareMathOperator{\Tr}{Tr} % trace
\DeclareMathOperator{\STr}{STr} % super trace

\DeclareMathOperator{\arsinh}{\mathrm{arsinh}}
\DeclareMathOperator{\arcoth}{arcoth}
\DeclareMathOperator{\artanh}{artanh}

\newcommand{\Det}{\ensuremath{\mathrm{Det}}}
\newcommand{\diag}{\ensuremath{\mathrm{diag}}}

\DeclareMathOperator{\DLi}{DLi} % DLi

\DeclareMathOperator*{\sumint}{% Sum-Integral operator [https://tex.stackexchange.com/a/68357]
\mathchoice%
	{\ooalign{$\displaystyle\sum$\cr\hidewidth$\displaystyle\int$\hidewidth\cr}}
	{\ooalign{\raisebox{.14\height}{\scalebox{.7}{$\textstyle\sum$}}\cr\hidewidth$\textstyle\int$\hidewidth\cr}}
	{\ooalign{\raisebox{.2\height}{\scalebox{.6}{$\scriptstyle\sum$}}\cr$\scriptstyle\int$\cr}}
	{\ooalign{\raisebox{.2\height}{\scalebox{.6}{$\scriptstyle\sum$}}\cr$\scriptstyle\int$\cr}}
}

\newcommand{\smallergtrsim}{\mkern1.5mu\raisebox{-0.5pt}{\scalebox{0.8}{$\gtrsim$}}\mkern1.5mu}
\newcommand{\smallerlesssim}{\mkern1.5mu\raisebox{-0.5pt}{\scalebox{0.8}{$\lesssim$}}\mkern1.5mu}

% *** O(N), SU(2), and Z_2 commands *** %
\newcommand{\ON}{\texorpdfstring{$O(N)$}{O(N)}}
\newcommand{\ONn}[1]{\texorpdfstring{$O(#1)$}{O(#1)}}

\newcommand{\SU}[1]{\texorpdfstring{$SU(#1)$}{SU(#1)}}
\newcommand{\suNt}[1]{T_{#1}}
\newcommand{\suNtup}[1]{T^{#1}}
\newcommand{\suNta}[1]{\tilde{T}^{#1}}
\newcommand{\suNU}{\mathcal{U}}
\newcommand{\suIIt}[1]{t_{#1}}
\newcommand{\suIItup}[1]{t^{#1}}
\newcommand{\suIItij}[3]{(t_{#1}){}^{#2}\vts{}_{#3}}
\newcommand{\suIIIdij}[2]{(\Id_t){}^{#1}\vts{}_{#2}}
\newcommand{\suIIta}[1]{\tilde{t}\mkern1.0mu{}^{#1}}
\newcommand{\suIItvec}{\vec{t}}

\newcommand{\uIIt}[1]{t_{#1}}
\newcommand{\uIItup}[1]{t^{#1}}

\newcommand{\suL}{\mathrm{L}}
\newcommand{\suR}{\mathrm{R}}
\newcommand{\suA}{\mathrm{A}}
\newcommand{\suV}{\mathrm{V}}

\newcommand{\ZII}{\texorpdfstring{$\mathds{Z}_2$}{Z2}}

% *** Multiplication wrappers \cm and \ncm ***
\newcommand{\cm}[1]{%
	\foreach \entry [count=\ni] in {#1} {%
		\ifnum\ni=1{\entry}\else{\mkern1mu\entry}\fi%
	}%
}

\newcommand{\ncm}[1]{%
	\foreach \entry [count=\ni] in {#1} {%
		\ifnum\ni=1{\entry}\else{\mkern1.5mu\entry}\fi%
	}%
}

% *** Primed equations ***
% WARNING: Use this only inside align to avoid very weird referencing errors
% WARNING: Never manually iterate an equation counter outside an equation environment! To again avoid very weird and incorrect crossreferencing without any warnings or errors
\newcounter{eqPrime}
\renewcommand{\theeqPrime}{\theequation$'$}
\newcommand{\newEqBlockPrime}{\setcounter{eqPrime}{0}}
\newcommand{\eqTagPrime}{\stepcounter{eqPrime}\tag{\theeqPrime}}

% *** Subequations ***
% WARNING: Use this only inside align to avoid very weird referencing errors 
% WARNING: Never manually iterate an equation counter outside an equation environment! To again avoid very weird and incorrect crossreferencing without any warnings or errors
% For a plain block of subequations use subequations environment: \begin{subequations}\label{eq:...}\begin{align}...\end{align}\end{subequations} 
\newcounter{subeqMain}
\setcounter{subeqMain}{0}
\newcounter{subeq}
\newcounter{subeqPrime}
\renewcommand{\thesubeq}{\theequation\alph{subeq}}
\renewcommand{\thesubeqPrime}{\theequation\alph{subeqPrime}$'$}

\newcommand{\newSubEqBlock}{\setcounter{subeq}{0}}
\newcommand{\newSubEqBlockPrime}{\setcounter{subeqPrime}{0}}

\newcommand{\subEqTag}{\refstepcounter{subeqMain}\stepcounter{subeq}\tag{\thesubeq}}
\newcommand{\subEqTagPrime}{\refstepcounter{subeqMain}\stepcounter{subeqPrime}\tag{\thesubeqPrime}}