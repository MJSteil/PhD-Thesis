\section{Units}
For equations and numerical computations in this work we use a \nun{} system in which the units are defined such that the Boltzmann constant $k_\mathrm{B}$, the reduced Planck constant $\hbar$, and the speed of light in vacuum $c$ are dimensionless and exactly unity, \ie{},
\begin{align}
	(\kB)_\mathrm{NU}  \equiv  (\hbar)_\mathrm{NU}  \equiv  (c)_\mathrm{NU}  \equiv  1 \, .
\end{align}
This \nun{} system simplifies equations and numerical computations and it is realized by using energy as dimension for the quantities mass and temperature and reciprocal energy as dimensions for time and length.
We chose electronvolt ($\mathrm{eV}$) as the base unit of energy in our \nun{} system.
For discussions and figures which benefit/require the use of units, we usually adopt units common in \hep{} which are obtained from results in \nun{} by multiplication with appropriate powers of the conversion factor
\begin{align}
	\hbarc = 197.328\,980\,4\ldots\, \MeVfm \,,
\end{align}
see, \eg{}, \cref{eq:MeV4fm3,eq:MeV4MeVfm3}.

Selected conversion factors to SI units and derived units are
\begin{alignat}{2}
	\big(1\,\MeV\big)_\mathrm{NU} &= 1\,\MeV \cdot c^{-2} &&= 1.782\,661\,9\ldots\,\times10^{-30}\,\mathrm{kg} \, ,\\
	\big(1\,\MeV^{-1}\big)_\mathrm{NU} &= 1\,\MeV^{-1} \cdot \hbarc &&= 1.973\,269\,8\ldots\,\times10^{-13}\,\mathrm{m} \, ,\\
	\big(1\,\MeV^{-1}\big)_\mathrm{NU} &= 1\,\MeV^{-1} \cdot \hbar &&= 6.582\,119\,5\ldots\,\times10^{-22}\,\mathrm{s} \, ,\\
	\big(1\,\MeV\big)_\mathrm{NU} &= 1\,\MeV \cdot 1&&= 1.602\,176\,6\ldots\,\times10^{-13}\,\mathrm{J} \, ,\\
	\big(1\,\MeV\big)_\mathrm{NU} &= 1\,\MeV \cdot \kB^{-1} &&= 1.160\,451\,8\ldots\,\times10^{+10}\,\mathrm{K} \, ,\\
	\big(1\,\MeV^3\big)_\mathrm{NU} &= 1\,\MeV^3 \cdot (\hbarc)^{-3} &&= 1.301\,489\,2\ldots \times10^{-7}\,\fmIII \, , \label{eq:MeV4fm3}\\
	\big(1\,\MeV^4\big)_\mathrm{NU} &= 1\,\MeV^4 \cdot (\hbarc)^{-3} &&= 1.301\,489\,2\ldots \times10^{-7}\,\MeVfmIII \, ,\label{eq:MeV4MeVfm3}
\end{alignat}
for mass, length, time, energy, temperature, number density, and energy density respectively.
Depending on the context pressure and density are typically expressed in units of $\MeVfmIII$ in \hep{} with the following conversion factors
\begin{align}
	1\,\MeVfmIII& \mathrel{\widehat{=}} 1.602\,176\,6\ldots\,\times10^{+32}\,\mathrm{Pa}%
	\mathrel{\widehat{=}} 1.782\,661\,9\ldots\,\times 10^{+15}\, \mathrm{kg}\mkern1.0mu\mathrm{m}^{-3}
\end{align}
to the respective (derived) SI units. 

For the exact SI values of the fundamental constants \dash{} \viz{} $k_\mathrm{B}$, $\hbar$, $c$, and the electronvolt in joules \dash{} used in our numerical computations we use the recommended values of CODATA~\cite{CODATA2018}.