\section{Field space notation}\label{app:FS}
The \acrlong{fs}{} notation introduced in this appendix is based on the one put forward in Refs.~\cite{Pawlowski:2005xe,Rennecke:2015lur,PawlowskiScript} and allows for a unified treatment of fermionic (Grassmann-valued) and bosonic (non-Grassmann-valued) fields.\bigskip

In the following we collect all Grassmann-valued fields of the theory under consideration in $\FSf{\MFpsi}$ and $\FSf{\MFpsib}$ and all other fields in $\FSf{\MFphi}$.
In the FS notation introduced in this appendix, we will suppress all discrete and continuous internal indices of those fields and collect all fields in one multi-field $\FSf{\chi}$.
FS indices, related to components of the multi-field $\FSf{\chi}$, are typeset as bold roman indices.
The multi-field $\FSf{\chi}$ includes all fundamental fields of the theory as well as 
all optional composite fields under consideration.
The FS components of this multi-field
\begin{align}
	(\FSfd{\FSsf}{a})&\equiv (\FSf{\MFphi},\FSf{\MFpsi},\FSf{\MFpsib})^\tr \, ,\label{eq:FSfd}\\
	(\FSfu{\FSsf}{a})&\equiv (\FSf{\MFphi},\FSf{\MFpsib},-\FSf{\MFpsi}) \, ,\label{eq:FSfu}
\end{align}
are chosen to implement the contraction
\begin{align}
	\FSfu{\FSsf}{m}\FSfd{\FSsf}{m}
		=\FSguu{m}{n}\FSfd{\FSsf}{n}\FSfd{\FSsf}{m}
		=\FSfu{\FSsf}{m}\FSfu{\FSsf}{n}\FSgdd{n}{m}
		=\FSf{\MFphi}^2+\FSf{\MFpsib}\FSf{\MFpsi}-\FSf{\MFpsi}\FSf{\MFpsib}=\FSf{\MFphi}^2+2\FSf{\MFpsib}\FSf{\MFpsi} \, ,\label{eq:FSchi2contraction}
\end{align}
where we introduced the FS metric\footnote{%
	Our FS metric of \cref{eq:FSguugdd} is consistent with the sign conventions used in \ccite{Rennecke:2015lur,PawlowskiScript} while in \ccite{Pawlowski:2005xe} the signs in the FS metric of Eq.~(A.1) should be switched for Eqs.~(A.2) and (A.7) of \ccite{Pawlowski:2005xe} to be consistent.
	The $\pm1$ for the components of the FS metric related to Grassmann-valued fields are just convention ($\mp1$ is also possible) but they affect the relation between $(\FSfd{\FSsf}{a})$ and $(\FSfu{\FSsf}{a})$, see \cref{eq:FSfd,eq:FSfu}, and consequently signs in contractions like the one in \cref{eq:FSchi2contraction}.%
}
\begin{align}
	\del[1]{\FSguu{a}{b}}&\equiv\del[1]{\FSgdd{a}{b}}\equiv
		\begin{pmatrix}
			1 & 0 & 0\\
			0 & 0 & 1\\
			0 & -1 & 0
		\end{pmatrix}\label{eq:FSguugdd}
\end{align}
and the Northwest-Southeast (NW-SE) convention: FS indices are always raised (N) from the left (W), $\FSfu{\FSsf}{a}= \FSguu{a}{m}\FSfd{\FSsf}{m}$, and  lowered (S) from the right (E), $\FSfd{\FSsf}{a}= \FSfu{\FSsf}{m}\FSgdd{m}{a}$.
Summation over FS and discrete internal indices as well as integration over continuous internal variables is implied for contractions.

The FS metric is non-diagonal in FS, which is necessary to take the Grassmann nature of some field components into account.
In all, in our notation currently suppressed, internal spaces the metric is trivial and the respective identity.
The FS metric introduced above has the following properties
\begin{alignat}{2}
	\FSgdu{a}{b} &= \FSgdd{a}{m}\FSguu{b}{m} &&= \FSgdd{m}{a}\FSguu{m}{b} = \FSd{b}{a} \, ,\\
	\FSgud{a}{b} &= \FSguu{a}{m}\FSgdd{m}{b} &&= \FSguu{m}{a}\FSgdd{b}{m} = \FSc{\FSidx{a},\FSidx{b}}\FSd{a}{b} \, ,
\end{alignat}
where we introduced the generalized sign factor
\begin{align}
	\FSc{\FSidx{a},\FSidx{b}}\equiv%
	\begin{cases}%
		-1 & \text{if the components $\FSidx{a}$ and $\FSidx{b}$ are Grassmann-valued}\\
		+1 &\text{otherwise}%
	\end{cases} \, ,\label{eq:FScDef}
\end{align}
which tracks possible sign flips due to the commutation of FS components $\FSidx{a}$ and $\FSidx{b}$.
Products of such sign factors appear frequently in practical computations therefore we introduce the compact notation
\begin{align}
	\FSc{\FSidx{a},\FSidx{b},\FSidx{c},\FSidx{d},\FSidx{e},\FSidx{f}}\equiv\FSc{\FSidx{a},\FSidx{b}}\FSc{\FSidx{c},\FSidx{d},\FSidx{e},\FSidx{f}} \equiv \FScSkeleton{\FSidx{a},\FSidx{b}}{2}{}\, . \label{eq:FScAbr}
\end{align}

The defined metric together with the NW-SE convention implements the anticommutation of Grassmann-valued fields properly, \eg,
\begin{align}
	\FSfu{\FSsf}{m}\FSfd{\FSsf}{m}
		=\FSguu{m}{n}\FSfd{\FSsf}{n}\FSfu{\FSsf}{l}\FSgdd{l}{m}
		=\FSfd{\FSsf}{n}\FSfu{\FSsf}{l}\FSgud{n}{l}
		=\FSc{\FSidx{n},\FSidx{l}}\FSd{n}{l}\FSfd{\FSsf}{n}\FSfu{\FSsf}{l}
		=\FSc{\FSidx{m},\FSidx{m}}\FSfd{\FSsf}{m}\FSfu{\FSsf}{m} \, ,
\end{align}
which is of course consistent with \cref{eq:FSchi2contraction}.

Sources related to components of $\FSf{\FSsf}$ can also be treated in the introduced notation
\begin{align}
	(\FSfd{J}{a})&\equiv (J_{\FSf{\MFphi}},J_{\FSf{\MFpsib}}, J_{\FSf{\MFpsi}}),\\
	(\FSfu{J}{a})&=(\FSguu{a}{m}\FSfd{J}{m})= (J_{\FSf{\MFphi}},J_{\FSf{\MFpsi}}, -J_{\FSf{\MFpsib}}) \, ,
\end{align}
which implements
\begin{align}
	\FSfu{J}{m}\FSfd{\FSsf}{m}=\FSc{\FSidx{m},\FSidx{m}}\FSfu{\FSsf}{m}\FSfd{J}{m}
		=J_{\FSf{\MFphi}}\FSf{\MFphi}+ J_{\FSf{\MFpsi}}\FSf{\MFpsi}- J_{\FSf{\MFpsib}}\FSf{\MFpsib}
		=J_{\FSf{\MFphi}}\FSf{\MFphi}+ J_{\FSf{\MFpsi}}\FSf{\MFpsi}+ \FSf{\MFpsib}J_{\FSf{\MFpsib}} \, .
\end{align}

Functional derivatives in FS are always taken from the left.
The  order of those derivatives is important since derivatives \wrt{} Grassmann-valued fields or sources anticommute with Grassmann-valued components.
For the product rule this implies
\begin{align}
	\frac{\delta}{\delta \FSfu{J}{b}} \del{ \frac{\delta f[\FSf{J}\mkern1.5mu]}{\delta \FSfu{J}{a}} g[\FSf{J}\mkern1.5mu] } =
		\del{\frac{\delta}{\delta \FSfu{J}{b}}\frac{\delta f[\FSf{J}\mkern1.5mu]}{\delta \FSfu{J}{a}}} g[\FSf{J}\mkern1.5mu]
		+\FSc{\FSidx{a},\FSidx{b}}\frac{\delta f[\FSf{J}\mkern1.5mu]}{\delta \FSfu{J}{a}} \frac{\delta g[\FSf{J}\mkern1.5mu]}{\delta \FSfu{J}{b}} \, , \label{eq:FSproductrule}
\end{align}
while for the chain rule
\begin{align}
	\frac{\delta}{\delta \FSfu{J}{a}} f[g[\FSf{J}\mkern1.5mu]] = f'[g[\FSf{J}\mkern1.5mu]] \frac{\delta g[\FSf{J}\mkern1.5mu]}{\delta \FSfu{J}{a}} \label{eq:FSchainrule}
\end{align}
applies in the usual manner.

For functional derivatives we frequently adopt the compact notations
\begin{align}
	\FSeaa_k^{,\row{\FSidx{x},\FSidx{m}_n,\mkern-1mu\ldots,\FSidx{m}_2,\FSidx{m}_1}}[\FSmf{\FSsf}]%
		\equiv\frac{\delta}{\delta \FSmfd{\FSsf}{x}}\FSeaa_k^{,\row{\FSidx{m}_n,\mkern-1mu\ldots,\FSidx{m}_2,\FSidx{m}_1}}[\FSmf{\FSsf}]%
		\equiv\frac{\delta}{\delta \FSmfd{\FSsf}{x}} \frac{\delta}{\delta \FSmf{\FSsf}_{\FSidx{m}_n}}\ldots \frac{\delta}{\delta \FSmf{\FSsf}_{\FSidx{m}_2}} \frac{\delta}{\delta \FSmf{\FSsf}_{\FSidx{m}_1}} \FSeaa_k [\FSmf{\FSsf}] \, ,\\
	W_{k,\row{\FSidx{x},\FSidx{m}_n,\mkern-1mu\ldots,\FSidx{m}_2,\FSidx{m}_1}}[\FSf{J}\mkern1.5mu]%
		\equiv\frac{\delta}{\delta \FSfu{J}{x}}W_{k,\row{\FSidx{m}_n,\mkern-1mu\ldots,\FSidx{m}_2,\FSidx{m}_1}}[\FSmf{\FSsf}]%
		\equiv \frac{\delta}{\delta \FSfu{J}{x}} \frac{\delta}{\delta \FSf{J}^{\FSidx{m}_n}}\ldots \frac{\delta}{\delta \FSf{J}^{\FSidx{m}_2}} \frac{\delta}{\delta \FSf{J}^{\FSidx{m}_1}} W_k [\FSf{J}\mkern1.5mu] \, ,
\end{align}
where additional derivatives are attached from the left, derivatives to the right are performed first, taking derivatives \wrt{} upper (lower) FS indices lowers (raises) the index in question and functional derivatives are understood as derivatives \wrt{} the argument of the functional in question.

For FS components of operators which are not directly related to functional derivatives and which already carry sub- or superscript characters we use a semicolon as separator, \eg, $\FSf{\FSsf}_{k;\FSidx{a}}[\FSff{\FSsf}]$ and $G_{k;\FSidx{ab}}[\FSmf{\FSsf}]$.