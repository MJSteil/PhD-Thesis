% Individual commands to generate standalone abstract
\renewcommand{\ie}{\texorpdfstring{\textit{i.e.}}{i.e.}}
\renewcommand{\viz}{\texorpdfstring{\textit{viz.}}{viz.}}
\renewcommand{\twoDimensional}{\texorpdfstring{${(1 \mkern-3mu + \mkern-3mu 1)}$-dimensional}{(1+1)-dimensional}}
\renewcommand{\fourDimensional}{\texorpdfstring{${(3 \mkern-3mu + \mkern-3mu 1)}$-dimensional}{(3+1)-dimensional}}
\renewcommand{\csb}{\texorpdfstring{$\chi$SB}{χSB}}
%
In this work, we study strongly interacting quantum field theories using the functional renormalization group (FRG) as our primary computational method.
The goal is to facilitate FRG computations in the context of quantum chromodynamics (QCD) to study the phase structure of dense strong-interaction matter.
The main part of this work is split into three chapters, differing in the space-time dimension of the theories under consideration.

We begin by studying zero-dimensional theories, which ultimately involves solving ordinary integrals with complicated FRG flow equations. 
Initially, this might seem like an unnecessarily convoluted way to solve a simple problem.
However, it is this very fact \dash{} applying the FRG to such simple theories \dash{} that allows us to gain enormous insights into the FRG in a rigorous manner.
Arguably, the most relevant development is the novel understanding of FRG flow equations in a fluid-dynamic context.
This allows for the application of methods and concepts from the highly developed field of computational fluid dynamics (CFD) to the FRG.
Two key findings are the identification of bosonic (fermionic) fluctuations as convective (source- or sink-like) contributions to the FRG flow and the resulting link between the CFD concept of numerical entropy and the irreversibility of non-perturbative renormalization group (RG) flows.
These developments serve as a vital stepping stone facilitating the following applications.

We proceed with computations in the \twoDimensional{} Gross-Neveu (GN) model.
We use it to study spontaneous chiral symmetry breaking (\csb{}) \dash{} a phenomenon vital to the understanding of QCD.
Using the previously established CFD methods for the FRG, we study the effects of fermionic and crucially bosonic quantum and thermodynamic fluctuations on spontaneous \csb{}.
The main result of this part of our research is that thermal bosonic fluctuations prevent \csb{} in the \twoDimensional{} GN model.
We further study inhomogeneous \csb{} indirectly using a stability analysis in mean-field (MF) approximation, \ie{}, considering only fermionic fluctuations.
Our research helps to establish this method as a robust tool for both qualitative and quantitative statements about inhomogeneous \csb{}.

We conclude the main part of this thesis with our studies of the \fourDimensional{} Quark-Meson (QM) model, which we primarily consider as a low-energy effective theory of QCD.
We focus on inhomogeneous chiral condensates by studying the QM model within the FRG framework, using a position-dependent ansatz for the chiral condensate, \viz{} the chiral density wave (CDW) for which we have been able to derive explicit FRG flow equations.
We again investigate the effects of fluctuations on spontaneous \csb{} by solving those flow equations in RG-consistent MF calculations.
Thus establishing contact with existing literature results for the QM model with CDW condensates.
These computations \dash{} incorporating only fermionic fluctuations \dash{} are a first step towards a complete solution of the derived flow equations using our established CFD methods.