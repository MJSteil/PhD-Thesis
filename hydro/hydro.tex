\section{Conservation laws, hydrodynamics, and the finite volume method}\label{sec:conservationLaws}
\begin{disclaimer}
	Parts of this section are based on Secs.~IV.B\dash{}C of \nbccite{Koenigstein:2021syz} and on the Apps.~B\dash{}E of \nbccite{Steil:2021cbu}. 
\end{disclaimer}
In this section we discuss systems of \pdes{} in one effective temporal direction $(t)$ and one effective spatial direction $(x)$ of the following generic form
\begin{align}
	&\dift\! u_i ( t, x ) + \difx\!  F_i [ t, x, \{u_i ( t, x )\} ] = \difx\! Q_i [ t, x, \{u_i ( t, x )\}, \{\partial_x u_i ( t, x )\} ]\,+\notag\\*[.2em] % no page break here
	&\hphantom{\dift\! u_i ( t, x ) + \difx\!  F_i [ t, x, \{u_i ( t, x )\} ] = \difx\! Q_i [ t, x, \{u_i ( t, x )\}xxx}+S_i [ t, x, \{u_i ( t, x )\} ] \, , \label{eq:FVadsEq}\\[.2em]
	&\dift\! u_i ( t, x ) + \difx\! C_i [ t, x, \{u_i ( t, x )\}, \{\partial_x u_i ( t, x )\} ] = S_i [ t, x, \{u_i ( t, x )\} ] \, , \label{eq:FVcsEq}
\end{align}
where we distinguish between advective contributions $\difx\! F_i[\ldots]$, diffusive contributions$\difx\! Q_i[\ldots]$, and source/sink contributions $S_i [\ldots]$.
Whether $S_i$ acts as a source or sink in the dynamics of $\{u_i(t,x)\}$ depends on its explicit form.
Nevertheless we will usually refer to $S_i$ as a source term.
We also introduced the convective contribution $\difx\! C_i[\ldots]\equiv \difx\! F_i[\ldots] - \difx\! Q_i[\ldots]$, which incorporates both advective and diffusive contributions for the sake of discussion. 
In this section we will explicitly distinguish between partial $(\partial_x\equiv\partial/\partial x)$ and total  $(\difx\!\equiv\!\dif/\!\dif x)$ derivatives.
In the following, we occasionally suppress the $t$- and $x$-dependencies of $u_i$, $C_i$, $F_i$, $Q_i$, and $S_i$ for the sake of simplicity.
Equation \eqref{eq:FVadsEq} is a system of \pdes{} describing the evolution of $m$ conserved quantities $\{u_i ( t, x )\}\equiv \{u_i\}\equiv\{ u_1 , \ldots , u_m \}$ in $t$ and $x$. 
Depending on the problem at hand these two directions are not necessarily identical with physical spatial and temporal dimensions, but for the following discussion we denote them as such.
The functions 
\begin{itemize}
	\item $F_i [ \{u_i\} ] \equiv F_i [ t, x, \{u_i ( t, x )\} ]$ are components of a non-linear advection flux, 
	\item $ Q_i [ \{u_i\} , \{\partial_x u_i\} ] \equiv Q_i [ t, x, \{u_i ( t, x)\} , \{\partial_x u_i ( t, x )\} ]$ are components of a non-linear diffusion(dissipation) flux, and 
	\item $S_i [ \{u_i\} ] \equiv S_i [ t, x, \{u_i ( t, x)\} ]$ are components of a source term.
\end{itemize}
The aforementioned fluxes are discussed in detail in \cref{subsec:hydroAdvection,subsec:hydroDiffusion,subsec:hydroSource} respectively.
The concepts discussed in the following can be generalized beyond one spatial dimension to \dxyDimensional{s}{1} space-time: $( x, t ) \rightarrow ( \vec{x}, t ) = ( x_1, \ldots, x_d, t )$.
Equation systems similar or even identical to \cref{eq:FVadsEq} are often referred to as conservation laws and appear in many areas of the natural sciences, engineering, and economics.
They are extensively studied in the field of \acrrepeat{cfd}.

A complete and self-contained introduction to conservation laws in the context of \cfd{} is beyond the scope of this thesis.
We rather focus on introducing the computational methods we employ, establishing relevant nomenclature, and discussing selected properties of the system~\eqref{eq:FVadsEq}.
We will discuss the different contributions to \cref{eq:FVadsEq} in \cref{subsec:hydroAdvection,subsec:hydroDiffusion,subsec:hydroSource,subsec:hydroEuler} using explicit educational examples.
The methodological introduction of \cref{subsec:hydroFV,subsec:hydroKT} is meant to facilitate the computations and discussions in the main part of this thesis, \ie{}, \cref{chap:zeroONSU2,chap:GN}.
For more details we refer the interested reader to the vast literature discussing conservation laws, \cf{} \ccite{LeVeque:1992,Eymard2000Jan,LeVeque:2002,Hesthaven2007,RezzollaZanotti:2013,Hunter2014,MishraLectureNotes,Vazquez-Cendon2015,polyanin2016handbook,Rezolla2020}.

This section has a corresponding digital auxiliary file~\cite{Steil:2023PhDFVNB}, which includes our latest \WAM{}-implementation of the \acrshort{kt}/\acrshort{knp} scheme discussed in \cref{subsec:hydroKT} and the code to produce all the explicit (numerical) examples discussed in \cref{subsec:hydroAdvection,subsec:hydroDiffusion,subsec:hydroSource,subsec:hydroEuler}.

\clearpage
\subsection{The finite volume method}\label{subsec:hydroFV}
\begin{disclaimer}
	This subsection follows the discussion in Sec.~IV.B of \nbccite{Koenigstein:2021syz} with slight modifications due to the fact that we want to discuss a system of conservation laws for $\{u_i ( t, x )\}$ instead of a single conservation law for $u ( t, x )$. 
\end{disclaimer}
In this section we discuss a numerical solution scheme for convection/advection-diffusion equations\footnote{%
	Oftentimes, such equations are also referred to as ``convection-diffusion equations''.
	The semantically correct term is nevertheless ``advection-diffusion equation'' because ``convection'' also includes diffusive processes besides the transport by bulk motion (advection), see also \ccite{LeVeque:1992}.%
} 
with source terms of the generic type \eqref{eq:FVcsEq}. 
Considering the convection equation~\eqref{eq:FVcsEq} with specified terms $F_i$, $Q_i$, and $S_i$ in a finite computational domain $\Omega = \mathcal{V} \times [ t_0, t_N ]$, where $\mathcal{V} \subset \Reals{}^1$ denotes the spatial volume, with an \ic{} $u_i( t_0, x )$ and Dirichlet (Neumann) \bcs{} specifying $( \partial_x ) u_i ( t, x ) |_{x \in \partial \mathcal{V}}$, the natural question arises how to evolve the \ic{} in time from $t_0$ to $t_N>t_0$ to acquire a solution $u ( t_N, x )$ respecting the specified \bcs{} \dash{} \ie{}, how to explicitly solve the posed initial-value problem.
For most problems of the type~\eqref{eq:FVcsEq} an analytic/symbolic explicit solution is not known or even considered to be non-existent.
Strategies for finding numerical (weak) solutions are required.
Numerical schemes in the broad class of so-called \acrrepeat{fv} methods are very popular for the numerical solution of \pdes{} describing the conservation or balance of quantities.
For additional details regarding especially the \fv{} method we refer the interested reader to the textbooks~\cite{LeVeque:2002,Vazquez-Cendon2015}.
Alternative \hrsc{} schemes in modern computational fluid dynamics are among others: finite-difference schemes including flux limiters/numerical viscosity or finite-element methods.

The concept that all numerical \fv{} methods share is a discretization of the computational domain into space-time control volumes $\mathcal{V}_j \times [ t^l, t^{l+1} ]$, where the set of spatial control volumes $\mathcal{V}_j$ covers the spatial computational domain $\mathcal{V}$.
Integrating \cref{eq:FVcsEq} over such a control volume centered at $x$, using the divergence theorem (Gauss-Ostrogradsky theorem) on the convection flux, and introducing the sliding cell average
\begin{align}
	\bar{u}_{i} ( t, x ) &\equiv \frac{1}{| \mathcal{V}_j |} \int_{\mathcal{V}_j} \dif\xi \, u_i( \xi, t )\\
	\Rightarrow \bar{u}_{i,j}(t)\equiv \bar{u}_{i} ( t, x_j ) &\equiv \frac{1}{\Delta x_j} \int_{x_{j-\frac{1}{2}}}^{x_{j+\frac{1}{2}}} \dif\xi \, u_i( \xi, t )\, ,\label{eq:FVaverage}
\end{align}
where $\mathcal{V}_j = \{ \xi : | \xi - x_j | \leq \Delta x_j/2 \}$, we arrive at an equivalent integral form of \cref{eq:FVcsEq},
\begin{align}
	\bar{u}_i ( t^{l+1}, x ) = \bar{u}_i ( t^l, x ) - \bar{C}_i[ t^{l+1},t^l, x,\{u_i\} ] + \bar{S}_i[ t^{l+1},t^l, x,\{u_i\} ]\label{eq:FVintEq}\,,
\end{align}
with the integral over the convection flux 
\begin{align}
	\bar{C}_i[ t^{l+1},t^l, x,\{u_i\} ]&\equiv \tfrac{1}{\Delta x_j}\int_{t^l}^{t^{l+1}} \dif\tau \, C_i \big[ \tau, x + \tfrac{\Delta x_j}{2}, u_i \big\{ \big( \tau, x + \tfrac{\Delta x_j}{2} \big) \big\} \big]\, -\notag\\* % no page break
	&\qquad\qquad-\tfrac{1}{\Delta x_j}\int_{t^l}^{t^{l+1}} \dif\tau \, C_i \big[ \tau, x - \tfrac{\Delta x_j}{2},  \big\{u_i \big( \tau, x - \tfrac{\Delta x_j}{2} \big)  \big\}\big]\,,\label{eq:FVcIntEq}
\end{align}
and the integral over the source term
\begin{align}
	\bar{S}_i[ t^{l+1},t^l, x,\{u_i\} ]&\equiv \frac{1}{\Delta x_j}  \int_{t^l}^{t^{l+1}} \dif\tau \int_{x_{j-\frac{1}{2}}}^{x_{j+\frac{1}{2}}} \dif\xi \, S_i [ \tau, \xi, \{u_i ( \tau, \xi )\} ] \,,\label{eq:FVsIntEq}
\end{align}
with $x\in\mathcal{V}_j$.
Considering a system without explicit source terms $(S_i=0)$, \cref{eq:FVintEq} with \cref{eq:FVcIntEq} implies that the change in the cell average $\bar{u}_i ( t^{l+1}, x )-\bar{u} ( t^l, x )$ is given by the time-integral over the convection flux \eqref{eq:FVcIntEq} \dash{} \ie{}, the time-integral over the in- and out-flux at the cell interfaces $x\pm\frac{\Delta x_j}{2}$.
Assuming appropriate closed \bcs{} on the compact spatial volume $\mathcal{V}$, the spatial integral over the cell averages $\bar{u}_{i} ( t, x )$ is conserved since change in the individual control volumes is only possible due to in- or out-flux through the cell interfaces into the neighboring control volumes \dash{} hence the name conservation laws.
In a system with closed \bcs{} changes in the sum/integral over the cell averages $\bar{u}_{i} ( t, x )$ are only possible due to source/sink terms $S_i$, which modify the conservation law according to \cref{eq:FVintEq} with \cref{eq:FVsIntEq}.
The reformulation of \cref{eq:FVcsEq} in terms of cell averages has several advantages which we will discuss in the following.
Arguably the most important one, which we want to mention right away, is the fact, that the integral/weak formulation of \cref{eq:FVintEq} in terms of cell averages allows for a proper treatment of discontinuous solutions $u_{i} ( t, x )$, which are notorious in the context of conservation laws.\bigskip

The solution of \cref{eq:FVintEq} \dash{} \ie{}, the explicit computation of \cref{eq:FVcIntEq,eq:FVsIntEq} \dash{} presents the central challenge for an explicit \fv{} scheme.
Details regarding the explicit resolution and computation of \cref{eq:FVcIntEq,eq:FVsIntEq} are discussed in \cref{subsec:hydroKT} and specifically in \cref{subsec:hydroAdvection,subsec:hydroDiffusion,subsec:hydroSource} respectively.

A central aspect of each practical \fv{} scheme is an appropriate and informed choice of the space-time control volumes, which, depending on the scheme and problem at hand, might change during the time evolution. 
Given a set of control volumes and a corresponding set of cell averages $\bar{u}_i ( t^l, x_j ) \equiv \bar{u}^l_{i,j}$ the time evolution to ${t^{l+1}\equiv t^l+\Delta t}$ requires the solution of the Riemann problems at each cell interface.
A Riemann problem in this \cfd{} context is the \ivp{} related to the time evolution of two initially spatially constant states left and right of an initial interface, see, \eg{}, \ccite{Toro2009,Lax1973,Ames:1992,LeVeque:1992,LeVeque:2002,Hesthaven2007,RezzollaZanotti:2013,Vazquez-Cendon2015} for details.
Part of these problems are the fluxes through the cell boundaries.
The computation of those fluxes requires a reconstruction of the values of $u_i$ on the cell interfaces located at $x_{j + \ttfrac{1}{2}}$, which we denote as $u_{i,j + \ttfrac{1}{2}}^l$, from the given set of cell averages $\bar{u}_{i,j}^l$.
This is usually done by means of a carefully constructed polynomial approximation respecting the given cell averages of the neighboring cells.
The order of the chosen approximation is one of the factors contributing to the overall spatial order (of the error) of the scheme at hand.

Given the cell averages $\bar{u}_{i,j}^l$ and fluxes through the cell interfaces at $t=t^l$ it remains to solve the Riemann problems at each cell interface.
The solution of the Riemann problem amounts to the exact evaluation of the flux integrals on the \rhs{} of \cref{eq:FVintEq}.
Depending on the complexity of the underlying conservation equation an exact solution of the Riemann problems at the cell boundaries might be either impossible or simply not feasible.
Most explicit \fv{} schemes, especially those for general advection-diffusion equations, either use approximate Riemann solvers (\eg{}, the Roe~\cite{ROE1981357} or the HLLE~\cite{HLLE1,HLLE2} solver) or do not require Riemann solvers at all (\eg{}, the KT~\cite{KTO2-0} scheme).
For a pedagogic introduction into the broad field of \fv{} methods and \hrsc{} schemes in general we refer the interested reader to \ccite{Ames:1992,LeVeque:1992,LeVeque:2002,Hesthaven2007,RezzollaZanotti:2013,Vazquez-Cendon2015} and references therein.

In the following \cref{subsec:hydroKT} we will introduce the particular \fv{} scheme, which we have chosen for its flexibility, efficiency, and relative simplicity.

\subsection{The \texorpdfstring{\acrshort{kt}/\acrshort{knp}}{KT/KNP} scheme and the \texorpdfstring{\acrshort{muscl}}{MUSCL} reconstruction}\label{subsec:hydroKT}
\begin{disclaimer}
	This subsection follows the discussion in Sec.~IV.C of \nbccite{Koenigstein:2021syz} and App.~E.1 of \nbccite{Steil:2021cbu} with slight modifications due to the fact that we want to discuss a system of conservation laws for $\{u_i ( t, x )\}$ instead of a single conservation law for $u ( t, x )$. 
\end{disclaimer}
In this subsection we will summarize the central scheme presented in \ccite{KTO2-0} by A.~Kurganov and E.~Tadmor, which we will refer to in the following as \acrrepeat{kt} scheme, and a variant of it introduced in \ccite{KTO2-1} by A.~Kurganov, S. Noelle, and G. Petrova, which we will refer to in the following as \knpScheme{}.
The \kt{} and \knpScheme{} differ only in their implementation of the advection flux and we will use the term \knpScheme{} in the following only when discussing specifics of this variant.
The \kt{} scheme can be implemented and applied as a black-box solver for systems of the type of \cref{eq:FVadsEq}.
Apart from the \pdes{} with their initial and boundary conditions the only additional information about the system required for its solution using the \ktScheme{} are selected eigenvalues of the Jacobian of the advection term, see \cref{eq:FVajp12} and the related discussion.
The scheme does not require a Riemann solver of any kind and as such does not rely on a characteristic decomposition of the advection flux.

The \kt{} scheme provides a direct method for evaluating the flux integrals on the \rhs{} of \cref{eq:FVintEq}.
The main focus lies on the treatment and implementation of the flux integrals for the advection flux $F_i[\ldots]$ \dash{} specific applications will be discussed in \cref{subsec:hydroAdvection}.
A careful treatment of the advection flux $F_i[\ldots]$ is imperative when dealing with (non-linear) advection terms.
The diffusion and source terms are treated separately and will be discussed at the end of this subsection with specific applications in \cref{subsec:hydroDiffusion,subsec:hydroSource}.

The \ktScheme{} admits a meaningful ${t^{l+1}-t^l\equiv\Delta t\rightarrow 0}$ limit in the context of \cref{eq:FVintEq} and is thus an improvement on its predecessor: the \ntScheme{}~\cite{NT}, with which it shares its piecewise-linear \muscl{} reconstruction~\cite{MUSCL}.
We will focus on the \ktScheme{} in its so-called semi-discrete from \dash{} \ie{}, in the limit $\Delta t\rightarrow 0$ \dash{} which involves only an explicit spatial discretization.
The \ktScheme{} is formally second-order accurate in the spatial direction and as such an improved version of the first-order accurate \lxfScheme{}~\cite{LxF1,LxF2}.
A semi-discrete form reduces the \pdes{} \eqref{eq:FVcsEq} or equivalently \eqref{eq:FVintEq} to a set of coupled \odes{}, which can be solved by a large class of general-purpose \ode{} solvers.
This is especially useful when working on stiff problems or \pde{} systems coupled to additional \odes{}.
We will proceed with the introduction of quantities involved in the semi-discrete form \eqref{eq:FVKTO2} of the \ktScheme{}.
The following quantities are especially relevant for the numerical advection flux~\eqref{eq:definition_h_kt_scheme}.\bigskip

Consider a set of volume averages $\bar{u}_{i,j}^l$ at $t^l$ based on an equidistant\footnote{%
	The generalization of the \ktScheme{} to non-uniform grids is on a conceptual level straightforward and especially useful for higher-dimensional extensions and for adaptive or moving mesh variants, see, \eg{}, \ccite{KTmovingMesh}.
	Such generalizations require a more involved implementation and are not needed in this work. 
	However, in the context of \frg{} flow equations this might be relevant for models with multiple condensate directions, see, \eg{}, \ccite{Strodthoff:2011tz,Mitter:2013fxa,Rennecke:2016tkm,Lakaschus:2020caq,Fukushima:2010ji,Fejos:2020lli}.
} grid of volume cells ${\mathcal{V}_j \equiv [ x_{j - \ttfrac{1}{2}}, x_{j + \ttfrac{1}{2}} ]}$, with $\forall j, \ x_{j + \ttfrac{1}{2}} - x_{j - \ttfrac{1}{2}}=\Delta x$.
At the initial time $t_0$ an \ic{} \dash{} \ie{}, the corresponding set of volume averages $\bar{u}_{i,j}^0$ \dash{} has to be provided to initialize the flow. 
If the \ic{} is provided in functional form $u_i(t_0,x)$, the corresponding cell averages $\bar{u}_{i,j}^0$ should be computed according to \cref{eq:FVaverage}.
Approximating the averages at $t_0$, \eg{}, using the midpoint values $\bar{u}_{i,j}^0\approx u_i(t_0,x_j)$, can introduce significant errors \dash{} especially when the volume grid is coarse ($\Delta x$ is large) or when the \ic{} $u_i(t_0,x)$ contains significant discontinuities. 
To prevent such errors we use the proper cell averages according to \cref{eq:FVaverage} either by integrating/averaging the \ic{} $u_i(t_0,x)$ symbolically or numerically.
The implementation of \bcs{} in the \ktScheme{} will be discussed at the end of this section after the introduction of some useful nomenclature.

The time evolution of the averages $\bar{u}_{i,j}^l$ at $t^l$ to averages at $\bar{u}_{i,j}^{l+1}$ at $t^{l+1}=t^l+\Delta t$ on the same volume grid is a three-step process:
\begin{enumerate}[label=\textbf{\arabic*}.]
	\item \textbf{The Reconstruction} (piecewise-linear \muscl{}) is computed from the cell averages:
	\begin{align}
		 \tilde{u}_i ( t^l, x ) &= \, \sum_{j=0}^{n-1} \big\{ \bar{u}_{i,j}^l + (\partial_x u )_{i,j}^l \, ( x - x_j ) \big\} \Id_{[ x_{j - \frac{1}{2}}, x_{j + \frac{1}{2}} ]} \, ,	\label{eq:FVmuscl}
	\end{align}
	where the sum runs over all $n$ volume cells and with the projection operator $ \Id_{[ x_{j - \ttfrac{1}{2}}, x_{j + \ttfrac{1}{2}} ]} $, which is one if $x_{j - \ttfrac{1}{2}}\le x \le x_{j + \ttfrac{1}{2}}$ and zero otherwise.
	The reconstruction step is needed to gain access to the function values $\tilde{u}_i ( t^l, x )$ and it uses approximations to the exact derivatives $( \partial_x u )_{i,j}^l$ by employing a scalar \acrrepeatEmph{tvni}\footnote{%
		In literature \tvd{} is often used as a less precise synonym for \acrrepeat{tvni}, \cf{} Sec.~9.2.2 of \ccite{RezzollaZanotti:2013}.
		Throughout this work we will adopt the more precise term \tvni{}.
	} reconstruction~\cite{LeVeque:1992,LeVeque:2002,HARTEN1983357},
	\begin{align}
		( \partial_x u )_{i,j}^l = \frac{\bar{u}_{i,j+1}^l - \bar{u}_{i,j}^l}{\Delta x} \, \phi \bigg( \frac{\bar{u}_{i,j}^l - \bar{u}_{i,j-1}^l}{\bar{u}_{i,j+1}^l - \bar{u}_{i,j}^l} \bigg) \, ,	\label{eq:FVuxjn}
	\end{align}
	with a \tvni{} limiter $\phi(r)$. An overview of \tvni{} flux limiters can be found, \eg{}, on the web page~\cite{wikiFluxLimiter}, in \ccite{LeVeque:1992,LeVeque:2002}, or in Sec.~9.3.1 of \ccite{RezzollaZanotti:2013}.
	Here, we follow \nbccite{KTO2-0} and use the so-called \textit{minmod} limiter~\cite{MinModRoe}\footnote{%
		We also implemented and tested other flux limiters, which however did not influence our numerical results in a significant manner \dash{} thus we restrict our discussions to results obtained with the \textit{minmod} limiter \eqref{eq:FVminmod}.
		A problem specific optimization of the choice of flux limiters with regard to numerical performance could be part of future work.%
	},
	\begin{align}
		\phi ( r ) = \, & \max[ 0, \min( 1, r )] \, .	\label{eq:FVminmod}
	\end{align}
	The limiter $\phi$ is used in \cref{eq:FVuxjn} to limit the slopes during the reconstruction process.
	This is crucial to prevent spurious oscillations around discontinuities, \eg{}, shocks, in high-resolution schemes like the \ktScheme{}.
	The \ktScheme{} can also be used with higher-order reconstruction schemes\footnote{%
		Examples for such improvements are the use of the third-order central weighted essentially non-oscillatory (C-WENO) reconstruction~\cite{WENO,WENO-C,WENO-C2} outlined in \ccite{KTO3-0}, the fifth-order WENO scheme (WENO5)~\cite{WENO2,WENO5} employed in \ccite{KTO5-0}, or the fifth-order monotonicity-preserving (MP5) reconstruction~\cite{MP5} used in \ccite{KT-MP5}.
	} to increase the spatial accuracy of the scheme.
	
	When using a piecewise-constant or -linear reconstruction the cell averages $\bar{u}_{i,j}^l$ coincide with the midpoint values $u_{i,j}^l$.
	While we employ a piecewise-linear reconstruction, we still maintain the distinction between averages and midpoint values for the sake of clarity.
	
	\item	\textbf{The time step} from $t^{n}$ to $t^{l+1}$ is performed by computing the flux integrals on the \rhs{} of \cref{eq:FVintEq} using the reconstruction $\tilde{u}_i ( t^l, x )$ from \cref{eq:FVmuscl} and carefully chosen control volumes discussed below. In the limit ${t^{l+1}-t^l\equiv\Delta t\rightarrow 0}$ only the expressions for $a_{j + \ttfrac{1}{2}}^{l,-}$, $a_{j + \ttfrac{1}{2}}^{l,+}$, $u_{i,j + \ttfrac{1}{2}}^{l,-}$, and $u_{i,j + \ttfrac{1}{2}}^{l,+}$ from \cref{eq:FVampjp12,eq:FVumpjp12} respectively are relevant for the semi-discrete \ktScheme{}. The other quantities discussed for this second step of the \ktScheme{} are however necessary to understand the underlying algorithm.
	
	At each cell interface $x_{j + \ttfrac{1}{2}}$ the respective left- and right-sided local speed of propagation $a_{j + \ttfrac{1}{2}}^{l, \mp}$ is estimated in the \knpScheme{} using
	\begin{subequations}\label{eq:FVampjp12}
	\begin{align}
		a_{j + \frac{1}{2}}^{l,-} &\equiv\,\max\bigg\{\lambda_m\Big[\Big\{ u_{i,j + \frac{1}{2}}^{l,-}\Big\} \Big],\lambda_m\Big[\Big\{ u_{i,j + \frac{1}{2}}^{l,+}\Big\} \Big], 0\bigg\}\label{eq:FVamjp12} \\[0.1em]
		a_{j + \frac{1}{2}}^{l,+} &\equiv\,\min\bigg\{\lambda_1\Big[\Big\{ u_{i,j + \frac{1}{2}}^{l,-}\Big\} \Big],\lambda_1\Big[\Big\{ u_{i,j + \frac{1}{2}}^{l,+}\Big\} \Big], 0\bigg\}\label{eq:FVapjp12}
	\end{align}
	\end{subequations}
	with the left- and right-sided intermediate values $u_{i,j + \ttfrac{1}{2}}^{l, \mp}$ of $\tilde{u}_i ( t^l, x )$ at the cell interface $x_{j + \ttfrac{1}{2}}$:
	\begin{subequations}\label{eq:FVumpjp12}
	\begin{align}
		u_{i,j + \frac{1}{2}}^{l,-} = \, & \bar{u}_{i,j}^l + \tfrac{\Delta x}{2} \, ( \partial_x u )_{i,j}^l \, ,	\label{eq:FVumjp12}\\[0.1em]
		u_{i,j + \frac{1}{2}}^{l,+} = \, & \bar{u}_{i,j+1}^l -\tfrac{\Delta x}{2}  \, ( \partial_x u )_{i,j+1}^l \, . \label{eq:FVupjp12}
	\end{align}
	\end{subequations}
	The original \kt{} variant uses a simplified/balanced estimate for left- and right-sided local speed of propagation:
	\begin{subequations}\label{eq:FVajp12}
	\begin{align}
		a_{j + \frac{1}{2}}^{l} &\equiv\, -a_{j + \frac{1}{2}}^{l, -,\mathrm{KT}}\equiv\, +a_{j + \frac{1}{2}}^{l, +,\mathrm{KT}}\,\equiv \label{eq:FVajpmKT}\\
		&\equiv\, \max \bigg\{ \rho\bigg( \frac{\partial F}{\partial u} \Big[\Big\{ u_{i,j + \frac{1}{2}}^{l,+}\Big\} \Big] \bigg) , \rho\bigg( \frac{\partial F}{\partial u} \Big[\Big\{ u_{i,j + \frac{1}{2}}^{l,-}\Big\} \Big] \bigg) \bigg\}\, ,
	\end{align}
	\end{subequations}
	with the spectral radius $\rho ( M ) \equiv \max_i | \lambda_i ( M ) |$.
	\Cref{eq:FVampjp12,eq:FVajp12} include information from the eigenvalue spectrum $\lambda_1 < \ldots < \lambda_m$ of the Jacobian $\frac{\partial F}{\partial u}$.
	The \kt{} and \knpScheme{} are limited to systems with \textit{strictly hyperbolic} advection fluxes signaled by a non-degenerate eigenvalue spectrum $\lambda_1 < \ldots < \lambda_m$ of the Jacobian~$ \frac{\partial F}{\partial u} $ for all $x$, $t$, and $u$~\cite{KTO2-0,KTO2-1}\footnote{%
		A further improvement in terms of estimates of local speeds of propagation engineered for non-convex hyperbolic (systems of) conservation laws is presented in \ccite{KTO5-0} using further information about the eigensystem of the Jacobian $\frac{\partial F}{\partial u}$. 
		When an explicit evaluation of the Jacobian is impossible or unfeasible numerical approximations can be employed~\cite{KTO2-0,LiuTadmore2000,Jiang97non-oscillatorycentral}.
		Throughout this work, however, we employ the exact or symbolic expressions for the Jacobian.%
	}.
	For the numerical applications in this thesis, the simple balanced estimate \eqref{eq:FVajp12} \dash{} the \ktScheme{} \dash{} has proven to be sufficient for most computations.
	The slightly more involved \knpScheme{} with its more refined estimates \eqref{eq:FVamjp12} and \eqref{eq:FVapjp12} is primarily used in \cref{paragraph:knpLargeN}.
	For single-valued conserved quantities $u \equiv \{ u_1\}$ the expressions \eqref{eq:FVampjp12} and \eqref{eq:FVajp12} for local speed of propagation simplify significantly.
	Details regarding advection phenomena and (hyperbolicity) constraints on advection fluxes are discussed in \cref{subsec:hydroAdvection}.

	Using the estimated local speed of propagation, a space-time control volume
	\begin{align}
		[x_{j + \frac{1}{2}, \mathrm{L}}^l , x_{j + \frac{1}{2}, \mathrm{R}}^l ] \times [ t^l, t^l + \Delta t ]
	\end{align}
	around each cell interface $x_{j + \ttfrac{1}{2}}$ is chosen.
	The spatial extent corresponds to the domain which is causally affected by information propagating with the local velocities away from the cell interface at $x_{j + \ttfrac{1}{2}}$.
	The flux integrals of \cref{eq:FVintEq} are performed on these space-time control volumes separately using the midpoint rule to approximate the flux integrals and leading to averages $\bar{\omega}_{i,j}^{l+1}$ and $\bar{\omega}_{i,j+\ttfrac{1}{2}}^{l+1}$ based on the new intermediate spatial grid spanned by the points
		\begin{subequations}
		\begin{align}
			x_{j + \frac{1}{2}, \mathrm{L}}^l = \, & x_{j + \frac{1}{2}} + a_{j + \frac{1}{2}}^{l,-}\Delta t \, ,\\[.1em]
			x_{j + \frac{1}{2}, \mathrm{R}}^l = \, & x_{j + \frac{1}{2}} + a_{j + \frac{1}{2}}^{l,+}\Delta t \, .
		\end{align}
		\end{subequations}
	In the regions $[ x_{j - \ttfrac{1}{2}, \mathrm{R}}^l, x_{j + \ttfrac{1}{2}, \mathrm{L}}^l ]$ the solutions underlying the computed averages $\bar{\omega}_{i,j}^{l+1}$ are smooth.
	The solutions underlying the computed averages $\bar{\omega}_{j+\ttfrac{1}{2}}^{l+1}$ based on the regions $[ x_{j + \ttfrac{1}{2}, \mathrm{L}}^l, x_{j + \ttfrac{1}{2}, \mathrm{R}}^l ]$ are non-smooth.
	Details of this step can be found in \ccite{KTO2-0,KTO2-1}.
	
	\item	\textbf{The projection:} A \muscl{}-type piecewise-linear reconstruction based on $\bar{\omega}_{i,j + \ttfrac{1}{2}}^{l+1}$ and $\bar{\omega}_{i,j}^{l+1}$ is used to project these averages back onto the original uniform grid spanned by $x_{j + \ttfrac{1}{2}}$.
	This results in a fully discrete second-order central scheme, see Eq.~(3.9) of \ccite{KTO2-0} and Eq. (3.7) of \ccite{KTO2-1}, which gives an algebraic expression for $\bar{u}_{i,j}^{l+1}$ in terms of the averages
		\begin{align}
			\big\{ \{\bar{u}_{i,j - 2}^l\}, \{\bar{u}_{i,j - 1}^l\}, \{\bar{u}_{i,j}^l\}, \{\bar{u}_{i,j + 1}^l\}, \{u_{i,j + 2}^l\} \big\}	\label{eq:kt_stencil}
		\end{align}
	and $\{ a_{j \pm \frac{1}{2}}^{l,\pm} \}$.
	A pictographic representation of the multi-step evolution procedure with the involved quantities and grids can be found in Fig.~3.2 of \ccite{KTO2-0} and the corresponding Fig.~3.1 of \ccite{KTO2-1}.
	The numerical viscosity of this second-order scheme is $\order ( \Delta x^3 )$ and does not depend on $\Delta t$, which represents the mentioned an improvement when compared to the $\Delta t$-dependent numerical viscosities $\order ( \Delta x^2/\Delta t )$ and $\order ( \Delta x^4/\Delta t )$ of the \lxf{} and \nt{} schemes, respectively~\cite{KTO2-0,KTO2-1}.
\end{enumerate}

The $\Delta t$-independent numerical viscosity allows for a controlled limit $\Delta t \rightarrow 0$, in \cref{eq:FVintEq}, resulting in a reduction to a practical semi-discrete scheme in conservative form~\cite{KTO2-0}, which can be implemented straightforwardly:
\begin{align}
	\partial_t \bar{u}_{i,j} = - \frac{H_{i,j + \frac{1}{2}} - H_{i,j - \frac{1}{2}}}{\Delta x}\, + \skeleton{2} \, ,	\label{eq:FVKTO2H}
\end{align}
where $\skeleton{2}$ denotes the sum of the diffusion and source fluxes.
The numerical advection flux $H_{i,j + \ttfrac{1}{2}}$ is given by
	\begin{align}
		H_{i,j + \frac{1}{2}}^\mathrm{KT} \equiv \, & \frac{F_i \Big[ t, x_{j + \frac{1}{2}}, \big\{u_{i,j + \frac{1}{2}}^+\big\} \Big] + F_i \Big[ t, x_{j + \frac{1}{2}}, \big\{u_{i,j + \frac{1}{2}}^-\big\} \Big]}{2} - a_{j + \frac{1}{2}} \, \frac{u_{i,j + \frac{1}{2}}^{+} - u_{i,j + \frac{1}{2}}^{-}}{2} \, ,	\label{eq:definition_h_kt_scheme}
	\end{align}
for the \kt{} variant and by
\begin{align}
	H_{i,j + \frac{1}{2}}^\mathrm{KNP} \equiv \, & \frac{a_{j + \frac{1}{2}}^+ F_i \Big[ t, x_{j + \frac{1}{2}}, \big\{u_{i,j + \frac{1}{2}}^-\big\} \Big] - a_{j + \frac{1}{2}}^- F_i \Big[ t, x_{j + \frac{1}{2}}, \big\{u_{i,j + \frac{1}{2}}^+\big\} \Big]}{a_{j + \frac{1}{2}}^+-a_{j + \frac{1}{2}}^-}\,+ \notag\\* % no page break here
	&\qquad\qquad\qquad\qquad\qquad\qquad\qquad+ \frac{a_{j + \frac{1}{2}}^+a_{j + \frac{1}{2}}^+}{a_{j + \frac{1}{2}}^+-a_{j + \frac{1}{2}}^-} \,\del{u_{i,j + \frac{1}{2}}^{+} - u_{i,j + \frac{1}{2}}^{-}} \, ,	\label{eq:definition_h_knp_scheme}
\end{align}
for the \knp{} variant.
Note that \cref{eq:definition_h_kt_scheme} presents as a simplification of \cref{eq:definition_h_knp_scheme} when using the balanced approximation \eqref{eq:FVajpmKT} for the left- and right-sided local speed of propagation.
This semi-discrete scheme is formally second-order accurate in $\Delta x$ and can be used in conjunction with various \ode{} time-step algorithms.
In this work, we use \WAM{}'s \textit{NDSolve}~\cite{Mathematica:12.1,Mathematica:13.0} and \textit{solve\_ivp} with the \textit{LSODA} option using an Adams/BDF method with automatic stiffness detection and switching from the \textit{SciPy~1.0} library~\cite{2020SciPy-NMeth}, \cf{} \cref{chap:zeroONSU2,chap:GN}.
Time-stepping has not been a focus of our work and we refer the interested reader to the excellent \ccite{Ihssen:2023qaq} discussing the issue in the context of \frg{} in detail.

The \kt{}/\knpScheme{} for a position-independent advection flux is conservative, meaning that detailed balance at the cell interfaces is maintained.
It is also \tvni{}~\cite{HARTEN1983357,LeVeque:1992,LeVeque:2002} when used with appropriate flux limiters like the minmod limiter \eqref{eq:FVminmod}.
The \tv{}~\cite{HARTEN1983357} \dash{} which is simply the arc length \dash{} of the solution $u ( t, x )$ is given by
\begin{align}
	\mathrm{TV}_i [ \partial_x u_i ( t, x ) ] \equiv \int_{x_{-\frac{1}{2}}}^{x_{n-\frac{1}{2}}} \dif x \, | \partial_x u_i ( t, x ) | \, ,	\label{eq:TVcontinuous}
\end{align}
on the (computational) interval $[ x_{-\ttfrac{1}{2}},x_{n-\ttfrac{1}{2}} ]$.
On a \fv{} grid, a typical discretized version of \cref{eq:TVcontinuous} is given by, \cf{}\ \ccite{HARTEN1983357,LeVeque:1992,LeVeque:2002,RezzollaZanotti:2013},
\begin{align}
	\mathrm{TV}_i [ \{ \bar{u}_{i,j} ( t ) \} ] \equiv \sum_{j = 0}^{n-1} | \bar{u}_{i,j+1} ( t ) - \bar{u}_{i,j} ( t ) | \, ,	\label{eq:TVdiscrete}
\end{align}
where a first-order forward stencil is used to discretize the first derivative\footnote{
	Please note that $\{ \bar{u}_{i,j} ( t ) \}$ in the definition \cref{eq:TVdiscrete} of $\mathrm{TV}_i$ refers to the set of all the \ith{i} component cell averages and not the set of all components in the \ith{j} volume cell.
}. 
(Weak) solutions of broad classes of hyperbolic and parabolic conservation laws \dash{} without source terms \dash{} are \tvni{} during time evolution when considered on a finite interval, see, \eg{}, \ccite{HARTEN1983357,LeVeque:1992,Toro2009} and especially \ccite{Redheffer1974Mar}: meaning their arc length only decreases.
The differences $\mathrm{TV} [ \{ \bar{u}_i ( t^{m } ) \} ] - \mathrm{TV} [ \{ \bar{u}_i ( t^{m+1} ) \} ]$ on a discrete trajectory $\bar{u}_i ( t )$ of an admissible solution at different times separated by one time step $\Delta t$, where $t^{m+1} = t^m + \Delta t$, is greater or equal to zero for all $t^m$, \ie{}, \tv{} is non-increasing.
This \tvni{} property of discrete weak solutions is an important guiding principle in the construction of numerical schemes in \cfd{} meant to resolve shocks and discontinuities, since \tvni{} schemes do not produce spurious oscillations around discontinuities.
Such spurious oscillations would violate the TVNI property since they would increase arc-length.
The \tv{} will be important for our discussion of numerical entropy and irreversibility of \rg{}-flows in \cref{subsec:0dO1Entropy} and we will also comment on it in \cref{subsec:hydroAdvection} in the context of backward time integration.
	
\paragraph{Diffusion and source/sink terms}\phantomsection\label{paragraph:KTQS}\mbox{} \\
So far we only considered the advection term $\difx\! F[\ldots]$ in the discussion of the \ktScheme{}.
We will now turn our attention to diffusion fluxes $\difx\! Q[\ldots]$ completing our discussion of convective contributions.
When considering a non-linear diffusion flux $\difx\! Q_i [ t, x, \{u_i ( t, x )\}, \{\partial_x u_i ( t, x )\} ]$ \cref{eq:FVadsEq} can manifest as a strongly degenerate parabolic equation system admitting potentially non-smooth solutions.
In the \ktScheme{} the hyperbolic and parabolic parts of the \pde{} \eqref{eq:FVadsEq} are treated simultaneously based on the strict splitting between $F$ and $Q$.
Kurganov and Tadmor~\cite{KTO2-0} presented a discretization of the diffusion flux based on a central-difference approximation,
\begin{align}
	 P_{i,j + \frac{1}{2}} = \, & \tfrac{1}{2}\, Q_i \Big[ t, x_j, \big\{\bar{u}_{i,j}\big\},  \big\{\tfrac{\bar{u}_{i,j + 1} - \bar{u}_{i,j}}{\Delta x} \big\}\Big] + \tfrac{1}{2}\, Q_i \Big[ t, x_{j + 1}, \big\{\bar{u}_{i,j+1}\big\}, \big\{ \tfrac{\bar{u}_{i,j + 1} - \bar{u}_{i,j}}{\Delta x} \big\}\Big] \, .	\label{eq:kt_original_diffusion}
\end{align}
An alternative second-order discretization, like the one put forward in App.~B of \nbccite{Chertock2005}, can also be successfully employed with the \ktScheme{}, \cf{} Eq. (124) and (125) of \ccite{Koenigstein:2021syz} and the corresponding discussion.
We will limit our discussion to results obtained with the diffusion flux \eqref{eq:kt_original_diffusion}.
Improved \kt-type schemes employing higher-order reconstructions (like, \eg{}, C-WENO/WENO5 or MP5~\cite{KTO3-0,KTO5-0,KT-MP5}) also use higher-order discretizations for the diffusion flux like the fourth-order one put forward in Eqs.~(4.9) and (4.10) of \ccite{KTO3-0}.\bigskip

The inclusion of source-/sink-terms in the semi-discrete \ktScheme{} is rather simple, but again relying on a strict separation from the convective terms, \cf{} Example 9 of \ccite{KTO2-0}.
In the following we use $S_{i,j}$ as the source/sink contribution to the flow $\partial_t \bar{u}_{i,j}$ of the \ith{i} component in the \ith{j} volume cell.
The implementation of $S_{i,j}$ is rather problem specific depending on the explicit structure of the source term $S_i [ t, x, \{u_i ( t, x )\} ]$.
For simple $u$-independent source terms $S_i [ t, x]$ symbolically evaluating \cref{eq:FVsIntEq} can be a good choice.
For more complicated $u$-dependent source terms an approximation based on the midpoint value $S_{i,j}=S_i [ t, x_j, \{\bar{u}_{i,j}\} ]$ can be beneficial.
We reserve further discussion of source terms for our explicit application involving sources/sinks in \cref{subsec:hydroSource,subsec:hydroEuler} and especially in \cref{paragraph:chemical_potential_shock_wave}.

\paragraph{Semi-discrete form and boundary conditions}\phantomsection\label{paragraph:KTBC}\mbox{} \\
To summarize: in full semi-discrete \ktScheme{} the time evolution equation for the \ith{i} component of the \ith{j} cell average is given by
\begin{align}
	\partial_t \bar{u}_{i,j} = \, & - \frac{H_{i,j + \frac{1}{2}} - H_{i,j - \frac{1}{2}}}{\Delta x} + \frac{P_{i,j + \frac{1}{2}} - P_{i,j - \frac{1}{2}}}{\Delta x} 
	+ S_{i,j}\, , \label{eq:FVKTO2}
\end{align}
which includes advection, diffusion, and source fluxes.
\Cref{eq:FVKTO2} is second-order accurate in $\Delta x$ and presents as a \fv{} method-of-lines discretization of the original \pde{} system \eqref{eq:FVadsEq}.
The initial cell averages $\bar{u}_{i,j}^0$ provide the \ic{} for the \ode{} system \eqref{eq:FVKTO2}.

The adept reader might immediately point out that this system is underdetermined: at the \ith{j} cell the convection flux depends on the five-point stencil \eqref{eq:kt_stencil}, which in the first cell \nolinebreak[2]${(j=0)}$ includes $\{\bar{u}_{i,- 2}\}$ and $\{\bar{u}_{i,-1}\}$, while in the last cell $j=n-1$ it involves $\{\bar{u}_{i,n}\}$ and $\{\bar{u}_{i,n+1}\}$.
Those so-called \textit{ghost cells} formally lie outside the computational spatial domain $\mathcal{V}: x_{-\ttfrac{1}{2}}\le x\le x_{n-\ttfrac{1}{2}}$ and are centered around $x_{-2}$, $x_{-1}$, $x_{n}$, and $x_{n+1}$, respectively.
Specific spatial \bcs{} $( \partial_x ) u_i ( t, x ) |_{x \in \partial \mathcal{V}}$ can be implemented in the \ktScheme{} by an appropriate choice for the volume averages of those ghost cells. 
The implementation of \bcs{} using ghost cells is not unique to the \ktScheme{} and in fact quite common in \fv{} methods, see, \eg{}, \ccite{LeVeque:1992,LeVeque:2002,Vazquez-Cendon2015} for a detailed discussion.
Using ghost cells is a very flexible and programmatically simple way to practically implement \bcs{}.

Specific \bcs{} for \frg{} flow equations are discussed at length in \cref{subsec:boundary_conditions_finite_volume} in the main part of this thesis.
Additionally we will discuss \bcs{} for canonical examples in \cref{subsec:hydroAdvection,subsec:hydroDiffusion,subsec:hydroEuler} to conceptualize and facilitate the discussion of \bcs{} for the \frg{} flow equations.

\paragraph{Total variation and explicit position- and time-dependent fluxes}\phantomsection\label{paragraph:KTxdep}\mbox{} \\
At this point we have to remark that the original \kt{} numerical scheme presented in \ccite{KTO2-0} was constructed for position- and time-independent convection fluxes.
Since we employ the \ktScheme{} in its semi-discrete form a resolution of potentially highly complicated and non-linear dynamics in $t$ is possible and ultimately outsourced to the \ode{} solver.
Thus an explicit $t$-dependence in $H_{i,j + \ttfrac{1}{2}}$ and $P_{i,j + \ttfrac{1}{2}}$ is expected to be unproblematic when using \cref{eq:FVKTO2}.

Explicitly position-dependent advection and diffusion terms on the other hand are more worrisome, since they spoil the proper split between advective, diffusive, and source contributions.
By performing the total derivatives in \eqref{eq:FVadsEq} to study the equation in its primitive form, \cf{} \cref{subsec:hydroAdvection,subsec:hydroDiffusion}, we note that explicit $x$-dependencies in $F_i[\ldots]$ and $Q_i[\ldots]$ manifest as internal source-/sink-like contributions, see explicitly \cref{eq:FVFprimitive,eq:FVQprimitive}.
Those internal and also explicit source terms spoil the \tvni{} property, which will be especially relevant for \cref{paragraph:cTheoremzerodO1}.
Defining or constructing an explicit numerical entropy functional for general non-linear conservation laws is a difficult task, especially when source terms are involved, see, \eg{}, \ccite{Monthe:2001,Beneito2008,Chen2011May,Bessemoulin:2012} and references therein. 
Similarly explicit $u$-dependencies $Q_i[\ldots]$ lead to advective contributions in primitive form which are not treated on the same level as the ones stemming from $F_i[\ldots]$, \cf{} again \cref{subsec:hydroDiffusion} and \cref{eq:FVQprimitive}. 

In the scope of this work we could not trace any practical problems back to the explicit position- and time-dependence of the advection and diffusion fluxes.
The comparisons in \cref{chap:zeroONSU2} between results obtained from a direct computation of correlation functions using the generating functional and the results computed using \frg{} flow equations via the \ktScheme{} (with $t$ and $x$-dependent fluxes) can be seen as hard tests for both \dash{} the \frg{} methodology as well as the (slightly modified) \ktScheme{} \dash{} depending on the respective perspective. 
In total, the precision of our results for the non-trivial test cases gives us some confidence, that our approach is generically justified and the \ktScheme{} is suitable for our application throughout the main part of this thesis.

\paragraph{First-order reduction}\phantomsection\label{paragraph:KTO1}\mbox{} \\
An in $\Delta x$ first-order accurate reduction of the semi-discrete \kt{}/\knpScheme{} of \cref{eq:FVKTO2} can be obtained by switching from the piece-wise linear \muscl{} reconstruction \eqref{eq:FVmuscl} to a piecewise constant reconstruction with $(\partial_x u )_{i,j}^l=0$ in the numerical advection fluxes $H_{i,j + \ttfrac{1}{2}}$. 
For explicit expressions we refer to Remark 3 in Sec. 3.1 of \ccite{KTO2-1}, Eq. (4.8) of \nbccite{KTO2-0}, and our explicit implementation in \ccite{Steil:2023PhDFVNB}.
To have a consistent order in $\Delta x$ for the convective contributions, the numerical diffusion flux $P_{i,j + \ttfrac{1}{2}}$ has to be changed to a first-order accurate one, \ie{}, for our purposes to the first-order upwind flux
\begin{align}
	 P_{i,j + \frac{1}{2}}^{\ordern{1}} = \, & Q_i \Big[ t, x_j, \big\{\bar{u}_{i,j}\big\},  \big\{\tfrac{\bar{u}_{i,j + 1} - \bar{u}_{i,j}}{\Delta x} \big\}\Big] \, .	\label{eq:kt_original_diffusion_O1}
\end{align}

\paragraph{Implementation}\phantomsection\label{paragraph:KTimp}\mbox{} \\
The semi-discrete, method-of-lines \fv{} discretization \eqref{eq:FVKTO2} of the \ktScheme{} and the \knp{} variant can be implemented in only a few lines of code in most modern programming languages using list or vector based operations.
Its relative simplicity however should not unsettle the uninitiated: the scheme is almost shockingly powerful as a a black-box solver even for complicated, non-linear systems \eqref{eq:FVadsEq} when paired with a robust \ode{} time-stepper.
Such time-steppers are available in libraries for most modern programming languages, \ie{}, \textit{NDSolve} for \WAM{}~\cite{Mathematica:12.1}, \textit{solve\_ivp} from \textit{SciPy~1.0} for \Python{}~\cite{2020SciPy-NMeth}, or \textit{SUNDIALS}~\cite{hindmarsh2005sundials,gardner2022sundials} for \Cpp{}.
With the auxiliary file~\cite{Steil:2023PhDFVNB} we provide the \WAM{} notebook \textit{``Computational fluid dynamic''} which includes my latest \WAM{}-implementation of the \acrshort{kt}/\acrshort{knp} scheme. 
It includes a completely modular black box solver using \textit{compiled} functions for performance\footnote{
	The performance of a \WAM{} implementation with proper low-level \textit{compiled} functions is on a par with implementations in \Cpp{} and \Python{} using \textit{SUNDIALS}~\cite{hindmarsh2005sundials,gardner2022sundials} and \textit{solve\_ivp}~\cite{2020SciPy-NMeth} respectively.
}.
The auxiliary file~\cite{Steil:2023PhDFVNB} includes the numerical computations of the following \cref{subsec:hydroAdvection,subsec:hydroDiffusion,subsec:hydroSource,subsec:hydroEuler}.
The numerical results in \cref{chap:zeroONSU2} and selected results of \cref{chap:GN} were obtained with older versions of this code~\cite{Steil:2023zeroD,Steil:2023zeroDN1,Steil:2023zeroDlargeN}.

\subsection{Advection and shocks}\label{subsec:hydroAdvection}
In this subsection we will discuss the advective contributions to \cref{eq:FVadsEq}, \viz{} the ones governed by the advection term $\difx\! F_i [ t, x, \{u_i ( t, x )\} ]$.
Focusing on the latter, let us consider the non-linear system of advection equations
\begin{align}
	\dift\! u_i ( t, x )+\difx\!  F_i [ t, x, \{u_i ( t, x )\} ]=0\,,\label{eq:FVadsEqFonly}
\end{align}
which can be brought into \textit{primitive form}\footnote{
	In the \fv{} context the following equation should be understood in integral form \eqref{eq:FVintEq} to avoid any conceptual and mathematical problems especially when dealing with discontinuities~\cite{LeVeque:2002,RezzollaZanotti:2013,Vazquez-Cendon2015}.
}
\begin{align}
	\partial_t u_i ( t, x )+\frac{\partial F_i[ t, x, \{u_i ( t, x )\} ]}{\partial u_l(t,x)}\,\partial_x u_l ( t, x ) = -\partial_x F_i [ t, x, \{u_i ( t, x )\} ] \,.\label{eq:FVFprimitive}
\end{align}
\Cref{eq:FVFprimitive} includes the Jacobian $\frac{\partial F}{\partial u}$ \dash{} the matrix of advection speeds \dash{} in the actual advection term $\frac{\partial F_i}{\partial u_l}\partial_x u_l$ and on the \rhs{} an internal source term $-\partial_x F_i[\ldots]$ related to the explicit position dependency of the advection flux $F$.
In this sense position-dependence of the advection flux can be understood in terms of additional/internal source terms.
\Cref{eq:FVFprimitive} or in our context the flux $F$ itself is understood to be \textit{hyperbolic}, if the Jacobian/the matrix of advection speeds $\frac{\partial F}{\partial u}$ is diagonalizable with a set of real eigenvalues $\lambda_1 \le \ldots \le \lambda_m$ for all $t$, $x$, and $u_i ( t, x )$ under consideration.
The application of the \kt{}/\knp{} numerical advection fluxes $H_{i,j + \frac{1}{2}}$ \eqref{eq:definition_h_kt_scheme}/\eqref{eq:definition_h_knp_scheme} requires \textit{strictly hyperbolic} systems with a non-degenerate eigenvalue spectrum $\lambda_1 < \ldots < \lambda_m$.
When studying only one conserved quantity \dash{} not a system of $m$ conserved quantities $\{u_1 ( t, x ),\ldots,u_m ( t, x )\}$ \dash{} its conservation equation is hyperbolic if the advection speed $\frac{\partial F}{\partial u}$ is real and finite for all $t$, $x$, and $u( t, x )$ under consideration.
For hyperbolic systems \ivps{} are well posed~\cite{LeVeque:2002,RezzollaZanotti:2013,Vazquez-Cendon2015}.
Hyperbolic systems typically describe processes where information or disturbance are propagated through space-time in a wave-like manner with a finite advection speed~\cite{LeVeque:1992,LeVeque:2002,Hesthaven2007,Vazquez-Cendon2015,polyanin2016handbook}, \cf{} \cref{sububsec:LAEBBE} for instructive and canonical examples.

For the remainder of this subsection we will limit our discussion to the single conservation law
\begin{align}
	&\partial_t u ( t, x )+\difx\! F[ t,x,u ( t, x )]=0\,,\\[.2em]
	&\partial_t u ( t, x )+\frac{\partial F[ t,x,u ( t, x )]}{\partial u(t,x)}\,\partial_x u ( t, x ) = -\partial_x F[ t,x,u ( t, x )]\, ,\label{eq:FVFsimple}
\end{align}
for $u(t,x)$, which is the relevant scenario for our numerical computations in \cref{chap:zeroONSU2,chap:GN}. 
An application to a canonical system of conservation laws, \ie{}, the Euler equations of ideal fluid dynamic, will be presented in \cref{subsec:hydroEuler}.

\subsubsection{Method of characteristics and Rankine–Hugoniot (jump) condition}\label{subsec:MoC}
\begin{disclaimer}
	This subsubsection is based on the first parts of Apps. C and D of \ccite{zerod3}.
\end{disclaimer}
An important computational tool for \textit{quasilinear} \dash{} \ie{}, linear in the derivatives $\partial_t u$ and $\partial_x u$ but not necessarily linear in $u$ \dash{} hyperbolic \pdes{} of the form \eqref{eq:FVFsimple} is the \textit{method of characteristics}, \cf{} \ccite{Delgado2006Aug} and \ccite{polyanin2016handbook,LeVeque:2002} for a general overview.

The method of characteristics states that a quasilinear hyperbolic \pde{}
\begin{align}
	a ( t, x, u ) \, \partial_t u ( t, x ) + b ( t, x, u ) \, \partial_x u ( t, x ) = c ( t, x, u )	\label{eq:MoC_pde}
\end{align}
presents as a set of \odes{} along so-called characteristic curves, which are given by the Lagrange–Charpit equations~\cite{Delgado2006Aug} (also called characteristic equations):
\begin{subequations}\label{eq:MoC_odes}
\begin{align}
	\frac{\partial t ( \tau )}{\partial \tau} = \, & a ( t ( \tau ), x ( \tau ), u ( \tau ) ) \, ,
	\label{eq:MoC_ode_1}\\
	\frac{\partial x ( \tau )}{\partial \tau} = \, & b ( t ( \tau ), x ( \tau ), u ( \tau ) ) \, ,
	\label{eq:MoC_ode_2}\\
	\frac{\partial u( \tau )}{\partial \tau} = \, & c ( t ( \tau ), x ( \tau ), u ( \tau ) ) \, ,
	\label{eq:MoC_ode_3}
\end{align}
\end{subequations}
with the curve-parameter $\tau$ and \ics{}
\begin{subequations}\label{eq:MoC_ics}
\begin{align}
	t ( \tau = 0 ) = \, & t_0 \, ,
	\label{eq:MoC_ics_1}\\
	x ( \tau = 0 ) = \, & x_0 \, ,
	\label{eq:MoC_ics_2}\\
	u ( \tau = 0 ) = \, & u_0 ( t_0, x_0 ) \, ,
	\label{eq:MoC_ics_3}
\end{align}
\end{subequations}
related to the original \pde{}~\eqref{eq:MoC_pde}. 
Solving this \ode{} system yields the functions $t(\tau)$, $x(\tau)$ and $u(\tau)$, which can be used to extract information about the actual solution of the \pde{}~\eqref{eq:MoC_pde} including, in some cases, the full solution itself.
More details can be found in the textbooks~\cite{Ames:1992,LeVeque:1992,LeVeque:2002,Hesthaven2007,polyanin2016handbook,Vazquez-Cendon2015} and explicit applications follow in \cref{sububsec:LAEBBE,subsec:0dLargeN} as well as in \cref{paragraph:largeNchars,app:method_of_characteristics}. 

Using $t(\tau)$, $x(\tau)$ and $u(\tau)$ one can reconstruct the solution $u(t,x)$ as an implicit solution in terms of $\tau$.
Such a reconstruction however is only valid as long as the characteristic curves $t(\tau)$ and $x(\tau)$ do not intersect.
If they do, the implicit solution $u(t,x)$ becomes multivalued and the \pde{} no longer has a solution in the classical sense~\cite{Vazquez-Cendon2015,LeVeque:2002,RezzollaZanotti:2013}.
The physical/weak-solution develops a \textit{shock} \dash{} a discontinuity in $u(t,x)$ \dash{} which can only be treated properly using a weak formulation, \eg{}, a \fv{} discretization like \cref{eq:FVintEq}.
The speed $\partial_t\xi_s(t)$ of such a shock can be determined with the \textit{Rankine-Hugoniot condition}~\cite{Rankine:1870,Hugoniot:1887}
\begin{align}
	F_\mathrm{R}(t)-F_\mathrm{L}(t)=\partial_t\xi_s(t)(u_\mathrm{R}-u_\mathrm{L})\label{eq:FVRHcondition}\, ,
\end{align}
where the subscripts $\mathrm{R}$ and $\mathrm{L}$ denote the state to the right and left of the shock $\xi_s(t)$, see, \eg{}, \ccite{Ames:1992,LeVeque:2002,Hesthaven2007,Toro2009} for further details.
The characteristic curves on both sides of the shock run into the shock wave with $\lambda(u_\mathrm{L})>\partial_t\xi_s(t)>\lambda(u_\mathrm{R})$.

An instructive example of shock formation is part of \cref{sububsec:LAEBBE} with our discussion of the \acrlong{bbe} \eqref{eq:bbe} and the phenomenon is further discussed in \cref{subsec:hydroEuler,chap:zeroONSU2,chap:GN}.

\subsubsection{Linear and non-linear advection equations}\label{sububsec:LAEBBE}
\begin{disclaimer}
	The part of this subsubsection discussing the \acrlongEmph{bbe} \eqref{eq:bbe} includes material from Sec.~III.A and App.~A of \nbccite{Koenigstein:2021numericalSchemes}.
\end{disclaimer}
Up to this point our discussion of \cfd{} has been extremely technical with the goal to introduce relevant nomenclature and the explicit \fv{} method we use for numerical computation, \viz{} the \ktScheme{}. 
It is high time to ``show some pictures'' \dash{} discuss and present some applications of the framework to explicit problems.
We will do so using two explicit examples: the \acrlongEmph{lae} and the non-linear \acrlongEmph{bbe}.\bigskip

\paragraph{The linear advection equation}\phantomsection\label{paragraph:LAE}\mbox{} \\
We begin this discussion with an archetypical advection equation, \viz{} the \lae{}
\begin{align}
	\partial_t u(t,x) +\partial_x u(t,x)&=0 \label{eq:lae}\, ,\\[.25em]
	u(t,x)&=u(t=0,x-t)\label{eq:laeSol}\, ,\\[.25em]
	\stepcounter{equation}\newSubEqBlock
	u_\mathrm{I}(t=0,x)&=\Theta\!\del[1]{x-\tfrac{1}{8}}-\Theta\!\del[1]{x-\tfrac{3}{8}}\, ,\subEqTag\label{eq:laeICS1}\\
	u_\mathrm{II}(t=0,x)&=\tfrac{1}{2}+\tfrac{1}{2}\sin\del[1]{2\piu x}\, ,\label{eq:laeICS2}\subEqTag
\end{align}
with the linear hyperbolic \pde{} \eqref{eq:lae}, its analytic solution \eqref{eq:laeSol}, and two explicit initial conditions~\eqref{eq:laeICS1} and~\eqref{eq:laeICS2}, which we will use in the following.
With $F=u(t,x)$ in the \laeq{}, the characteristic speed is constant: $\partial F/\partial u=1$.
The \lae{} with an \ic{} very similar to \cref{eq:laeICS1} is also discussed in Sec. 6.1 of \ccite{KTO2-0} as a standard benchmark problem.
The analytic solution \eqref{eq:laeSol} follows directly from the method of characteristics, where the Lagrange–Charpit Eqs.~\eqref{eq:MoC_odes} can be integrated trivially for the \lae{} with $a=b=1$ and $c=0$, \cf{} \cref{subsec:MoC}.
According to \cref{eq:laeSol} the \lae{} simply transports/advects the \ic{} through the computational domain from left to right with a constant speed of $1$ without changing its initial shape.

\fullWidthFigure%
	[t]%
	{hydro/figures/lae_P1.pdf}% Graphics
	[fig:lae_P1flow,fig:lae_P1error]% Sublabels
	{%
	Analytical (solid lines) and numerical solution (dots) to the \laeq{} with \ic{} \eqref{eq:laeICS1} and open \bc{} \eqref{eq:BClinExt} at different times on the left \subref{fig:lae_P1flow} and relative error \eqref{eq:L1error} between the analytical and numerical solution on the right \subref{fig:lae_P1error}.
	The solution $u_{\mathrm{I}}^-(0.0,x)$ has been computed by numerically evolving $\{\bar{u}_{\mathrm{I},j}(t=0.4)\}$ back in time to $t=0$.
	The numerical solution has been computed with the \ktScheme{} using $n=151$ volume cells equidistantly spaced in the interval $x_0=0$ and $x_{150}=1$, \cf{} subsection 2.1.1 of the auxiliary notebook~\cite{Steil:2023PhDFVNB}.
	}%Caption
	{fig:lae_P1}%Label
The analytical and numerical solution of the \laeq{} with the non-analytic \ic{} \eqref{eq:laeICS1} are visualized in \cref{fig:lae_P1flow} with open \bcs{} implemented in the \ktScheme{} by linear extrapolation for the ghost cells~\cite{LeVeque:1992,LeVeque:2002,Vazquez-Cendon2015}, \ie{},
\begin{subequations}\label{eq:BClinExt}
\begin{align}
	\bar{u}_{-2}&=3\bar{u}_0-2\bar{u}_1,\quad\quad\ \ \bar{u}_{-1}=2\bar{u}_0-\bar{u}_1\,,\\
	\bar{u}_{n}&=2\bar{u}_{n-1}-\bar{u}_{n-2},\quad\bar{u}_{n+1}=3\bar{u}_{n-1} -2\bar{u}_{n-2}\,.
\end{align}
\end{subequations}
From a numerical standpoint, discretizing and evolving the \laeq{}, especially when allowing for discontinuous \ic{} like \cref{eq:laeICS1}, is quite challenging.
Naive discretization schemes, like \fds{}, and even more involved global collocation methods are notoriously ill-suited to tackle this problem~\cite{Bernstein1951Jan,LeVeque:1992,LeVeque:2002,Hesthaven2007,Vazquez-Cendon2015,polyanin2016handbook}.
The relative error in $L^1$-norm, \ie{}, the sum of all errors divided by the number of volume cells $n$:
\begin{align}
	\epsilon_{L^1}(t)=\frac{1}{n}\sum_{j=0}^{n-1}|\bar{u}_j(t)-u(t,x_j)|	\label{eq:L1error}
\end{align}
using \cref{eq:laeSol} as reference, is plotted in \cref{fig:lae_P1error}.
After the uncertainty due to the initial discretization of the discontinuous \ic{} \eqref{eq:laeICS1}, the error starts to plateau, allowing for a stable numerical evolution of the discretized initial condition.

The \lae{} is in theory time-reversible: switching the sign of the advection speed one can evolve a solution $u(t,x)$ backwards in time without any conceptual difficulties.
The simple linear advection of the \ic{} prescribed by the \lae{} in an ideal case is free of numerical entropy production.
In practice time-reversal is studied in \cref{fig:lae_P1} by evolving the computed solution $\{\bar{u}_{\mathrm{I},j}(t=0.4)\}$ back in time to $t=0$ by flowing with a reversed advection speed $\partial F/\partial u=-1$ arriving at a numerical solution $\{\bar{u}_{\mathrm{I},j}^-(t=0)\}$ visualized in green  in \cref{fig:lae_P1}.
We notice that the relative $L^1$ error grows slowly as we try to reconstruct the \ic{} at $t=0$.
This is due to inevitable inaccuracies in time integration and spatial discretion errors.
Nevertheless we recover a solution $\{\bar{u}_{\mathrm{I},j}^-(t=0)\}$ very close to the discretized version of the \ic{} $u_\mathrm{I}(t=0,x)$.

\fullWidthFigure%
	[t]%
	{hydro/figures/lae_P2.pdf}% Graphics
	[fig:lae_P2flow,fig:lae_P2error]% Sublabels
	{%
	Analytical (solid lines) and numerical solution (dots) to the \laeq{} with \ic{}~\eqref{eq:laeICS2} and periodic \bc{} \eqref{eq:BCperiodic} at different times on the left \subref{fig:lae_P2flow} and relative error \eqref{eq:L1error} between the analytical and numerical solution on the right \subref{fig:lae_P2error}.
	The numerical solution has been computed with the \ktScheme{} using $n=101$ volume cells equidistantly spaced in the interval $x_0=0$ and $x_{100}=1$, \cf{} subsection 2.1.2 of the auxiliary notebook~\cite{Steil:2023PhDFVNB}.
	}%Caption
	{fig:lae_P2}%Label
The analytical and numerical solution of the \laeq{} with the analytic \ic{}~\eqref{eq:laeICS2} are visualized in \cref{fig:lae_P2flow} with periodic \bcs{} implemented in the \ktScheme{} by mirroring the opposite end of the computational interval in the ghost cells\footnote{%
	Whether or not to enforce $\bar{u}_0=\bar{u}_{n-1}$ depends on the explicit choice of the location of the first and last volume cell $x_0$ and $x_{n-1}$.
	We chose to center both on the respective boundaries of the computational domain and thus enforce $\bar{u}_0=\bar{u}_{n-1}$.
}~\cite{LeVeque:1992,LeVeque:2002,Vazquez-Cendon2015}, \ie{},
\begin{align}
	\bar{u}_{-2}&=\bar{u}_{n-3}\,,\quad \bar{u}_{-1}=\bar{u}_{n-2}\,,\quad \bar{u}_0=\bar{u}_{n-1}\,,\quad
	\bar{u}_{n}=\bar{u}_1\,,\quad \bar{u}_{n+1}=\bar{u}_{2}\,.
	\label{eq:BCperiodic}
\end{align}
Discretizing the smooth \ic{} \eqref{eq:laeICS2} even with slightly less volume cells ($n=101$ instead of $n=150$ in \cref{fig:lae_P1}) the overall error is around an order of magnitude lower when comparing \cref{fig:lae_P1error,fig:lae_P2error}.
Forward and backward time integration with the \ktScheme{} is possible, but due to numerical errors we note a visible difference (most notable at $x=0.25$ and $x=0.75$ in \cref{fig:lae_P2flow}) between the initial set of volume averages $\{\bar{u}_{\mathrm{II},j}(t=0)\}$ and the one computed with backward time integration from $t=0.4$ to $\{\bar{u}_{\mathrm{II},j}^-(t=0)\}$.

The relative errors in $L^1$ norm, the rate of convergence in the scheme $\mathrm{S}$
\begin{align}
	r_\mathrm{S} \equiv \, & \ln \big( \tfrac{\epsilon_{\mathrm{S}, i}}{\epsilon_{\mathrm{S}, i - 1}} \big)/\ln \big( \tfrac{n_{i - 1}}{n_{i}} \big) \, ,\label{eq:rateS}
\end{align}
and the wall time in seconds for different number of volume cells $n=\{32,\ldots,1024\}$ for the first- and second-order \ktScheme{}, \cf{} \cref{paragraph:KTO1}, are shown in \cref{tab:lae_P2convergence} for the smooth \ic{} \eqref{eq:laeICS2}.
We observe a rather consistent rate of convergence of $\sim0.97$ for the formally first-order accurate scheme and $\sim1.89$ for the formally second-order accurate scheme.
For this rather simple example we can numerically integrate up to $t=0.4$ in under two seconds using $n=1024$ cells and a notebook CPU from 2018.

\begin{table}[t]
	\centering
	\caption{\label{tab:lae_P2convergence}
		Relative errors in $L^1$-norm $\epsilon$ using \cref{eq:laeSol,eq:laeICS2} as reference in \cref{eq:L1error}, convergence rate $r$ from \cref{eq:rateS}, and wall time $t_\mathrm{w}$ in seconds for first- and second-order \ktScheme{} numerical solutions at $t=0.4$ of the \laeq{} with \ic{} \nolinebreak[3]\eqref{eq:laeICS2} and periodic \bc{} \eqref{eq:BCperiodic} in the interval $x_0=0$ and $x_{n}=1$ with varying number $n$ of equidistantly spaced volume cells, \cf{} subsection 2.1.3 of the auxiliary notebook~\cite{Steil:2023PhDFVNB}.
		The wall time refers to a single run on an \intel{} running up to 6 threads simultaneously using the auxiliary notebook~\cite{Steil:2023PhDFVNB} with \WAMvwR{}.
	}
	\vspace{\TableAbovecaptionskip}
	\renewcommand{\arraystretch}{1.15}
	\small
	\begin{tabular}{l c c c p{0.2em} c c c}
		\toprule
		&	\multicolumn{3}{c}{KT $\ordern{\Delta x^1}$}&														&	\multicolumn{3}{c}{KT $\ordern{\Delta x^2}$}
		\\\cmidrule(lr){2-4}																	\cmidrule(lr){6-8}
		$n$		&	$\epsilon_\mathrm{KTO1}$	&	$r_\mathrm{KTO1}$	&	$t_\mathrm{W}\,(\mathrm{s})$&	&	$\epsilon_\mathrm{KTO2}$	&	$r_\mathrm{KTO2}$	&	$t_\mathrm{W}\,(\mathrm{s})$
		\\
		\midrule\addlinespace[0.25em]
			$32$	&	$1.43\cdot 10^{-1}$	&	-	&	$3.96\cdot 10^{-3}$	&		&	$3.12\cdot 10^{-2}$	&	-	&	$7.70\cdot 10^{-2}$\\
			$64$	&	$7.51\cdot 10^{-2}$	&	$0.93$	&	$4.96\cdot 10^{-3}$	&		&	$8.67\cdot 10^{-3}$	&	$1.85$	&	$1.48\cdot 10^{-1}$\\
			$128$	&	$3.84\cdot 10^{-2}$	&	$0.97$	&	$1.50\cdot 10^{-2}$	&		&	$2.38\cdot 10^{-3}$	&	$1.87$	&	$1.79\cdot 10^{-1}$\\
			$256$	&	$1.94\cdot 10^{-2}$	&	$0.98$	&	$4.00\cdot 10^{-2}$	&		&	$6.36\cdot 10^{-4}$	&	$1.90$	&	$1.77\cdot 10^{-1}$\\
			$512$	&	$9.76\cdot 10^{-3}$	&	$0.99$	&	$1.60\cdot 10^{-1}$	&		&	$1.69\cdot 10^{-4}$	&	$1.91$	&	$3.81\cdot 10^{-1}$\\
			$1024$	&	$4.90\cdot 10^{-3}$	&	$1.00$	&	$4.43\cdot 10^{-1}$	&		&	$4.44\cdot 10^{-5}$	&	$1.93$	&	$1.32\cdot 10^{+0}$\\
		\bottomrule
	\end{tabular}
\end{table}
		
\paragraph{A non-linear advection equation}\phantomsection\label{paragraph:BBE}\mbox{} \\
So far we discussed the arguably rather simple \laeq{}.
In the following we want to focus on one particular non-linear but still quasilinear advection equation, \viz{} the \bbe{}~\cite{Bateman1915,Burgers1948}
\begin{align}
	\partial_t u(t,x) +\partial_x\del[1]{\tfrac{1}{2}u(t,x)^2}&=\partial_t u(t,x) +u(t,x)\partial_x u(t,x)=0\, , \label{eq:bbe}\\[.25em]
	x(\tau)&=u(t=0,\tau)t+\tau\label{eq:bbeChar}\, ,\\
	u(t,x)&=u(t=0,\tau)=u(t=0,x-u(t,x) t)\label{eq:bbeSolImp}\, ,\\[.25em]
	u(t=0,x)&=\sin\del[1]{2\piu x},\label{eq:bbeICS}
\end{align}
with the \bbeq{} and an explicit \ic{} \eqref{eq:bbeICS}, which we will use in the following.
The \bbe{} with this \ic{} \eqref{eq:bbeICS} is also discussed in Sec. 6.2 of \ccite{KTO2-0} as a standard benchmark problem.
Using the method of characteristics with the \cref{eq:MoC_odes,eq:MoC_ics} yields the non-trivial characteristic curve \eqref{eq:bbeChar} and 
in consequence the implicit symbolic solution \eqref{eq:bbeSolImp} in terms of the initial condition $u(t=0,x)$.
\fullWidthFigure%
	[t]%
	{hydro/figures/bbe_flow.pdf}% Graphics
	[fig:bbe_flow,fig:bbe_char]% Sublabels
	{%
	Analytical/reconstructed (solid lines) and numerical solution (dots) to the \bbe{} with \ic{} \eqref{eq:bbeICS} and periodic \bc{} \eqref{eq:BCperiodic} at different times on the left \subref{fig:bbe_flow} and corresponding characteristic curves on the right \subref{fig:bbe_char}.
	The shock position at $\xi_s(t)=1/2$ for $t\ge 1/(2\uppi)$ is marked as a black line, the shock formation time is marked as a red-dashed line, and the constant function value on the characteristics are color coded (yellow for $+1$, green for $0$ and blue for $-1$) on the right \subref{fig:bbe_char}.
	The numerical solution has been computed with the \ktScheme{} using $n=101$ volume cells equidistantly spaced in the interval $x_0=0$ and $x_{100}=1$, \cf{} section 2.2 of the auxiliary notebook~\cite{Steil:2023PhDFVNB}.
	}%Caption
	{fig:bbe}%Label
The analytical and numerical solution of the \bbe{} with the \ic{} \eqref{eq:bbeICS} are visualized in \cref{fig:bbe} including the corresponding characteristic curves.
We observe that the first characteristic curves intersect at $x=1/2\equiv \xi_s$ and $t=1/(2\uppi)\equiv t_s$, thus, according to \cref{subsec:MoC}, a shock forms.
Specifically, by means of the Rankine-Hugoniot condition \eqref{eq:FVRHcondition}, we find a standing shock at $\xi_s(t)=1/2$ for $t\ge 1/(2\uppi)$.
Since the flow is exactly mirrored around $x=1/2$, we observe that the opposing waves annihilate since they travel at opposite velocity according to $\partial F/\partial u = u$.
For times $t>t_s$ \dash{} after the shock formation \dash{} the implicit solution \eqref{eq:bbeSolImp} becomes multi-valued, but due to the symmetry of the problem and the fact that the shock is standing, it is not difficult to reconstruct the solution for $t>t_s$, \cf{} section 2.2 of the auxiliary notebook~\cite{Steil:2023PhDFVNB}.
For $0\le t\le t_s$ we use the implicit solution \eqref{eq:bbeSolImp} and for $t>t_s$ we use the reconstructed solution as reference.
Intuitively (imaging the situation in terms of waves) and for the \cfd{} initiated this example of shock formation might seem rather natural, but let us make one important mathematical observation:
non-linear advection equations can develop shocks \dash{} even if the initial condition is smooth and even if the advection flux is simple, \ie{}, for the \bbeq{} $F[u]=\frac{1}{2}u^2$.
In turn, any numerical scheme deployed in practice for non-linear advection equations or equations involving such non-linear hyperbolic contributions should be able to handle discontinuities in $u(t,x)$. 
This simple statement outright disqualifies large classes of methods for \pdes{} for this application.
Such methods are for example naive \acrpllong{fd} (the large gradients induced by the shock will destabilize the method-of-lines \ode{} system) and global collocation methods, which in principle rely on analyticity or at least smoothness (the shock discontinuity will again destabilize the \ode{} system in spectral space due to the Wilbraham-Gibbs phenomenon~\cite{Wilbraham:1848,Gibbs:1898,Gibbs:1899}, see, \eg{}, the textbook~\cite{boyd2001chebyshev}). 
Recall the related discussion of such issues in the context of truncated flow equations in \cref{paragraph:taylorExpansion}.

\fullWidthFigure%
	[t]%
	{hydro/figures/bbe_errors.pdf}% Graphics
	[fig:bbe_tvd,fig:bbe_L1errorrs]% Sublabels
	{%
	Numerical entropy $\mathcal{C}_\mathrm{TV}$ for the \bbe{} with \ic{} \eqref{eq:bbeICS} and periodic \bc{} \eqref{eq:BCperiodic} at different times on the left \subref{fig:bbe_tvd} and related relative errors on the right \subref{fig:bbe_L1errorrs}.
	The solid line corresponds to forward time integration and the three dashed lines in the right figure \subref{fig:bbe_L1errorrs} are errors of three attempts of backwards time integration starting from $t=1/(2\uppi)$, $t=1.25/(2\uppi)$, and $t=2/(2\uppi)$ respectively.
	The numerical solution has been computed with the \ktScheme{} using $n=101$ volume cells equidistantly spaced in the interval $x_0=0$ and $x_{100}=1$, \cf{} section 2.1 of the auxiliary notebook~\cite{Steil:2023PhDFVNB}.
	}%Caption
	{fig:bbe_errors}%Label
The shock formation in this example also has implications for the reversibility of non-linear advection equations.
It is well known from the study of non-linear advection equations, see, \eg{}, the textbooks~\cite{Lax1973,Ames:1992,LeVeque:1992,LeVeque:2002,Hesthaven2007,Toro2009,RezzollaZanotti:2013}, that there is a meaningful notion of numerical entropy and that its increase is linked to the appearance and/or interaction of discontinuities like shocks.
An increase in numerical entropy signals the irreversibility of the underlying flow, see, \eg{}, \ccite{LeVeque:2002,RezzollaZanotti:2013} for this in the context of non-linear (especially hyperbolic) conservation laws.
It turns out that the \tv{} \dash{} arc-length \dash{} from \cref{eq:TVdiscrete} can, due to the \tvni{} property, be used to define a numeric entropy functional
\begin{align}
	\mathcal{C}_\mathrm{TV}[\{ \bar{u}_j (t) \}]\equiv \mathrm{TV}[ \{ \bar{u}_j ( 0 ) \} ]-\mathrm{TV}[ \{ \bar{u}_j (t) \} ]\, ,\label{eq:TVDentropy}
\end{align}
which is due to the chosen sign convention only increasing: $\partial_t \mathcal{C}_\mathrm{TV}[\{ \bar{u}_j (t) \}]\ge 0$ during time evolution.
The notion of numerical entropy (and \tv{} as a possible candidate for it) is very important in the study, construction and numerical computation of physical weak solutions of conservative equations, see, \eg{}, the textbooks~\cite{Lax1973,Ames:1992,LeVeque:1992,LeVeque:2002,Hesthaven2007,Toro2009,RezzollaZanotti:2013} for further details.
We plot $\mathcal{C}_\mathrm{TV}$ for the current example in \cref{fig:bbe_tvd} and observe an increase of numerical entropy in general as expected.
However, after shock formation \dash{} if only delayed at $t\sim1.5$ \dash{} a massive increase in numerical entropy can be observed, which we identify with a delayed signal of the shock formation in $\mathcal{C}_\mathrm{TV}$.
\Cref{fig:bbe_L1errorrs} includes the corresponding relative $L^1$ errors, including three attempts at backwards time integration. 
The first one starting exactly at  $t=1/(2\uppi)$ \dash{} the point where the shock forms \dash{}  and integrating back to $t=0$ is comparable to the ones in \cref{fig:lae_P1error,fig:lae_P2error} and is successful in reproducing the \ic{} up to an observed accuracy.
The two attempts of starting from $t=1.25/(2\uppi)$ and $t=2/(2\uppi)$ fail: the numerical error is significantly increased and even more severe since the qualitative features of the \ic{} at $t=0$ are not reproduced correctly. 

We conclude the discussion of the \bbe{} with remarks on the rate of convergence and errors, \cf{} \cref{tab:BBEconvergence}.
The wall time for the $n=1024$ computation in \cref{tab:BBEconvergence} is still below two seconds, \cf{} \cref{tab:lae_P2convergence}, even for this non-linear advection equation.
Before the shock at $t=0.5/(2\uppi)$ we observe a rate of convergence of $\sim 1.9$ for $\epsilon_{L^1}$ and $\sim 1.3$ for $\epsilon_{L^\infty}$, which quantifies the largest deviation in the \fv{} grid from the reference solution:
\begin{align}
	\epsilon_{L^\infty}(t)=\sup_j|\bar{u}_j(t)-u(t,x_j)|\,.	\label{eq:Linftyerror}
\end{align}
After the shock at $t=1.5/(2\uppi)$ we observe a reduced rate of convergence of $\sim 1.1$ for $\epsilon_{L^1}$ and no convergence in $\epsilon_{L^\infty}$, since there are always some outliers directly at the shock front. 
For a more refined discussion of errors and scaling post-shock-formation in this scenario we refer the interested reader to \ccite{KTO2-0,Nessyahu2006Jul}.
\begin{table}[t]
	\centering
	\caption{\label{tab:BBEconvergence}
		Relative errors in $L^1$-norm $\epsilon_{L^1}$ using \cref{eq:L1error} and errors in $L^\infty$-norm  $\epsilon_{L^\infty}$ using \cref{eq:Linftyerror} and corresponding convergence rates at $t=0.5/(2\uppi)<t_s$ and $t=1.5/(2\uppi)>t_s$ \dash{} before and after shock formation \dash{} between the constructed reference solution and the numerical solutions obtained with the \ktScheme{} for the \bbeq{} with \ic{} \eqref{eq:bbeICS} and periodic \bc{} \eqref{eq:BCperiodic} in the interval $x_0=0$ and $x_{n}=1$ with varying number $n$ of equidistantly spaced volume cells, \cf{} subsection 2.2.3 of the auxiliary notebook~\cite{Steil:2023PhDFVNB}.
	}
	\vspace{\TableAbovecaptionskip}
	\renewcommand{\arraystretch}{1.15}
	\small
	\begin{tabular}{l c c c c  p{0.2em}  c c c c}
		\toprule
		&	\multicolumn{4}{c}{$t=0.5/(2\uppi)<t_s$}							&						&	\multicolumn{4}{c}{$t=2.0/(2\uppi)>t_s$}
		\\\cmidrule(lr){2-5}	\cmidrule(lr){7-10}
		$n$		&	$\epsilon_{L^1}$	&	$r$	&	$\epsilon_{L^\infty}$	&	$r$	&	&$\epsilon_{L^1}$	&	$r$	&	$\epsilon_{L^\infty}$	&	$r$	
		\\
		\midrule\addlinespace[0.25em]
			$32$	&	$6.50\cdot 10^{-3}$	&	-	&	$2.42\cdot 10^{-2}$	&	-	&		&	$2.01\cdot 10^{-2}$	&	-	&	$2.68\cdot 10^{-1}$	&	-\\
			$64$	&	$1.75\cdot 10^{-3}$	&	$1.89$	&	$1.04\cdot 10^{-2}$	&	$1.22$	&		&	$9.02\cdot 10^{-3}$	&	$1.16$	&	$2.64\cdot 10^{-1}$	&	$+0.02$\\
			$128$	&	$4.59\cdot 10^{-4}$	&	$1.93$	&	$4.21\cdot 10^{-3}$	&	$1.30$	&		&	$4.32\cdot 10^{-3}$	&	$1.06$	&	$2.68\cdot 10^{-1}$	&	$-0.02$\\
			$256$	&	$1.20\cdot 10^{-4}$	&	$1.93$	&	$1.70\cdot 10^{-3}$	&	$1.31$	&		&	$2.16\cdot 10^{-3}$	&	$1.00$	&	$2.68\cdot 10^{-1}$	&	$-0.00$\\
			$512$	&	$3.23\cdot 10^{-5}$	&	$1.90$	&	$6.88\cdot 10^{-4}$	&	$1.30$	&		&	$1.07\cdot 10^{-3}$	&	$1.01$	&	$2.68\cdot 10^{-1}$	&	$+0.00$\\
			$1024$	&	$8.46\cdot 10^{-6}$	&	$1.93$	&	$2.79\cdot 10^{-4}$	&	$1.30$	&		&	$3.91\cdot 10^{-4}$	&	$1.45$	&	$1.96\cdot 10^{-1}$	&	$+0.46$\\
		\bottomrule
	\end{tabular}
\end{table}

\subsection{Diffusion and the heat equation}\label{subsec:hydroDiffusion}
\begin{disclaimer}
	Parts of this subsubsection include material from Sec.~III.B and App.~B of \nbccite{Koenigstein:2021numericalSchemes}.
\end{disclaimer}
In this subsection we will discuss the diffusive contributions to \cref{eq:FVadsEq}, \viz{} the ones governed by the diffusive term $\difx\! Q_i [ t, x, \{u_i ( t, x )\}, \{\partial_x u_i ( t, x )\} ]$.
Focusing on the latter, let us consider the non-linear system of equations
\begin{align}
	\dift\! u_i ( t, x )=\difx\! Q_i [ t, x, \{u_i ( t, x )\}, \{\partial_x u_i ( t, x )\} ]\,,\label{eq:FVadsEqQonly}
\end{align}
which can be brought into primitive form
\begin{align}
	\partial_t u_i ( t, x )-\frac{\partial Q_i [ \ldots ]}{\partial (\partial_x u_l)} \partial_x^2 u_l(t,x) = \frac{\partial Q_i [\ldots]}{\partial u_l} \partial_x u_l(t,x) +\partial_x Q_i [\ldots]\,.\label{eq:FVQprimitive}
\end{align}
\Cref{eq:FVQprimitive} includes the matrix of diffusion coefficients $\frac{\partial Q_i}{\partial (\partial_x u_l)}$ on the \lhs{} in the actual diffusion term $\frac{\partial Q_i}{\partial (\partial_x u_l)} \partial_x^2 u_l$ and on the \rhs{} an advection term $\frac{\partial Q_i}{\partial u_l} \partial_x u_l$ and an internal source term $\partial_x Q_i$ related to the explicit $u$- and $x$-dependency of the diffusion flux $Q$.
In this sense position dependency of the diffusion flux can be understood in terms of additional/internal source terms and dependency on the velocities $u$ manifest as additional advective contributions.
Depending on the application it might be advantageous to separate those out to integrate them properly in the (numerical) advection flux $F$ ($H_{i,j+\ttfrac{1}{2}}$).
\Cref{eq:FVQprimitive}, or in our context the flux $Q$ itself, is understood to be \textit{parabolic}, if the matrix of diffusion coefficients satisfies the \textit{(weak) parabolicity condition} $\frac{\partial Q_i}{\partial (\partial_x u_l)}\ge 0$ for all $t$, $x$, and $u_i ( t, x )$ under consideration~\cite{KTO2-0}.
In the following we will usually refer to this constraint using the more descriptive term: \textit{positivity of diffusion coefficient(s)}.
The numerical \kt{} diffusion flux \eqref{eq:kt_original_diffusion} $P_{i,j + \ttfrac{1}{2}}$ is applicable to such diffusion fluxes.
Diffusive systems fulfilling the (weak) parabolicity condition typically have well posed \ivps{}, if suitable boundary conditions are supplied~\cite{Krylov1987,Lieberman1996,Hunter2014,KTO2-0}.
Parabolic systems typically describe diffusive/dissipative processes which involve a smoothing of information through space-time with formally infinite propagation speeds\footnote{\label{footnote:HEinf}%
	This infinite speed of information propagation can be studied with the fundamental solution (heat kernel) of the heat \cref{eq:he}: a point-like heat source spreads out like a Gaussian function.
	This implies that, even after an infinitesimal time-step, the effect of the heat source is felt in the entire computational domain.
	Albeit only exponentially suppressed at large distances, this still formally constitutes a propagation of information at infinite speed.
	For details, see, \eg{}, Chap. 5 of \ccite{Hunter2014} and Sec. 6.5 of \ccite{RezzollaZanotti:2013} for a comment on this in the context of relativistic hydrodynamics.
}~\cite{Hunter2014,RezzollaZanotti:2013,Rezolla2020}.

Parabolic \pdes{} describe inherently irreversible processes from a \cfd{} point of view: time reversal for a parabolic \pde{} amounts to forward time integration with a negative diffusion coefficient.
Such a process of \textit{reverse diffusion} \dash{} transport towards regions of higher concentration \dash{} amplifies existing gradients in $u(t,x)$ without any limit.
In a practical numerical computation any discretization errors (to an extent present in all discretization schemes with a finite number of cells) get amplified throughout the computational domain and introduce rapidly growing spurious oscillations making time reversal practically impossible.
This is again linked to the aforementioned concept of numerical entropy, which in parabolic systems grows as the arc length of solutions decreases.
The latter can be intuitively linked to the dissipative nature of parabolic systems: gradients get smoothed out resulting in a reduction of arc length.

\paragraph{The heat equation}\phantomsection\label{paragraph:HE}\mbox{} \\
In the following we will limit our discussion to one archetypical parabolic diffusion equation, \viz{} the \he{}~\cite{Fourier2009Jul}
\begin{align}
\partial_t u(t,x) &= \partial_x^2u(t,x) \label{eq:he}
\end{align}
in its mathematical form, see, \eg{}, the textbooks~\cite{Cannon:1984,LeVeque:1992,LeVeque:2002} for additional details.
In physics this equation can be used to study heat diffusion and goes back to Joseph J. B. Fourier's seminal work~\cite{Fourier2009Jul}: \emph{Th{\ifmmode\acute{e}\else\'{e}\fi}orie \
analytique de la chaleur} on the topic from 1822. 
When using it in this context $u(t,x)$ would be a temperature distribution $T(t,x)$ and the \rhs{} of \cref{eq:he} would include a dimensionful, positive diffusion coefficient $\alpha$, called the \textit{thermal diffusivity} of the medium, \ie{}, $\alpha=111\,\mathrm{mm}^2/\mathrm{s}$ for copper at 25\textdegree{}C~\cite{CASALEGNO201083}.
In order to define a well-posed \ivp{} for the \heq{} it is imperative to specify \bcs{}.
In the following we will study the \heq{} with two different boundary conditions \eqref{eq:heICS1DBC} and \eqref{eq:heICS2NBC} while using the identical \ics{}~\eqref{eq:heICS1} and \eqref{eq:heICS2}.

\fullWidthFigure%
	[t]%
	{hydro/figures/he.pdf}% Graphics
	[fig:heP1,fig:heP2]% Sublabels
	{%
	Semi-analytical (solid lines) and numerical solution (dots) to the \heq{} with identical \ics{} \eqref{eq:heICS1}/\eqref{eq:heICS2}, Dirichlet \bc{} \eqref{eq:heICS1DBC} on the left \subref{fig:heP1}, and  Neumann \bc{}~\eqref{eq:heICS2NBC} on the right \subref{fig:heP2} at different times.
	The numerical solution has been computed with the \ktScheme{} using $n=101$ volume cells equidistantly spaced in the interval $x_0=0$ and $x_{100}=1$, \cf{} section 2.3 of the auxiliary notebook~\cite{Steil:2023PhDFVNB}.
	}%Caption
	{fig:he}%Label
Setting \textit{Dirichlet} \bcs{} by fixing the values of $u(t,x)$ on the boundaries of the computational domain ($x_{-1/2}$ and $x_{n-1/2}$) amounts to connecting the \he{} with a heath bath:
\begin{align}
\stepcounter{equation}\newSubEqBlock
u_\mathrm{I}(t=0,x)&=1+\tfrac{1}{2}\cos(\piu\, x)\, ,\label{eq:heICS1}\subEqTag\\
u_\mathrm{I}(t,x_{-1/2})&=\tfrac{3}{2}, \quad u_\mathrm{I}(t,x_{n-1/2})=\tfrac{1}{2}\label{eq:heICS1DBC}\, ,\subEqTag\\
u_\mathrm{I}(t,x)&= \tfrac{3}{2} - x + \sum_{n = 1}^{\infty} \frac{2 \sin ( 2 n \piu \, x )}{( 2 n \piu ) \, ( (2 n )^2 + 1)} \, \eu^{-( 2 n \piu )^2 \, t}\, ,\label{eq:heICS1sol}\subEqTag\\
\newSubEqBlockPrime\setcounter{subeqPrime}{2}
 \Rightarrow u_{\mathrm{I},\mathrm{e}}(x)&\equiv \lim_{t\rightarrow\infty}u_\mathrm{I}(t,x)=\tfrac{3}{2}-x\subEqTagPrime\label{eq:heICS1asymp}\, .
\end{align}
Choosing \textit{Neumann} \bcs{} by fixing the derivatives $\partial_x u(t,x)$ to zero on the boundaries of the computational domain amounts to isolating \bc{} \dash{} preventing any in- or out-flow of heat into or out of the computational domain:
\begin{align}
\stepcounter{equation}\newSubEqBlock
u_\mathrm{II}(t=0,x)&\equiv u_\mathrm{I}(t=0,x) =1+\tfrac{1}{2}\cos(\piu\, x)\,\label{eq:heICS2}\subEqTag\\
\partial_x u_\mathrm{II}(t,x)|_{x=x_{-1/2}}&=\partial_x u_\mathrm{II}(t,x)|_{x=x_{n-1/2}}=0\,\label{eq:heICS2NBC}\subEqTag\\
u_\mathrm{II}(t,x)&= 1 + \tfrac{1}{2} \cos ( \piu \, x ) \, \eu^{- \piu^2 \, t} \label{eq:heICS2sol}\,\subEqTag\\* % no page break
\newSubEqBlockPrime\setcounter{subeqPrime}{2}
 \Rightarrow u_{\mathrm{II},\mathrm{e}}(x)&\equiv \lim_{t\rightarrow\infty}u_\mathrm{II}(t,x)=1\subEqTagPrime\label{eq:heICS2asymp}\,.
\end{align}
The specific \textit{Dirichlet} \bcs{} \eqref{eq:heICS1DBC} are readily implemented in our \fv{} scheme by choosing
\begin{align}
	\bar{u}_{-2}&=\tfrac{3}{2}\,,\quad \bar{u}_{-1}=\tfrac{3}{2}\,,\quad
	\bar{u}_{n}=\tfrac{1}{2}\,,\quad\quad \bar{u}_{n+1}=\tfrac{1}{2}\,,
	\label{eq:heICS1DBCFV}
\end{align}
for the ghost cells.
While the \textit{Neumann} \bcs{} \eqref{eq:heICS2NBC} are realized by choosing
\begin{align}
	\bar{u}_{-2}&=\bar{u}_{0}\,,\quad \bar{u}_{-1}=\bar{u}_{0}\,,\quad
	\bar{u}_{n}=\bar{u}_{n-1}\,,\quad\quad \bar{u}_{n+1}=\bar{u}_{n-1}\,.
	\label{eq:heICS2NBCFV}
\end{align}
The semi-analytical/symbolic solutions \eqref{eq:heICS1sol} and \eqref{eq:heICS2sol} can be computed using a separation ansatz and employing a Fourier/heat kernel expansion~\cite{Fourier2009Jul,Koenigstein:2021numericalSchemes}, see, \eg{}, Sec. 8.3 of \ccite{Rezolla2020}.
Resulting flows for both problems are visualized in \cref{fig:he}.
Using the numerical diffusion flux \eqref{eq:kt_original_diffusion} from \cref{paragraph:KTQS} we are able to reproduce the semi-analytical reference solutions \eqref{eq:heICS1sol} and \eqref{eq:heICS2sol}.
We can observe how the flows approach their respective equilibrium solutions \eqref{eq:heICS1asymp} and \eqref{eq:heICS2asymp} at late times.
\Cref{eq:heICS1asymp} describes a constant temperature gradient between the two heat baths left and right of the computation domain, while \cref{eq:heICS2asymp} describes a constant temperature distribution which is solely determined by the heat/energy of the \ic{}.
Even though we study the same \pde{} with identical \ics{} $u_\mathrm{I}(t=0,x)=u_\mathrm{II}(t=0,x)$, we observe completely different behavior based on the two distinct \bcs{}.
This highlights the imperative importance of \bc{} for convection and especially parabolic equations.

We conclude the discussion of the \he{} with remarks on the rate of convergence and errors, \cf{} \cref{tab:HEconvergence}.
The wall time for the $n=1024$ computation in \cref{tab:HEconvergence} is almost six seconds, \cf{} \cref{tab:lae_P2convergence}.
Somewhat curiously we observe a convergence rate of basically $1$, even though we are technically using a four-point stencil based on \cref{eq:kt_original_diffusion}, \cf{} \cref{paragraph:KTQS}, for which one would expect second-order accuracy.
This highlights the fact that the practical rate of convergence is rather problem specific in numerical applications of the \ktScheme{}.

\begin{table}[t]
	\centering
	\caption{\label{tab:HEconvergence}
		Relative errors in $L^1$-norm $\epsilon_{L^1}$ using \cref{eq:L1error} and errors in $L^\infty$-norm  $\epsilon_{L^\infty}$ using \cref{eq:Linftyerror} and corresponding convergence rates at $t=0.25$ between the semi-analytical solutions and the numerical solutions obtained with the \ktScheme{} for the \heq{} with \bcs{} \eqref{eq:heICS1DBC} and \eqref{eq:heICS2NBC} in the interval $x_0=0$ and $x_{n}=1$ with varying number $n$ of equidistantly spaced volume cells, \cf{} subsection 2.3.3 of the auxiliary notebook~\cite{Steil:2023PhDFVNB}.
	}
	\vspace{\TableAbovecaptionskip}
	\renewcommand{\arraystretch}{1.15}
	\small
	\begin{tabular}{l c c c c  p{0.2em}  c c c c}
		\toprule
		&	\multicolumn{4}{c}{Dirichlet \bc{} \eqref{eq:heICS1DBC} \dash{} Heat bath}							&						&	\multicolumn{4}{c}{Neumann \bc{} \eqref{eq:heICS2NBC} \dash{} Isolating}
		\\\cmidrule(lr){2-5}	\cmidrule(lr){7-10}
		$n$		&	$\epsilon_{L^1}$	&	$r$	&	$\epsilon_{L^\infty}$	&	$r$	&	&$\epsilon_{L^1}$	&	$r$	&	$\epsilon_{L^\infty}$	&	$r$	
		\\
		\midrule\addlinespace[0.25em]
			$32$	&	$1.56\cdot 10^{-2}$	&	-	&	$3.03\cdot 10^{-2}$	&	-	&		&	$4.48\cdot 10^{-3}$	&	-	&	$7.73\cdot 10^{-3}$	&	-\\
			$64$	&	$7.81\cdot 10^{-3}$	&	$1.00$	&	$1.54\cdot 10^{-2}$	&	$0.98$	&		&	$2.14\cdot 10^{-3}$	&	$1.06$	&	$3.73\cdot 10^{-3}$	&	$1.05$\\
			$128$	&	$3.90\cdot 10^{-3}$	&	$1.00$	&	$7.75\cdot 10^{-3}$	&	$0.99$	&		&	$1.05\cdot 10^{-3}$	&	$1.03$	&	$1.83\cdot 10^{-3}$	&	$1.03$\\
			$256$	&	$1.95\cdot 10^{-3}$	&	$1.00$	&	$3.89\cdot 10^{-3}$	&	$0.99$	&		&	$5.19\cdot 10^{-4}$	&	$1.02$	&	$9.08\cdot 10^{-4}$	&	$1.01$\\
			$512$	&	$9.76\cdot 10^{-4}$	&	$1.00$	&	$1.95\cdot 10^{-3}$	&	$1.00$	&		&	$2.58\cdot 10^{-4}$	&	$1.01$	&	$4.52\cdot 10^{-4}$	&	$1.01$\\
			$1024$	&	$4.88\cdot 10^{-4}$	&	$1.00$	&	$9.76\cdot 10^{-4}$	&	$1.00$	&		&	$1.29\cdot 10^{-4}$	&	$1.00$	&	$2.26\cdot 10^{-4}$	&	$1.00$\\
		\bottomrule
	\end{tabular}
\end{table}

\subsection{Sources and the heat equation}\label{subsec:hydroSource}
\fullWidthFigure%
	[t]%
	{hydro/figures/hes.pdf}% Graphics
	[fig:hesP1,fig:hesP2]% Sublabels
	{%
	Solutions to the \heq{} with \ics{} \eqref{eq:heICS1},\eqref{eq:heICS2}, Dirichlet \bc{} \eqref{eq:heICS1DBC} on the left \subref{fig:heP1}, and  Neumann \bc{} \eqref{eq:heICS2NBC} on the right \subref{fig:heP2} at different times including the source \eqref{eq:heSource}.
	The numerical solution has been computed with the \ktScheme{} using $n=101$ volume cells equidistantly spaced in the interval $x_0=0$ and $x_{100}=1$, \cf{} section 2.4 of the auxiliary notebook~\cite{Steil:2023PhDFVNB}.
	}%Caption
	{fig:hes}%Label
At this point we only want to comment briefly on the inclusion of explicit source terms.
We will do so by including the explicit source term
\begin{align}
	S(x)&=\sin(\uppi\, x)
	\label{eq:heSource}
\end{align}
on the \rhs{} of the \heq{} and by studying the two examples introduced in the previous \cref{paragraph:HE}.
In the case of Dirichlet \bc{} \eqref{eq:heICS1DBC} it is possible to construct a semi-analytic solution
\begin{subequations}\label{eq:heICS1S}
\begin{align}
	u_\mathrm{IS}(t,x)&= \frac{3}{2} - x+\frac{1-\eu^{- \piu^2 \, t}}{\uppi ^2}\sin (\uppi\,x) + \sum_{n = 1}^{\infty} \frac{2 \sin ( 2 n \piu  x )}{( n \piu ) \, ( (2 n )^2 - 1)} \, \eu^{-( 2 n \piu )^2 \, t}\label{eq:heICS1Ssol}\,,\\
	\Rightarrow u_{\mathrm{IS},\mathrm{e}}(t,x)&\equiv\lim_{t\rightarrow\infty}u_\mathrm{I}(t,x)=\frac{3}{2} - x+\frac{1}{\uppi ^2}\sin (\uppi\,x)\label{eq:heICS1Sasymp}\,,
\end{align}
\end{subequations}
by first computing an equilibrium solution $u_{\mathrm{IS},\mathrm{e}}(t,x)\equiv\lim_{t\rightarrow\infty}u_\mathrm{IS}(t,x)$ and then solving the homogeneous system with the established Fourier methods by considering perturbations around $u_{\mathrm{IS},\mathrm{e}}(t,x)$, \cf{} section 2.4 of the auxiliary notebook~\cite{Steil:2023PhDFVNB}.
For the isolating Neumann \bc{} \eqref{eq:heICS2NBC} such a construction is not possible, since there is no equilibrium solution \dash{} the continuous heating in the isolated computational interval leads to a steady increase in $u(x,t)$.
Results from the \ktScheme{} with the source term \eqref{eq:heSource} naively discretized in the volume cells by its midpoint value, \cf{} \cref{paragraph:KTQS}, are shown in \cref{fig:hes}.
\Cref{fig:hesP1} also includes the semi-analytical solution \eqref{eq:heICS1Ssol}, which we are able to reproduce with our numerical scheme.
For large times the equilibrium solution \eqref{eq:heICS1Sasymp} is approached in \cref{fig:hesP1}, while \cref{fig:hesP2} displays a continuously rising but flattening temperature distribution.

A detailed discussion of sinks and sources in the context of our research can be found in \cref{paragraph:chemical_potential_shock_wave}.

\subsection{Euler equations -- the KT scheme showing its MUSCLes}\label{subsec:hydroEuler}
To conclude our discussion of \cfd{} and the \ktScheme{} we want to present one more involved example of advection equations, \viz{} the system of \textit{Euler equations}~\cite{Euler1757}
\begin{align}
\partial_t\begin{pmatrix}
	\rho(t,x)\\
	\mu(t,x)\\
	\epsilon(t,x)
\end{pmatrix}
+\partial_x
\begin{pmatrix}
 \mu(t,x)  \\
 p(t,x)+\rho(t,x)  v(t,x)^2 \\
 v(t,x) (p(t,x)+\epsilon(t,x) )
\end{pmatrix}
=\begin{pmatrix} 0 \\ 0 \\ 0 \end{pmatrix}\, ,\label{eq:euler}
\end{align}
including the derived quantities pressure $p=(\gamma-1)(\epsilon-\tfrac{\rho}{2}v^2)=\rho \mathrm{R}_\mathrm{s} T$ and velocity $v=\mu/\rho$.
The vector of conserved quantities $\vec{u}=(\rho,\mu,\epsilon)^\tr$ includes the mass density $\rho$, the momentum density $\mu$, and the energy density $\epsilon$.
The \eulereqs{} describe the advective evolution of an inviscid, compressible, adiabatic, \ie{}, ideal, fluid in one spatial $x$ and one temporal $t$ dimension.
They are based on the conservation of mass, momentum, and energy and are frequently used in \cfd{} to construct challenging benchmark problems for numerical schemes.

\fullWidthFigure%
	[t]%
	{hydro/figures/ee_sod.pdf}% Graphics
	[fig:ee_sodrho,fig:ee_sodmu,fig:ee_sodepsilon]% Sublabels
	{%
	Semi-analytical (solid lines) and numerical solutions (dots) of the Sod shock tube problem \eqref{eq:sodICS} for the \eulereqs{}, with density on the left \subref{fig:ee_sodrho}, momentum density in the middle \subref{fig:ee_sodmu}, and energy density on the right \subref{fig:ee_sodepsilon} at $t=0.1644\,\mathrm{s}$ with the initial condition at $t=0$ as reference (dashed lines).
	The semi-analytical solution has been obtained using an exact Riemann-Solver~\cite{Toro2009,ToroExact:GitHub}.
	The numerical solution has been computed with the \ktScheme{} using $n=201$ volume cells equidistantly spaced in the interval $x_0=0$ and $x_{200}=1\,\mathrm{m}$, \cf{} subsection 2.5.2 of the auxiliary notebook~\cite{Steil:2023PhDFVNB}.
	}%Caption
	{fig:ee_sod}%Label
One established problem is the so-called \textit{Sod shock tube problem}~\cite{Sod1978Apr} which is a specific Riemann problem for the \eulereqs{}: an as one-dimensional idealized tube is considered with two constant states $\vec{u}_L$ and $\vec{u}_R$, separated at initial time $t_0=0$ by an diaphragm/burst-disc.
At the begin of the computation at $t=0_+$ this diaphragm is considered to be instantaneously removed and a pure Riemann problem is simulated. 
The computational domain (time) typically under consideration is large (short) enough such that explicit boundary conditions are not required, since the dynamics do not reach the spatial computational boundary. 
Following \textsc{Example} 5 of Sec.~6.3 of \nbccite{KTO2-0}, we will consider a computational domain stretching from $x_0=0$ to $x_{n-1}=1\,\mathrm{m}$ up to a time of $t=0.1644\,\mathrm{s}$ with the two constant initial states
\begin{align}
\vec{u}_L=\begin{pmatrix}
	1.0\,\kgmIII\\
	0\,\kgms\\
	2.5\,\JmIII
\end{pmatrix}\,,\qquad \vec{u}_R=\begin{pmatrix}
	0.125\,\kgmIII\\
	0\,\kgms\\
	0.25\,\JmIII
\end{pmatrix}\, ,
\label{eq:sodICS}
\end{align}
separated at $x=0.5\,\mathrm{m}$ and assuming an adiabatic index of $\gamma=1.4$.
A numerical solution obtained with the \ktScheme{} and the semi-analytical solution obtained using an exact Riemann-Solver~\cite{Toro2009,ToroExact:GitHub} is shown in \cref{fig:ee_sod}.
We observe a shock wave traveling into the dilute right half of the tube, which is followed by a contact discontinuity.
A rarefaction fan is spreading into the denser left half of the tube.
We observe that the \ktScheme{} is able to reproduce this complicated dynamic with several extreme discontinuities perfectly (to given accuracy based on the employed spatial discretization, \ie{}, number of volume cells) without any spurious oscillations or under/over shooting around the discontinuities.

To illustrate the dynamic inside such a shock tube further we adapt a Sod shock tube problem, in parts motivated by \ccite{Coll2018}, to study a more realistic scenario.
We consider a tube of 20 meters, filled with dry air (adiabatic index $\gamma=1.4$ and specific gas constant ${\mathrm{R}_\mathrm{s}= 0.287\cdot 10^3\,\mathrm{J}\mkern1.0mu\mathrm{kg}^{-1}\mkern1.0mu\mathrm{K}^{-1}}$~\cite{dry-air-properties}), and the two constant states
\begin{align}
\vec{u}_L=\begin{pmatrix}
	1.177\,\kgmIII\\
	0\,\kgms\\
	2.533 \cdot 10^5\,\JmIII
\end{pmatrix}\,,\qquad \vec{u}_R=0.1 \vec{u}_L = \begin{pmatrix}
	1.177 \cdot 10^{-1}\,\kgmIII\\
	0\,\kgms\\
	2.533 \cdot 10^4\,\JmIII
\end{pmatrix}\, ,
\label{eq:sodPhysicalICS}
\end{align}
separated at $x=10\,\mathrm{m}$.
The \ic{} \eqref{eq:sodPhysicalICS} is constructed to have a pressure of one standard atmosphere $p_L=101,325\,\mathrm{Pa}=1\,\mathrm{atm}$ at room temperature of $T_L=299.852\,\mathrm{K}$ in the left half of the tube.
The pressure in the right half of the tube is decreased by $90\%$ to $p_R=0.1p_L=0.1\, \mathrm{atm}$ while the temperature is still  $T_R=T_L=299.852\,\mathrm{K}$.
This amounts to an adiabatic speed of sound of $c_s\simeq 347\,\ms$ in both halves.
Overall we are considering a system which has units and magnitudes which we are accustomed to in our day-to-day life \dash{} far away from the extremes of \hep{} with temperatures in excess of $10^{10}\,\mathrm{K}$ and densities well above $10^{17}\,\kgmIII$.
We now want to study the time evolution with this \ic{} from $t=0$ up to $t=60\,\mathrm{ms}$ assuming reflective boundary conditions at the edges of the computational interval
${x_{-\frac{1}{2}}=-0.025\,\mathrm{m}}$ and ${x_{n-\frac{1}{2}}=20.025\,\mathrm{m}}$, which for the \eulereqs{} can be implemented by choosing
\begin{align}
	\bar{u}_{i,-2}&=c_i\bar{u}_{i,1}\,,\quad \bar{u}_{i,-1}=c_i\bar{u}_{i,0}\,,\quad
	\bar{u}_{i,n}=c_i\bar{u}_{i,n-1}\,,\quad\quad \bar{u}_{i,n+1}=c_i\bar{u}_{i,n-2}\,,
	\label{eq:eulerReflectiveBC}
\end{align}
for the ghost cells with $c_i=1$ for $i=1$ and $i=3$ ($\rho$ and $\epsilon$) and $c_i=-1$ for $i=2$ ($\mu$)~\cite{SpringelLectureNotes}.

We visualize the solution of this \textit{physical shock tube problem} in \cref{fig:ee_sodPhysical} as a set of six density plots (including contour lines).
We plot only two primitive, conserved quantities: $\rho$ and $\epsilon$ in \cref{fig:ee_sodPhysical_rho,fig:ee_sodPhysical_epsilon} and four derived quantities.
Note that the momentum density \dash{} the missing primitive, conserved quantity \dash{} is simply given by the product of density and velocity $\mu=\rho v$, the latter is plotted in \cref{fig:ee_sodPhysical_v}.
Pressure and temperature in \cref{fig:ee_sodPhysical_p,fig:ee_sodPhysical_T} follow from the ideal gas law $p=(\gamma-1)(\epsilon-\tfrac{\rho}{2}v^2)=\rho \mathrm{R}_\mathrm{s} T$, while the (adiabatic) speed of sound in \cref{fig:ee_sodPhysical_cs} follows from $c_s^2=(\partial p/\partial \rho)_s=\gamma p/\rho=\gamma\mathrm{R}_\mathrm{s}T$.
At the beginning of time evolution we recognize the dynamics of the Sod shock tube problem: a rarefaction fan travels to the left while a contact discontinuity and a shock wave travel to the right.
Here the left edge of the rarefaction fan travels into the denser left half with a velocity of $v_{rl}\simeq -347\,\ms$ at the speed of sound.
The right edge of the rarefaction fan travels at only $v_{rr}\simeq-5\,\ms$.
The contact discontinuity is driven into the dilute right half with a substantial subsonic velocity of $v_{cd}\simeq 285\,\ms$ while velocity and pressure stay constant across it.
The shock travels at a supersonic velocity of $v_s\simeq 558\,\ms$ towards the right wall.
The dilution, \ie{}, adiabatic expansion of the gas, due to the rarefaction fan, cools the left half.
The gas in the region between the contact discontinuity and the shock front gets compressed and heated by the shock.
The latter hits the right wall at $t\simeq 17.9\,\mathrm{ms}$ and the rarefaction fan hits the left wall at $t\simeq 28.8\,\mathrm{ms}$.
They both get reflected leading to a complicated interaction between the shock and the contact discontinuity at $t\simeq 25.9\,\mathrm{ms}$ and an interaction within the rarefaction fan at $t\smallergtrsim 28.8\,\mathrm{ms}$.
The computation with the \ktScheme{} using $n=401$ volume cells, hence solving a method-of-lines \ode{} system of $401\times 3 = 1203$ equations, takes 24 seconds on a single thread of an \intel{} using our latest \WAM{} code~\cite{Steil:2023PhDFVNB}.
\clearpage

\WarningFilter{latex}{Float too large for page}
\fullWidthFigure%
	{hydro/figures/ee_sodPhysical.pdf}% Graphics
	[fig:ee_sodPhysical_rho,fig:ee_sodPhysical_p,fig:ee_sodPhysical_v,fig:ee_sodPhysical_T,fig:ee_sodPhysical_cs,fig:ee_sodPhysical_epsilon]% Sublabels
	{Numerical solution of the physical shock tube problem \eqref{eq:sodPhysicalICS} for the \eulereqs{} with reflective \bcs{} \eqref{eq:eulerReflectiveBC} from $t=0$ to $t=60\,\mathrm{ms}$.
	For plotting the density $\rho$ in \subref{fig:ee_sodPhysical_rho}, the pressure $p$ in \subref{fig:ee_sodPhysical_p}, and the energy density $\epsilon$ in \subref{fig:ee_sodPhysical_epsilon} we use the canonical \textit{'jet'} color map~\cite{Hunter:2007}.
	For plotting the velocity $v$ in \subref{fig:ee_sodPhysical_v}, the temperature $T$ in \subref{fig:ee_sodPhysical_T}, and the speed of sound $c_s$ in \subref{fig:ee_sodPhysical_cs} we use the \textit{'coolwarm'} color map~\cite{Hunter:2007} diverging non-linearly around the respective initial values.
	The rarefaction fan is marked with \ding{192}, the contact discontinuity with \ding{193}, the shock wave with \ding{194}, and the reflected shock wave with \ding{195}.
	The numerical solution has been computed with the \ktScheme{} using $n=401$ volume cells equidistantly spaced in the interval $x_0=0$ and $x_{400}=20\,\mathrm{m}$, \cf{} subsection 2.5.3 of the auxiliary notebook~\cite{Steil:2023PhDFVNB}.
	}%Caption
	{fig:ee_sodPhysical}%Label
\WarningFilter*{latex}{Float too large for page}
\clearpage

\paragraph{Navier-Stokes equations and gravity}\phantomsection\label{paragraph:NSg}\mbox{} \\
The \eulereqs{} can be extended to include the effects of gravity by adding a simple source term, see, \eg{}, \ccite{Xing2013Feb,Touma2016Sep,QIAN201823,Kappeli2016Mar}.
A further extension to include viscous effects by means of a diffusion term incorporating the effects of thermal conductivity and bulk viscosity\footnote{%
	In this context also referred to as \textit{dynamic} or \textit{absolute} viscosity.%
} leads to a generalization of the \eulereqs{} to the one-dimensional \textit{Navier-Stokes equations}, see, \eg{}, \ccite{Chen2006Jul,Berg2011,Li2017Sep}, which in a constant gravitational field read
\begin{align}
\partial_t\begin{pmatrix}
	\rho\\
	\mu\\
	\epsilon
\end{pmatrix}
+\partial_x
\begin{pmatrix}
 \mu  \\
 p+\rho  v^2 \\
 v (p+\epsilon )
\end{pmatrix}
=\partial_x \begin{pmatrix}
 0  \\
 \lambda \partial_x v\\
 \lambda v\partial_x v+\kappa\partial_x T
\end{pmatrix}+\begin{pmatrix}
 0  \\
 -g \rho\\
 -g \mu
\end{pmatrix}\, ,\label{eq:navierStokes}
\end{align}
including the gravitational acceleration $g$, bulk viscosity coefficient $\lambda$, and thermal conductivity coefficient $\kappa$.
\Cref{eq:navierStokes} is a system of non-linear advection-diffusion-source equations and in this sense similar to the differential equations discussed in the main part of this thesis, \cf{} \cref{chap:zeroONSU2,chap:GN}.
While the Navier-Stokes equations (of course including the practically relevant applications in three dimension) are numerically well under control in the field of \cfd{}, there are still open mathematical questions regarding their structure and solutions \dash{} notably the unsolved \textit{Navier–Stokes existence and smoothness} problem~\cite{NavierStokesMilleniumProblem} stated as one of the seven \textit{Millennium Prize Problems}.
This is the second, \cf{} \cref{chap:introduction}, time we reference one of these problems.
Explicit additional numerical computations with the \ktScheme{} and Euler/Navier-Stokes equations with and without gravitational sources can be found at the end of the auxiliary notebook~\cite{Steil:2023PhDFVNB}.
We will not discuss these here, apart from the note, that the shock tube problems of this subsubsection are completely advection dominated: the relatively low bulk viscosity $\lambda=1.846\cdot 10^{-5} \mathrm{kg}\mkern1mu\mathrm{m}^{-1}\mkern1mu\mathrm{s}^{-1}$ and thermal conductivity $\kappa=2.624\cdot 10^{-2} \mathrm{J}\mkern1mu\mathrm{m}^{-1}\mkern1mu\mathrm{s}^{-1}\mathrm{K}^{-1}$ of air~\cite{dry-air-properties} lead to diffusive contributions which are several orders of magnitude smaller than the advective ones.